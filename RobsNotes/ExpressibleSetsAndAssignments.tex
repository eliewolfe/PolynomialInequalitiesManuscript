\pdfoutput=1
\PassOptionsToPackage{pdftex,
pdfversion=1.7,
pdfencoding=auto,
pdfnewwindow=true,
pdfusetitle=true,
%psdextra=true,
%pdftoolbar=true,
%pdfmenubar=true,
bookmarks=true,
bookmarksnumbered=true,
bookmarksopen=true,
pdfpagemode=UseThumbs,
bookmarksopenlevel=1,
pdfpagelabels=false,
breaklinks=true
}{hyperref}
\PassOptionsToPackage{usenames,dvipsnames,table}{xcolor}
\documentclass[aps,english,superscriptaddress,onecolumn,twoside,longbibliography,pra,floatfix,fleqn,notitlepage,nofootinbib]{revtex4-1}


% packages
\usepackage[OT1]{fontenc}
\usepackage{amsfonts}
\usepackage{amssymb}
\usepackage{amsthm}
\usepackage[intlimits,fleqn]{amsmath}
\usepackage{bm}
\usepackage{graphicx}
%\usepackage{adjustbox}
\usepackage[normalem]{ulem} %for sout
\usepackage{paralist}
\usepackage{microtype}
\usepackage{float}% (not with floatrow)
\usepackage{wrapfig}
\usepackage{array}
\usepackage{ragged2e}%for justifying text in tables
\usepackage{tabularx}
\usepackage{booktabs}
\usepackage{comment}
%\usepackage{showkeys}
\usepackage[intlimits,fleqn]{mathtools} %for mathclap and prescript and more. Learning to love this package. And DeclarePairDelimeter!

% tables stuff
\newcolumntype{R}{>{\raggedleft\arraybackslash}X}
\newcolumntype{C}{>{\centering\arraybackslash}X}
\newcolumntype{L}{>{\raggedright\arraybackslash}X}
\newcolumntype{J}{>{\justifying\arraybackslash}X}
\usepackage{adjustbox}
\usepackage{multirow}
\newcolumntype{T}[2]{%
    >{\adjustbox{angle=#1,lap=\width-(#2)}\bgroup}%
    l%
    <{\egroup}%
}
\setcounter{MaxMatrixCols}{30}

\usepackage{verbatim} %for comment command

% colours and hyperlink stuff
\usepackage[usenames,dvipsnames,table]{xcolor}
\definecolor{purple}{RGB}{128,0,128}
\definecolor{ultramarine}{RGB}{63, 0, 255}
\definecolor{medblue}{RGB}{0, 0, 100}
\definecolor{panblue}{RGB}{0,24,150}
\definecolor{carmine}{RGB}{150, 0, 24}
\definecolor{gray}{RGB}{150, 150, 150}
\usepackage[pdftex,pdfversion=1.7,pdfencoding=auto,pdfnewwindow=true,pdfusetitle=true,bookmarks=true,bookmarksnumbered=true,bookmarksopen=true,pdfpagemode=UseThumbs,bookmarksopenlevel=1,pdfpagelabels=false,breaklinks=true]{hyperref}
\hypersetup{colorlinks,
linkcolor=carmine,
citecolor=medblue,
urlcolor=panblue,
anchorcolor=OliveGreen}



% coloured text
\newcommand*{\mred}[1]{{\color{RawSienna}{\mathbf{#1}}}}
\newcommand*{\mgreen}[1]{{\color{OliveGreen}{\mathbf{#1}}}}
\newcommand*{\tred}[1]{{\color{carmine}{\textbf{#1}}}}
\newcommand*{\tblue}[1]{{\color{MidnightBlue}{\textbf{#1}}}}
\newcommand*{\tpurp}[1]{{\color{Plum}{{#1}}}}

% cleveref stuff
\usepackage[capitalise]{cleveref}
\Crefname{eqs}{Eqs.}{Eqs.}
\creflabelformat{eqs}{(#2#1#3)}
\crefrangelabelformat{equation}{(#3#1#4-#5#2#6)}
\Crefmultiformat{equation}{Eqs.~(#2#1#3}{,#2#1#3)}{,#2#1#3}{,#2#1#3)}
\crefrangelabelformat{eqs}{(#3#1#4-#5#2#6)}
\Crefmultiformat{eqs}{Eqs.~(#2#1#3}{,#2#1#3)}{,#2#1#3}{,#2#1#3)}
\Crefname{example}{Example}{Examples}
\Crefname{section}{Sec.}{Secs.}



% theorem environments
\newtheorem{theorem}{Theorem}
\newtheorem{lemma}[theorem]{Lemma}
\newtheorem{claim}[theorem]{Claim}
\newtheorem{conjecture}[theorem]{Conjecture}
\newtheorem{corollary}[theorem]{Corollary}
\newtheorem{definition}[theorem]{Definition}
\theoremstyle{definition}
%\newtheorem{example}{Example}
\newtheorem{exercise}{Exercise}
\newtheorem{notation}[theorem]{Notation}
\newtheorem{problem}[theorem]{Problem}
\newtheorem{remark}[theorem]{Remark}

\newcounter{example}[section]
\newenvironment{example}[1][]{\refstepcounter{example}\par\medskip
   \noindent \textbf{Example~\theexample}\hspace{1em}\rmfamily#1}{\par\medskip\par}
\Crefname{example}{Example}{Examples}
\creflabelformat{example}{#2#1#3}
\crefrangelabelformat{example}{#3#1#4-#5#2#6}
\Crefmultiformat{example}{Examples~#2#1#3}{, #2#1#3}{, #2#1#3 }{and #2#1#3}
\renewcommand{\theexample}{\arabic{example}}

% macros for our notation
\newcommand{\p}[2][]{{P_{#1}}\parenths{#2}}
\newcommand{\pfunc}[1]{P_{#1}}
\newcommand{\An}[2][]{{\mathsf{An}_{#1}}\parenths{#2}}
\newcommand{\Pa}[2][]{{\mathsf{Pa}_{#1}}\parenths{#2}}
\newcommand{\Ch}[2][]{{\mathsf{Ch}_{#1}}\parenths{#2}}
\newcommand{\SmallNamedFunction}[3][]{\operatorname{\mathsf{#2}}_{#1}\parenths{#3}}
\newcommand{\subgraph}[2][]{\SmallNamedFunction[#1]{SubDAG}{#2}}
\newcommand{\ansubgraph}[2][]{\SmallNamedFunction[#1]{AnSubDAG}{#2}}
\newcommand{\nodes}[1]{\SmallNamedFunction{Nodes}{#1}}
\newcommand{\obsnodes}[1]{\SmallNamedFunction{ObservedNodes}{#1}}
\newcommand{\latnodes}[1]{\SmallNamedFunction{LatentNodes}{#1}}
\newcommand{\inflations}[1]{\SmallNamedFunction{Inflations}{#1}}
\newcommand{\DAG}[1]{\SmallNamedFunction{DAG}{#1}}
\newcommand{\edges}[1]{\SmallNamedFunction{Edges}{#1}}
\newcommand{\aindep}{\perp} % for d-separation
\newcommand{\indep}{\perp\!\!\!\!\perp} % (conditional) independence
\newcommand{\cramp}[1]{\ensuremath{\mathord{#1}}}
\newcommand{\eql}{\cramp{=}}
\DeclarePairedDelimiter{\parens}{\lparen}{\rparen}
\DeclarePairedDelimiter{\parenths}{\lparen}{\rparen}
\DeclarePairedDelimiter{\braces}{\lbrace}{\rbrace}
\DeclarePairedDelimiter{\bracks}{\lbrack}{\rbrack}
\DeclarePairedDelimiter{\expec}{\langle}{\rangle}
\newcommand{\brackets}[1]{\braces*{#1}}

% more vertical spacing in multiline equations
\setlength{\jot}{6pt}





\begin{document}

\title{Expressible sets and expressible assignments}
\author{RWS}
\date{Nov. 20, 2016}                                           % Activate to display a given date or no date
\maketitle
%\tableofcontents

%\section{}
%\subsection{}

\section{Expressible sets}

\begin{example}[\tred{Incompatibility of Pienaar distribution with DAG \#16}]
\label{example:Pienaar}

Consider the DAG of \cref{fig:GDAG16}.  Henson, Lal and Pusey showed that this DAG is a candidate for being `interesting', that is, the compatible distributions satisfy constraints over and above the conditional independence relations that follow from d-separation relations in the DAG. 

% i.e. for which the set of distributions that are compatible with it is a strict subset of those with conditional independence relations that are implied by the d-separation criterion. 

% Pienaar subsequently showed that this was the case by demostrating that the following distribution is incompatible with the DAG:

\begin{figure}[htb]
\centering
\begin{minipage}[t]{0.4\linewidth}
\centering
\includegraphics[scale=1]{scen16DAG.pdf}
\caption{DAG \#16 in Ref.~\cite{pusey2014gdag}.}\label{fig:GDAG16}
\end{minipage}
\hfill
\begin{minipage}[t]{0.5\linewidth}
\centering
\includegraphics[scale=1]{scen16InflationDAG.pdf}
\caption{The Rocket inflation of \cref{fig:GDAG15}.}\label{fig:Inflated15}
\end{minipage}
\end{figure}

\citet{pianaar2016interesting} identified a distribution which satisfies the CI relations among the observed variables in DAG \#16, namely, $Y\indep C$ and $A\indep B | Y$~\cite{pusey2014gdag}, but is nonetheless incompatible with it:
\begin{align}\label{eq:pienaardistro}
    P^{\text{Pien}}_{A B C Y}:=\frac{[0000]+[0110]+[0001]+[1011]}{4},\quad\text{i.e.,}\quad P^{\text{Pien}}_{Y\! A B C}(y a b c):=\begin{cases}\tfrac{1}{4}&\text{if }  y\cdot c = a \text{ and }  (y \oplus 1)\cdot c = b, \\ 0&\text{otherwise}.\end{cases}
\end{align}
Note that we can rewrite Eq.~\eqref{eq:pienaardistro} as
\begin{align}
P^{\text{Pien}}_{A B C Y}= \frac{1}{2}([00]_{BC}+[11]_{BC})[0]_A [0]_Y + \frac{1}{2}([00]_{AC}+[11]_{AC}) [0]_B [1]_Y,
\end{align}
which makes it evident that the distribution can be described as follows: if $Y=0$, then $B$ and $C$ are in a maximally correlated state and $A=0$, while if $Y=1$, then $A$ and $C$ are maximally correlated and $B=0$.

Here, we will establish this incompatibility using the inflation technique.  To do so, we use the inflation of DAG \#16 depicted in \cref{fig:Inflated15}.  
%The expressible sets include $\{ A_1 B_1 C_2 Y_1 Y_2\}, \{A_2 B_2 C_1 Y_1 Y_2\}, \{A_2 B_2 C_2 Y_1 Y_2\}$ and $\{A_1 B_1 C_1 Y_1 Y_2\}$. 
To do so, we will make use of the fact that $\{ B_2 C_2 Y_2\}, \{ A_1 C_1 Y_1\}$ and $\{B_2 C_1 Y_2 \}$ are injectable sets, together with the fact that $\{ A_1 C_2 Y_1\}$ is an expressible set. 

We begin by demonstrating how the $d$-separation relations in the inflation imply that $\{ A_1 C_2 Y_1\}$ is expressible.  The expressibility of $\{ A_1 C_2 Y_1\}$ follows from the expressibility of $\{ A_1 B_1 C_2 Y_1\}$ and the fact that the distribution on the former can be obtained from the distribution on the latter by marginalization.  $\{ A_1 B_1 C_2 Y_1\}$ is expressible because the d-separation relation $A_1 \perp C_2 | B_1 Y_1$ implies that
\begin{align}
P_{A_1 B_1 C_2 Y_1} = \frac{P_{A_1 B_1 Y_1} P_{C_2 B_1 Y_1} }{P_{B_1 Y_1}},
\end{align}
and each of the sets $\{A_1 B_1 Y_1\}, \{C_2 B_1 Y_1\},$ and $\{B_1 Y_1\}$ are injectable.  
We therefore have 
\begin{align}
%P_{A_1 C_2 Y_1}(a_1 c_2 y_1) = \sum_{b_1} \frac{P_{A_1 B_1 Y_1}(a_1 b_1 y_1) P_{C_2 B_1 Y_1}(c_2 b_1 y_1) }{P_{B_1 Y_1}(b_1 y_1)},
P_{A_1 C_2 Y_1}(a c y) = \sum_{b} \frac{P^{\text{Pien}}_{A B Y}(a b y) P^{\text{Pien}}_{C B Y}(c b y) }{P^{\text{Pien}}_{B Y}(b y)},
\label{express1}
\end{align}

From the injectability of $\{ B_2 C_2 Y_2\}, \{ A_1 C_1 Y_1\}$ and $\{B_2 C_1 Y_2 \}$, we can infer that 
\begin{align}
%P_{B_2 C_2 |Y_2}(\cdot \cdot| 0)&= P^{\text{Pien}}_{BC|Y}(\cdot \cdot| 0) \nonumber\\
%P_{A_1 C_1  |Y_1}(\cdot \cdot| 1)&= P^{\text{Pien}}_{AC|Y}(\cdot \cdot| 1)\nonumber\\
%P_{B_2 C_1 |Y_2}(\cdot \cdot| 0)&= P^{\text{Pien}}_{BC|Y}(\cdot \cdot| 0) 
P_{B_2 C_2 |Y_2}(bc| y)&= P^{\text{Pien}}_{BC|Y}(bc|y) \nonumber\\
P_{A_1 C_1  |Y_1}(ac|y)&= P^{\text{Pien}}_{AC|Y}(ac|y)\nonumber\\
P_{B_2 C_1 |Y_2}(bc|y)&= P^{\text{Pien}}_{BC|Y}(bc|y) 
\end{align} 
which implies that
\begin{align}
P_{B_2 C_2 |Y_2}(\cdot \cdot| 0)&= \frac{1}{2}([00]_{B_2 C_2}+[11]_{B_2 C_2})\label{inj1}\\
P_{A_1 C_1  |Y_1}(\cdot \cdot| 1)&= \frac{1}{2}([00]_{A_1 C_1}+[11]_{A_1 C_1}) \label{inj2}\\
P_{B_2 C_1 |Y_2}(\cdot \cdot| 0)&= \frac{1}{2}([00]_{B_2 C_1}+[11]_{B_2 C_1})\label{inj3}
\end{align} 

For the expressible set $\{ A_1 C_2 Y_1\}$, Eq.~\eqref{express1} implies that
\begin{align}
%P_{A_1 C_2 Y_1}(a_1 c_2 y_1) = \sum_{b_1} \frac{P_{A_1 B_1 Y_1}(a_1 b_1 y_1) P_{C_2 B_1 Y_1}(c_2 b_1 y_1) }{P_{B_1 Y_1}(b_1 y_1)},
P_{A_1 C_2 |Y_1}(a c |y) &= \sum_{b} \frac{P^{\text{Pien}}_{A B Y}(a b y) P^{\text{Pien}}_{C B Y}(c b y) }{P^{\text{Pien}}_{B Y}(b y) P^{\text{Pien}}_{Y}(y)}\nonumber\\
&=\sum_{b} \frac{P^{\text{Pien}}_{A B |Y}(a b| y) P^{\text{Pien}}_{C B| Y}(c b |y) }{P^{\text{Pien}}_{B |Y}(b |y)},\label{A1C2Y1}
\end{align}
where we have simply used the definition of conditioning.  
%It follows that
%\begin{align}
%P_{A_1 C_2 |Y_1}(a c |1) =\sum_{b} \frac{P^{\text{Pien}}_{A B |Y}(a b| y) P^{\text{Pien}}_{C B| Y}(c b |y) }{P^{\text{Pien}}_{B |Y}(b |y)},
%\end{align}
%It will be useful in what follows to consider the joint probability for $A_1=0, C_2=0$ given $Y_1=1$
%\begin{align}
%P_{A_1 C_2 |Y_1}(00 |1) &=\sum_{b} \frac{P^{\text{Pien}}_{A B |Y}(0 b| 1) P^{\text{Pien}}_{C B| Y}(0 b |1) }{P^{\text{Pien}}_{B |Y}(b |1)}\nonumber\\
%&= \frac{P^{\text{Pien}}_{A B |Y}(0 0| 1) P^{\text{Pien}}_{C B| Y}(0 0 |1) }{P^{\text{Pien}}_{B |Y}(0 |1)}
%+  \frac{P^{\text{Pien}}_{A B |Y}(0 1| 1) P^{\text{Pien}}_{C B| Y}(0 1 |1) }{P^{\text{Pien}}_{B |Y}(1 |1)}\nonumber\\
%&= \frac{1}{4}\label{A1C2Y1}
%\end{align}

Now suppose that $Y_2=0$ and $Y_1=1$.  From Eq.~\eqref{inj3}, we infer that 
\begin{align}
\text{With probability 1/2,}\; B_2=0\; \text{and}\;C_1=0.
\end{align}
From Eq.~\eqref{inj1}, we infer that
\begin{align}
\text{if}\; B_2=0\; \text{then}\;C_2=0.
\end{align}
From Eq.~\eqref{inj2}, we infer that 
\begin{align}
\text{if}\; C_1=0\; \text{then}\;A_1=0.
\end{align}
These three results imply that 
\begin{align}
\text{The probability}\; p\; \text{that }\; C_2=0\; \text{and}\;A_1=0 \;\text{must be} \; \ge 1/2.
\end{align}
However, from Eq.~\eqref{A1C2Y1}, we infer that the probability of $C_2=0\; \text{and}\;A_1=0$ is only $p=1/4$.   
Explicitly, 
\begin{align}
P_{A_1 C_2 |Y_1}(00 |1) &=\sum_{b} \frac{P^{\text{Pien}}_{A B |Y}(0 b| 1) P^{\text{Pien}}_{C B| Y}(0 b |1) }{P^{\text{Pien}}_{B |Y}(b |1)}\nonumber\\
&= \frac{P^{\text{Pien}}_{A B |Y}(0 0| 1) P^{\text{Pien}}_{C B| Y}(0 0 |1) }{P^{\text{Pien}}_{B |Y}(0 |1)}
+  \frac{P^{\text{Pien}}_{A B |Y}(0 1| 1) P^{\text{Pien}}_{C B| Y}(0 1 |1) }{P^{\text{Pien}}_{B |Y}(1 |1)}\nonumber\\
&= \frac{1}{4}\label{A1C2Y1}
\end{align}
We have therefore arrived at a contradiction.  This establishes the incompatibility of the Pienaar distribution with DAG \#16.

\end{example}

\begin{comment}
 
We will here make use of the fact that $\{ A_1 B_1 C_2 Y_1 Y_2\}$ is an expressible set.  To demonstrate that this is the case, we make use of the d-separation relation $A_1 \perp C_2 \perp Y_2 | B_1 Y_1$ in the inflation to infer that
\begin{align}
P_{A_1 B_1 C_2 Y_1 Y_2} = \frac{P_{A_1 B_1 Y_1} P_{C_2 B_1 Y_1} P_{Y_2 B_1 Y_1}}{P_{B_1 Y_1}^2}.
\end{align}
We then use the d-separation relation $Y_2 \perp B_1 Y_1$ to decompose the final term of the numerator as
\begin{align}
P_{Y_2 B_1 Y_1} = P_{Y_2} P_{B_1 Y_1},
\end{align}
so that 
\begin{align}
P_{A_1 B_1 C_2 Y_1 Y_2} = \frac{P_{A_1 B_1 Y_1} P_{C_2 B_1 Y_1} P_{Y_2} }{P_{B_1 Y_1}}.
\end{align}
Because each of the sets $\{A_1 B_1 Y_1\}, \{C_2 B_1 Y_1\}, \{B_1 Y_1\}, \{ Y_2\}$ are injectable, it follows that the set $\{ A_1 B_1 C_2 Y_1 Y_2\}$ is expressible.  
Note that $\{ A_1 B_1 C_2 Y_1 Y_2\}$ is not preinjectable because we have used more than ancestral independences to prove its expressibility.
% Note that it is not preinjectable because $A_1$ has $X_1$ as ancestor, while $C_2$ has $X_2$ as ancestor [say more here].

We consider the case where $Y_2=0$ and $Y_1=1$.  

The injectable sets include $\brackets{A_1 B_1 C_1 Y_1}$ and $\brackets{A_2 B_1 C_2 Y_1}$.  They share the same  image on the original DAG, namely, the set of all observed variables, $\brackets{ABCY}$. It follows that
\begin{align}
P_{A_1 B_1 C_1 Y_1} = P_{A_2 B_1 C_2 Y_1} =P^{\text{Pien}}_{ABCY}.
\label{bridge1}
\end{align}

If $P^{\text{Pien}}_{ABCY}$ is compatible with the given inflation of DAG \#16, then by Lemma ?, $P_{A_1 B_1 C_1 Y_1}$ and $ P_{A_2 B_1 C_2 Y_1}$ are compatible with this inflation, and conversely if we can show incompatibility of the latter, we establish incompatibility of the former.  To show that $P^{\text{Pien}}_{ABCY}$ is indeed incompatible with the inflation of DAG \#16, we assume compatibility and derive a contradiction.

%With a slight rewriting of Eq.~\ref{eq:pienaardistro}, we have
Note first that we can rewrite Eq.~\eqref{eq:pienaardistro} as
\begin{align}
P^{\text{Pien}}_{A B C Y}= \frac{1}{2}([00]_{BC}+[11]_{BC})[0]_A [0]_Y + \frac{1}{2}([00]_{AC}+[11]_{AC}) [0]_B [1]_Y
\end{align}
It then follows from Eq.~\eqref{bridge1} that
\begin{align}
P_{A_1 C_1 | Y_1=1} = \frac{1}{2}([00]_{A_1 C_1}+[11]_{A_1 C_1}),\label{A1C1Y1e1}\\
P_{A_2 C_2 |Y_1=1} = \frac{1}{2}([00]_{A_2 C_2}+[11]_{A_2 C_2}).\label{A2C2Y1e1}
\end{align}
and that 
\begin{align}
P_{B_1 C_1 | Y_1=0} = \frac{1}{2}([00]_{B_1 C_1}+[11]_{B_1 C_1}),\label{B1C1Y1e0}\\
P_{B_1 C_2 |Y_1=0} = \frac{1}{2}([00]_{B_1 C_2}+[11]_{B_1 C_2}).\label{B1C2Y1e0}
\end{align}

Any distribution $P_{B_1 C_1 C_2 Y_1}$ which has marginals $P_{B_1 C_1 Y_1}$ and $P_{B_1 C_2 Y_1}$ that reproduce the conditional distributions of \cref{B1C1Y1e0} and \cref{B1C2Y1e0} respectively, must be such that 
\begin{align}
P_{B_1 C_1 C_2 | Y_1=0} = \frac{1}{2}([000]_{B_1 C_1 C_2}+[111]_{B_1 C_1 C_2})\label{B1C1C2Y1e0}.
\end{align}
The reason is that if $B_1$ and $C_1$ are perfectly correlated, as in \cref{B1C1Y1e0}, and $B_1$ and $C_2$ are perfectly correlated, as in \cref{B1C2Y1e0}, then $C_1$ and $C_2$ must be perfectly correlated as well.  \cref{B1C1C2Y1e0} ensures this: marginalizing this distribution over $B_1$, we obtain
\begin{align}
P_{C_1 C_2 | Y_1=0} = \frac{1}{2}([00]_{ C_1 C_2}+[11]_{C_1 C_2}).\label{C1C2Y1e0}
\end{align}

But the given inflation of DAG \#16 is such that $C_1 C_2$ and $Y_1$ are ancestrally independent, so that 
\begin{align}
P_{C_1 C_2 | Y_1} =P_{C_1 C_2},
\end{align}
and therefore $P_{C_1 C_2 | Y_1=0}=P_{C_1 C_2 | Y_1=1}$, so that we can infer from \cref{C1C2Y1e0} that
\begin{align}
P_{C_1 C_2 | Y_1=1} = \frac{1}{2}([00]_{ C_1 C_2}+[11]_{C_1 C_2})\label{C1C2Y1e1}.
\end{align}

Finally, we note that any distribution $P_{A_1 A_2 C_1 C_2 Y_1}$ which has marginals $P_{A_1 C_1 Y_1}$, $P_{A_2 C_2 Y_1}$, and $P_{C_1 C_2 Y_1}$ that reproduce the conditional distributions of \cref{A1C1Y1e1}, \cref{A2C2Y1e1} and \cref{C1C2Y1e1} respectively must be such that 
\begin{align}
P_{A_1 A_2 C_1 C_2 | Y_1=1} = \frac{1}{2}([0000]_{A_1 A_2 C_1 C_2}+[1111]_{A_1 A_2 C_1 C_2})\label{A1A2C1C2Y1e1}.
\end{align}
The reason is that if $A_1$ and $C_1$ are perfectly correlated, as in \cref{A1C1Y1e1}, and $A_2$ and $C_2$ are perfectly correlated, as in \cref{A2C2Y1e1}, and $C_1$ and $C_2$ are perfectly correlated, as in \cref{C1C2Y1e1}, then $A_1$ and $A_2$ must be perfectly correlated as well. 

Marginalizing \cref{A1A2C1C2Y1e1} over $C_1 C_2$, we obtain
\begin{align}
P_{A_1 A_2  | Y_1=1} = \frac{1}{2}([00]_{A_1 A_2}+[11]_{A_1 A_2})\label{A1A2Y1e1}.
\end{align}

Finally, we note that in the Parachute inflation of the Modified Triangle Scenario, $A_1$ is d-separated from $A_2$ given $Y_1$, which implies that $P_{A_1 A_2  | Y_1}=P_{A_1 | Y_1}P_{A_2  | Y_1}$.  This is inconsistent with \cref{A1A2Y1e1}, so we have derived a contradiction.

\color{black}
\end{comment}

\section{Instrumental inequality via expressible assignments in the Bell scenario}


Consider the instrumental scenario of Fig.~\ref{fig:instrumental}.  
\begin{figure}[htb]
\centering
\begin{minipage}[h!]{0.4\linewidth}
\centering
\includegraphics[scale=0.1]{Instrumental.jpg}
\caption{The instrumental scenario.}\label{fig:instrumental}
\end{minipage}
\hfill
\begin{minipage}[htb]{0.5\linewidth}
\centering
\includegraphics[scale=0.3]{InterruptedInstrumental.jpg}
\caption{The Bell scenario.}\label{fig:InterruptedInstrumental}
\end{minipage}
\end{figure}

Pearl has found causal compatibility inequalities that apply to it.  These are termed the instrumental inequalities.  If the observed variables are binary, then they have the following form:
\begin{align}
P_{XY|Z}(00|0) +  P_{XY|Z}(00|0) \le 1,\nonumber\\
P_{XY|Z}(10|0) +  P_{XY|Z}(11|1) \le 1,\nonumber\\
P_{XY|Z}(01|0) +  P_{XY|Z}(00|1) \le 1,\nonumber\\
P_{XY|Z}(11|0) +  P_{XY|Z}(10|1) \le 1. \label{IIlongform}
\end{align}
This can be summarized as 
\begin{align}
\forall x: \sum_y P_{XY|Z}(xy|y) \le 1,\label{IIbinary1}\\
\forall x: \sum_y P_{XY|Z}(x (y\oplus 1)|y) \le 1,\label{IIbinary2}
\end{align}

If we label the two pairs of cause-effect-related observed variables in the Bell scenario by $X_2$, $Y_2$ and $Z_1$, $X_1$ respectively, then we can think of the Bell scenario as supporting a quasi-inflation of the instrumental scenario: for every causal model in the instrumental scenario, we can define a causal model in the Bell scenario where every variable except $X_2$ depends on its parents in exactly the manner that the corresponding variable (i.e., the one where the index is dropped) did in the Instrumental scenario.  In this mapping, $X_2$ is presumed to be a root variable that is distributed in the same manner as $X$ is in the Instrumental scenario.  

We then note that although the set $\{ Y_2 Z_1 X_1 X_2\}$ is not an expressible set, assignments of the form $P_{Y_2 Z_1 |X_1 X_2} (yz|xx)$, where $X_1$ and $X_2$ take the same value, {\em are} expressible, in the sense that
\begin{align}
P_{Y_2 Z_1 |X_1 X_2} (yz|xx)= P_{YZ|X}(yz|x).\label{exass}
\end{align}
This equality follows from considering the consequences of conditioning on $X$ in the Instrumental Scenario.

Eq.~\eqref{exass}  in turn implies that
\begin{align}
P_{XY|Z}(xy|z) = P_{Y_2 X_1 |X_2 Z_2} (yx|xz).\label{modifiedexass}
\end{align}
The proof is as follows.  One notes that 
\begin{align}
P_{YZ|X} =\frac{P_{XY|Z} P_Z }{P_X}
\end{align}
and that
\begin{align}
P_{Y_2 Z_1 |X_1 X_2} &= \frac{P_{Y_2 X_1 |X_2 Z_2} P_{Z_2}}{P_{X_1|Z_2}}\nonumber\\
&= \frac{P_{Y_2 X_1 |X_2 Z_2} P_{Z_2}}{P_{X_1}}
\end{align}
where the second equality follows from the fact that $X_1 \perp X_2$ in the Bell scenario.  It is then sufficent to note that $\{X_1\}$ and $\{ Z_1\}$ are injectable in order to complete the proof.

We will shortly demonstrate how the Bell scenario implies the following causal compatibility inequalities:
\begin{align}
\sum_y P_{Y_2 X_1 |X_2 Z_2} (yx|xy) \le 1\label{BGAI}\\
\sum_y P_{Y_2 X_1 |X_2 Z_2} ((y\oplus 1) x|xy) \le 1\label{BGNAI}\\
\end{align}
Combining these with Eq.~\eqref{modifiedexass}, we obtain the Instrumental inequalities of \eqref{IIbinary1} and \eqref{IIbinary1}.  

We now show how these causal compatibility inequalities in the Bell scenario are instances of bounds on the performance of a distributed guessing game.

\subsection{Distributed guessing games}

We recall the definition of a distributed guessing game from Ref.~\cite{{localorthog}}
\begin{quote}
[A distributed guessing game is]  a non-local game in which a referee has access to a set of vectors of $n$ symbols with values in $\{0,...,d-1\}$. Denote this set by $S$ and by $|S|$ its size, which can be less than $d^n$ in general. Now, the referee chooses a vector $(\tilde{a}_1,\dots,\tilde{a}_n)$ uniformly at random from $S$, and encodes it into a new vector of, again, $n$ symbols using a function $f$. However, the new symbols can now take $m$ values and, thus, $f : S\to \{0, . . . , m-1\}^n$. The resulting vector is $(x_1,...,x_n) = f(\tilde{a}_1,...,\tilde{a}_n)$. These $n$ symbols are distributed among $n$ distant players who cannot communicate and must produce individual guesses $a_1, . . . , a_n$. Their goal is to guess the initial input to the function, that is, they win whenever $a_j = \tilde{a}_j$ for all $j$. Note that the encoding function $f$ and the set $S$ are known in advance to all the players.
\end{quote}

The game known as ``Guess your neighbour's input'', abbreviated GYNI, is an instance~\cite{GYNI}.  We here make use of a simplified version of GYNI wherein one of the inputs is fixed, so that only one of the player's guessing tasks is nontrivial.  

%We here make use of a two-party  distributed guessing game where only one party must guess her neighbour's input.  We call the game ``Bob must guess Alice's input'' (BGAI).  

The probability of succes in such a game corresponds to a probability of achieving particular outcomes in the Bell scenario, where the hidden common cause can be considered the strategy of the two players.  
%We imagine a causal model wherein
The game is defined as follows; Alice's binary setting, $Z_2$, is chosen uniformly at random.  Bob's binary setting, $X_2$ is fixed with value $x$. Thus $S= \{ (0,x),(1,x)\}$, and the input pair is chosen uniformly from this set.  The probability of success in the game is clearly 
\begin{align}
P_{\rm succ} &= \sum_{y=0}^{1}P_{Y_2 X_1 |X_2 Z_2} (yx|xy) P(y)\nonumber\\
&=\frac{1}{2} \sum_{y=0}^{1} P_{Y_2 X_1 |X_2 Z_2} (yx|xy),
\end{align}
It follows that to derive Eq.~\eqref{BGAI}, it suffices to demonstrate that 
\begin{align}
P_{\rm succ} \le \frac{1}{2}.
\end{align}
This inequality has, in fact, a highly intuitive proof.  Given that Bob's setting is fixed to be $x$, the optimal strategy involves Alice outputing $x$ with probability 1.   However, because there is no causal influence from Alice's setting variable to Bob's outcome variable, any classical strategy can do no better than the one wherein Bob simpy guesses Alice's input, in which case the probability of guessing correctly is $\frac{1}{2}$.

%Because we are assuming the settings and outcomes of our Bell scenario to be binary, our game 

Note that if one modifies the game to one wherein Bob must guess the {\em negation} of Alice's input, the probability of success is still bounded above by $\frac{1}{2}$, and we derive Eq.~\eqref{BGNAI}. 


\begin{comment}
\section{Failed attempt to understand Instrumental inequality via inflation}


\subsection{Alternate form of the inequality}

Consider  the instrumental inequalities for binary variables:
\begin{align}
P_{XY|Z}(00|0) +  P_{XY|Z}(00|0) \le 1,\nonumber\\
P_{XY|Z}(10|0) +  P_{XY|Z}(11|1) \le 1,\nonumber\\
P_{XY|Z}(01|0) +  P_{XY|Z}(00|1) \le 1,\nonumber\\
P_{XY|Z}(11|0) +  P_{XY|Z}(10|1) \le 1. \label{II1}
\end{align}
It turns out that these can be obtained as special cases of the following set of causal compatibility inequalities for the instrumental scenario:
\begin{align}
P_{XY|Z} \le P_Y.
\label{II2}
\end{align}
\color{red}
THIS IS WRONG. As Elie has noted, the latter inequality cannot be a valid causal compatibility inequality because of the following example: if $X=Z$, and $Y=X$ and $Z$ is uniformly distribution, then we have $P_Y(0)=P_Y(1)=1/2$, but $P_{XY|Z}(00|0) = 1$. 
\color{black}


To see that Eq.~\eqref{II2} implies \eqref{II1}, note that it implies 
\begin{align}
\forall x \forall z \forall y:  P_{XY|Z}(xy|z) \le P_Y(y),
\end{align}
which in turn implies that
\begin{align}
\forall x  \forall y:  P_{XY|Z}(xy|y) \le P_Y(y),
\end{align}
and 
\begin{align}
\forall x  \forall y:  P_{XY|Z}(xy|y\oplus 1) \le P_Y(y),
\end{align}
Summing over $y$ in each case, we obtain:
\begin{align}
\forall x : \sum_y  P_{XY|Z}(xy|y) \le 1,
\end{align}
and 
\begin{align}
\forall x : \sum_y  P_{XY|Z}(xy|y\oplus 1) \le 1,
\end{align}
respectively.  
These are the  instrumental inequalities for binary variables.  

%I have to think about whether the two conditions are in fact equivalent for binary variables.

The usual way of representing the instrumental inequality is as follows:
\begin{align}
\max_x \sum_y \max_z P(xy|z) \le 1.
\label{IIb1}
\end{align}
It too can be obtained as a special case of Eq.~\eqref{II2}.

To see that Eq.~\eqref{II2} implies \eqref{IIb1}, note that it implies 
\begin{align}
\forall x \forall y \forall z: P_{YX|Z}(yx|z) \le P_Y (y),
\end{align}
which in turn implies 
\begin{align}
\forall x \forall y :\max_z P_{YX|Z}(yx|z) \le P_Y (y),
\end{align}
and therefore
\begin{align}
\forall x : \sum_y \max_z P_{YX|Z}(yx|z) \le 1,
\end{align}
which entails finally 
\begin{align}
\max_x \sum_y \max_z P_{YX|Z}(yx|z) \le 1,
\end{align}
which is the standard form of the instrumental inequality.

\subsection{Deriving the instrumental inequality using inflation technique and {\em expressible assignments}}



Consider the instrumental scenario, depicted in Fig.~\ref{fig:instrumental}, and the inflation thereof depicted in Fig. \ref{fig:InflatedInstrumental}.
\begin{figure}[htb]
\centering
\begin{minipage}[h!]{0.4\linewidth}
\centering
\includegraphics[scale=0.1]{Instrumental.jpg}
\caption{The instrumental scenario.}\label{fig:instrumental}
\end{minipage}
\hfill
\begin{minipage}[htb]{0.5\linewidth}
\centering
\includegraphics[scale=0.1]{InstrumentalInflation.jpg}
\caption{The Greyhound inflation of the instrumental scenario.}\label{fig:InflatedInstrumental}
\end{minipage}
\end{figure}

\begin{figure}[h!]
\centering
\begin{minipage}[t]{0.4\linewidth}
\centering
\includegraphics[scale=1]{InstrumentalDAGpearl.pdf}
\caption{The instrumental scenario.}\label{fig:instrumental}
\end{minipage}
\hfill
\begin{minipage}[htb]{0.5\linewidth}
\centering
\includegraphics[scale=1]{instrumentalvariant.pdf}
\caption{The Greyhound inflation of the instrumental scenario, with the opposite labelling convention to the one I use below.}\label{fig:InflatedInstrumental}
\end{minipage}
\end{figure}

It is straightforward to verify that 
\begin{align}
P_{Y_1 Z_2 |X_2 X_1} ( \cdot \cdot | xx) = P_{Y Z |X} ( \cdot \cdot | x).
\label{exass1}
\end{align}
In this case, we say that $P_{Y_1 Z_2 |X_2 X_1}$ is an {\em expressible assignment}.
 
 Note that Eq.~\eqref{exass1} implies that 
  \begin{align}
\frac{P_{Y_1 Z_2 X_2 X_1} ( \cdot \cdot xx)}{P_{X_1}(x)P_{X_2}(x)} = \frac{P_{Y Z X} ( \cdot \cdot x)}{P_{X}(x)},
\end{align}
and because $X_1$ and $X_2$ are injectable, we infer that 
 \begin{align}
P_{Y_1 Z_2 X_2 X_1} ( \cdot \cdot xx) = P_{Y Z X} ( \cdot \cdot x),
\label{exass2}
\end{align}
so $P_{Y_1 Z_2 X_2 X_1} ( \cdot \cdot xx)$ is also an expressible assignment. 

\color{red}
THE ABOVE IS INCORRECT. 
Eq.~\eqref{exass1} implies that 
  \begin{align}
\frac{P_{Y_1 Z_2 X_2 X_1} ( \cdot \cdot xx)}{P_{X_1 X_2}(xx)} = \frac{P_{Y Z X} ( \cdot \cdot x)}{P_{X}(x)}. 
\end{align}
Because $\{X_1 X_2 \}$ is not an injectable set, neither is  $P_{Y_1 Z_2 X_2 X_1} ( \cdot \cdot xx)$ an expressible assignment. 

IT FOLLOWS THAT THE REST OF THE PROOF DOES NOT GO THROUGH
\color{black}

The d-separation relation $Y_1 \perp Z_2 $ in the inflation implies that 
\begin{align}
\sum_{X_1,X_2}P_{Y_1 Z_2 X_2 X_1}  = P_{Y_1} P_{Z_2}
%\sum_{x_1,x_2}P_{Y_1 Z_2 X_2 X_1} ( \cdot \cdot x_1 x_2)
\end{align}
This can be rewritten as
\begin{align}
\forall  y_1 \forall z_2 : \sum_{x}P_{Y_1 Z_2 X_2 X_1} ( y_1 z_2 x x) + \sum_{x_1 \ne x_2}P_{Y_1 Z_2 X_2 X_1} ( \cdot \cdot x_1 x_2)  \le P_{Y_1}(y_1) P_{Z_2}(z_2)
\end{align}
which implies that
\begin{align}
\forall  y_1 \forall z_2 : \sum_{x}P_{Y_1 Z_2 X_2 X_1} ( y_1 z_2 x x)   \le P_{Y_1}(y_1) P_{Z_2}(z_2).
\end{align}
Given that every term in the sum on the LHS is positive, the inequality holds for each such term,
\begin{align}
\forall  y_1 \forall z_2 \forall x: P_{Y_1 Z_2 X_2 X_1} ( y_1 z_2 x x)   \le P_{Y_1}(y_1) P_{Z_2}(z_2).
\end{align}
Conditioning on $Z_2$, we obtain
\begin{align}
\forall  y_1 \forall z_2 \forall x: P_{Y_1 X_2 X_1|Z_2 } ( y_1  x x|z_2 )   \le P_{Y_1}(y_1).
\end{align}

Finally, given that the singleton set $\{Y_1\}$ is an injectable set and given that $P_{Y_1 X_2 X_1|Z_2 } ( y_1  x x|z_2 ) $ is an expressible assignment (described in Eq.~\eqref{exass2}),  we conclude that
\begin{align}
\forall  y \forall z  \forall x: P_{Y X | Z } ( y x|z )   \le P_{Y}(y),
\end{align}
which we showed previously to imply the instrumental inequality.
\end{comment}

\begin{thebibliography}{10}

\bibitem{localorthog} Fritz, T., Sainz, A. B., Augusiak, R., Brask, J. B., Chaves, R., Leverrier, A., and Acin, A., Local orthogonality as a multipartite principle for quantum correlations. Nature communications {\bf 4},  2263 (2013).

\bibitem{GYNI} Acin, A., Almeida, M. L., Augusiak, R., and Brunner, N., Guess your neighbour�s input: no quantum advantage but an advantage for quantum theory. In Quantum Theory: Informational Foundations and Foils (pp. 465-496). Springer Netherlands (2-16). 

\end{thebibliography}

\end{document}  