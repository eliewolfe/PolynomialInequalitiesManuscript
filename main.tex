%\begin{filecontents}
%  @CONTROL{apsrev41Control,title="0"%,author="48",editor="1",pages="1",year="0"}
%\end{filecontents}
\RequirePackage[l2tabu, orthodox]{nag}
\RequirePackage{fixltx2e}
\RequirePackage{fix-cm}
\PassOptionsToPackage{pdftex,psdextra=true,
pdfversion=1.7,
pdfencoding=auto,
pdfnewwindow=true,
pdfusetitle=true,
psdextra=true,
%pdftoolbar=true,
%pdfmenubar=true,
bookmarks=true,
bookmarksnumbered=true,
bookmarksopen=true,
pdfpagemode=UseThumbs,
bookmarksopenlevel=1,
pdfpagelabels=false
}{hyperref}
\PassOptionsToPackage{usenames,dvipsnames}{xcolor}
\documentclass[aps,english,superscriptaddress,onecolumn,twoside,longbibliography,pra,floatfix,fleqn,nofootinbib]{revtex4-1}%


\usepackage[utf8]{inputenx}% for arXiv use encoding ansinew
\input{ix-utf8enc.dfu}
%\usepackage[utf8x]{inputenc}% for arXiv use encoding ansinew
%\usepackage{utf8mathlite}% custom style sheet for unicode-ish math
%\usepackage{newunicodechar}
%\newunicodechar{∫}{\int}
%\usepackage{unicode-math}
\usepackage[OT1]{fontenc}
\usepackage{ucs} %for unichar

\usepackage{amsfonts}
\usepackage{amssymb}
\usepackage{amsthm}
\usepackage[intlimits,fleqn]{amsmath}
%\usepackage{mathdots}
\usepackage{graphicx}%
\usepackage{placeins} %for FloatBarrier
\usepackage{afterpage} %for FloatBarrier in afterpage wrapper
%\usepackage{flushend}
%\usepackage{dblfloatfix}
\usepackage[normalem]{ulem} %for sout
\usepackage[raggedright,bf,nooneline]{subfigure}
\renewcommand{\thesubfigure}{\alph{subfigure}}
\usepackage{paralist}

%\usepackage{ellipsis}
\usepackage{float}% (not with floatrow)
\usepackage{wrapfig}
%\usepackage{floatrow}

\usepackage{marvosym} % for \Smiley and \Frowny
\usepackage{setspace}
\usepackage{array}
\usepackage{ragged2e}%for justifying text in tables
\usepackage{tabularx}
\def\tabularxcolumn#1{m{#1}}
\usepackage{booktabs}
%\usepackage{tabulary}
\newcolumntype{R}{>{\raggedleft\arraybackslash}X}
\newcolumntype{C}{>{\centering\arraybackslash}X}
\newcolumntype{L}{>{\raggedright\arraybackslash}X}
\newcolumntype{J}{>{\justifying\arraybackslash}X}
\usepackage{adjustbox}
\usepackage{multirow}
\newcolumntype{T}[2]{%
    >{\adjustbox{angle=#1,lap=\width-(#2)}\bgroup}%
    l%
    <{\egroup}%
}
\newcommand*\rot{\multicolumn{1}{T{90}{1em}}}% no optional argument here, please!

\setcounter{MaxMatrixCols}{30}
\providecommand{\U}[1]{\protect\rule{.1in}{.1in}}
%EndMSIPreambleData
\newtheorem{theorem}{Theorem}
\newtheorem{acknowledgement}[theorem]{Acknowledgement}
\newtheorem{algorithm}[theorem]{Algorithm}
\newtheorem{axiom}[theorem]{Axiom}
\newtheorem{claim}[theorem]{Claim}
\newtheorem{conclusion}[theorem]{Conclusion}
\newtheorem{condition}[theorem]{Condition}
\newtheorem{conjecture}[theorem]{Conjecture}
%\newtheorem{corollary}[theorem]{Corollary}
\newtheorem{corollary}{Corollary}[theorem]
\newtheorem{criterion}[theorem]{Criterion}
\newtheorem{definition}[theorem]{Definition}
%\newtheorem{example}[theorem]{Example}
\newtheorem{exercise}[theorem]{Exercise}
\newtheorem{lemma}[theorem]{Lemma}
\newtheorem{notation}[theorem]{Notation}
\newtheorem{problem}[theorem]{Problem}
\newtheorem{prop}{Proposition}
\newtheorem{taut}{Tautology}
\newtheorem{remark}[theorem]{Remark}
\newtheorem{solution}[theorem]{Solution}
\newtheorem{summary}[theorem]{Summary}
%\newenvironment{proof}[1][Proof]{\noindent\textbf{#1.} }{\ \rule{0.5em}{0.5em}}

% hyperlink stuff
\usepackage[usenames,dvipsnames]{xcolor}
\definecolor{ultramarine}{RGB}{63, 0, 255}
\definecolor{medblue}{RGB}{0, 0, 100}
\definecolor{panblue}{RGB}{0,24,150}
\definecolor{carmine}{RGB}{150, 0, 24}
\usepackage[breaklinks=true]{hyperref}
\hypersetup{colorlinks,
linkcolor=carmine,
citecolor=medblue,
urlcolor=panblue,
anchorcolor=OliveGreen}
%\usepackage{url}
\usepackage{pdfpages}

\definecolor{purple}{RGB}{128,0,128}
\definecolor{PURPLE}{RGB}{128,0,128}
\definecolor{BLACK}{RGB}{0,0,0}
\definecolor{ultramarine}{RGB}{63, 0, 255}
\definecolor{medblue}{RGB}{0, 0, 100}
\definecolor{panblue}{RGB}{0,24,150}
\definecolor{carmine}{RGB}{150, 0, 24}
\definecolor{gray}{RGB}{150, 150, 150}

\newcommand{\purp}[1]{{\color{purple}{#1}\color{black}}}
\newcommand*{\mred}[1]{{\color{RawSienna}{\mathbf{#1}}}}
\newcommand*{\mblue}[1]{{\color{MidnightBlue}{\ensuremath{#1}}}}
\newcommand*{\mpurp}[1]{{\color{Plum}{\mathbf{#1}}}}
\newcommand*{\mgreen}[1]{{\color{OliveGreen}{\mathbf{#1}}}}
\newcommand*{\tred}[1]{{\color{carmine}{\textbf{#1}}}}
\newcommand*{\tblue}[1]{{\color{MidnightBlue}{\textbf{#1}}}}
\newcommand*{\tpurp}[1]{{\color{Plum}{\textbf{#1}}}}
\newcommand*{\tgreen}[1]{{\color{Sepia}{\textbf{#1}}}}

\newcommand{\quoteby}{\raise.17ex\hbox{$\scriptstyle\sim$}}

\usepackage{verbatim} %for comment command
\usepackage{units}% for nicefrac
\newcommand{\half}[1]{\nicefrac{#1}{2}}

%\usepackage{braket} %provide \bra and \Bra and \set and \Set etc...
%\newcommand{\brackets}[1]{\lbrace{#1\rbrace}}
%\newcommand{\brackets}{\Set}



\usepackage{microtype}
%\usepackage{MnSymbol}
%\usepackage{mathabx}

\usepackage[capitalise]{cleveref}
\Crefname{eqs}{Eqs.}{Eqs.}

\creflabelformat{eqs}{(#2#1#3)}
\crefrangelabelformat{equation}{(#3#1#4-#5#2#6)}
%\crefmultiformat{equation}{eqs.~(#2#1#3)}{ and~(#2#1#3)}{, (#2#1#3)}{ and~(#2#1#3)}
\Crefmultiformat{equation}{Eqs.~(#2#1#3}{,#2#1#3)}{,#2#1#3}{,#2#1#3)}
\crefrangelabelformat{eqs}{(#3#1#4-#5#2#6)}
\Crefmultiformat{eqs}{Eqs.~(#2#1#3}{,#2#1#3)}{,#2#1#3}{,#2#1#3)}
\Crefname{prop}{\textbf{Prop}.}{\textbf{Props}.}
\Crefname{taut}{\textbf{Taut}.}{\textbf{Tauts}.}
\Crefname{section}{Sec.}{Secs.}

%\Crefname{ineq}{Ineq.}{Ineqs.}
%\creflabelformat{ineq}{(#2#1#3)}
%\crefrangelabelformat{ineq}{(#3#1#4-#5#2#6)}
%\Crefmultiformat{ineq}{Ineqs.~(#2#1#3}{,#2#1#3)}{,#2#1#3}{,#2#1#3)}

%\Crefname{ineqs}{Ineqs.}{Ineqs.}
%\creflabelformat{ineqs}{(#2#1#3)}
%\crefrangelabelformat{ineqs}{(#3#1#4-#5#2#6)}
%\Crefmultiformat{ineqs}{Ineqs.~(#2#1#3}{,#2#1#3)}{,#2#1#3}{,#2#1#3)}

\newenvironment{topic}[1][]{\par\medskip\noindent\textbf{\rmfamily#1}}{\par\medskip\par}

\newcounter{step}[section]
\newenvironment{step}[1][]{\refstepcounter{step}\par\medskip
   \noindent \textbf{Step~\thestep}\rmfamily#1}{\par\medskip\par}
%\newenvironment{step}[1][Step]{\noindent\textbf{#1.} }{\ \rule{0.5em}{0.5em}}
\Crefname{step}{Step}{Steps}
\creflabelformat{step}{#2#1#3}
\crefrangelabelformat{step}{#3#1#4-#5#2#6}
\Crefmultiformat{step}{Steps.~#2#1#3}{,#2#1#3}{,#2#1#3}{,#2#1#3}
\renewcommand{\thestep}{\arabic{step}}


\newcounter{example}[section]
\newenvironment{example}[1][]{\refstepcounter{example}\par\medskip
   \noindent \textbf{Example~\theexample}\rmfamily#1}{\par\medskip\par}
%\newenvironment{step}[1][Step]{\noindent\textbf{#1.} }{\ \rule{0.5em}{0.5em}}
\Crefname{example}{Exmpl.}{Exmpls.}
\creflabelformat{example}{#2#1#3}
\crefrangelabelformat{example}{#3#1#4-#5#2#6}
\Crefmultiformat{example}{Exmpls.~#2#1#3}{,#2#1#3}{,#2#1#3}{,#2#1#3}
\renewcommand{\theexample}{\arabic{example}}


\usepackage[intlimits,fleqn]{mathtools} %for mathclap and prescript and more. Learning to love this package. And DeclarePairDelimeter!
\DeclarePairedDelimiter{\ceil}{\lceil}{\rceil}
\DeclarePairedDelimiter{\floor}{\lfloor}{\rfloor}
\DeclarePairedDelimiter{\parens}{\lparen}{\rparen}
\DeclarePairedDelimiter{\parenths}{\lparen}{\rparen}
\DeclarePairedDelimiter{\abs}{\lvert}{\rvert}
\DeclarePairedDelimiter{\norm}{\lVert}{\rVert}
\DeclarePairedDelimiter{\braces}{\lbrace}{\rbrace}
\DeclarePairedDelimiter{\bracks}{\lbrack}{\rbrack}
\DeclarePairedDelimiter{\expec}{\langle}{\rangle}
\newcommand{\brackets}[1]{\braces*{#1}}

%\usepackage{nath} %automatically pair delimiters. Provides \inline and \displayed. Adjusts \frac and /

%\newcommand{\na}{\ensuremath{\mathring{a}}}
%\newcommand{\nb}{\ensuremath{\mathring{b}}}
%\newcommand{\nc}{\ensuremath{\mathring{c}}}
\newcommand{\na}{\ensuremath{\overline{a}}}
\newcommand{\nb}{\ensuremath{\overline{b}}}
\newcommand{\nc}{\ensuremath{\overline{c}}}

\newcommand{\naf}{\ensuremath{\lnot a}}
\newcommand{\nbf}{\ensuremath{\lnot b}}
\newcommand{\ncf}{\ensuremath{\lnot c}}

\newcommand{\n}[1]{\ensuremath{\overline{#1}}}
\newcommand{\ot}[1]{\ensuremath{\overline{#1}}}
\newcommand{\Nor}[1]{\operatorname{\mathsf{Nor}}\!\bracks*{#1}}

\newcommand{\larray}[1]{\ensuremath{\begin{array}{l}#1\end{array}}}
\newcommand{\lparens}[1]{\ensuremath{\parens*{\larray{#1}}}}
%\newcommand{\NamedFunction}[2]{\operatorname{\mathsf{#1}}\!\bracks*{#2}}
%\newcommand{\NamedFunction}[2]{\operatorname{\mathsf{#1}}\!\bracks*{\larray{#2}}}
\newcommand{\NamedFunction}[2]{\operatorname{\mathsf{#1}}\!\begin{bmatrix*}[l]#2\end{bmatrix*}}
%\newcommand{\SmallNamedFunction}[2]{\operatorname{\mathsf{#1}}\bracks{#2}}
\newcommand{\SmallNamedFunction}[3][]{{\operatorname{\mathsf{#2}}_{#1}}\bracks{#3}}
\newcommand{\nap}{\ensuremath{a'}}
\newcommand{\nbp}{\ensuremath{b'}}
\newcommand{\ncp}{\ensuremath{c'}}
\newcommand{\napp}{\ensuremath{a''}}
\newcommand{\nbpp}{\ensuremath{b''}}
\newcommand{\ncpp}{\ensuremath{c''}}

\newcommand{\p}[2][]{{P_{#1}}\parenths{#2}}
%\newcommand{\pdf}[1]{\operatorname{\mathsf{PDF}}\!\parenths{#1}}
\newcommand{\pdf}[2][]{{P_{#1}}\parenths{#2}}
\newcommand{\pfunc}[1]{P_{#1}}
\newcommand{\An}[2][]{{\mathsf{An}_{#1}}\parenths{#2}}
\newcommand{\Pa}[2][]{{\mathsf{Pa}_{#1}}\parenths{#2}}
\newcommand{\Ch}[2][]{{\mathsf{Ch}_{#1}}\parenths{#2}}
\newcommand{\subgraph}[2][]{{\operatorname{\mathsf{SubDAG}}_{#1}}\parenths[\big]{#2}}
\newcommand{\ansubgraph}[2][]{{\operatorname{\mathsf{AnSubDAG}}_{#1}}\parenths[\big]{#2}}
\newcommand{\pasubgraph}[2][]{{\operatorname{\mathsf{PaSubDAG}}_{#1}}\parenths[\big]{#2}}
\newcommand{\nodes}[1]{\SmallNamedFunction{Nodes}{#1}}
\newcommand{\edges}[1]{\SmallNamedFunction{Edges}{#1}}
\newcommand{\namedand}[1]{\SmallNamedFunction{And}{#1}}
\newcommand{\namedor}[1]{\SmallNamedFunction{Or}{#1}}
\newcommand{\aindep}{\ensuremath{\mathrel{\mathopen{\Lsh}{\scriptstyle\emptyset}\mathclose{\Rsh}}}}


%\newcommand{\subsim}[1]{\substack{\textstyle #1\\[-0.3ex]\sim}}
%\newcommand{\subsim}{\utilde}
%\def\subsim#1{\mathord{\vtop{\ialign{##\crcr
%$\hfil\displaystyle{#1}\hfil$\crcr\noalign{\kern1.5pt\nointerlineskip}
%$\hfil\tilde{}\hfil$\crcr\noalign{\kern1.5pt}}}}}
\newcommand{\subsim}[1]{\tilde{#1}}

\newcommand{\cramp}[1]{\ensuremath{\mathord{#1}}}
%\newcommand{\cramp}[1]{\ensuremath{\mathopen{}#1\mathclose{}}} oldway. New way is better.
\newcommand{\eql}{\cramp{=}}

\usepackage{bm}
\newcommand{\setlambda}{\bm{\lambda}}


%%%% Tobias: to mark my edits and stuff
\usepackage{showkeys}
\usepackage[draft]{fixme}
\newcommand{\btob}{\color{OliveGreen}}
\newcommand{\etob}{\color{black}}



\let\stdsection\section
%\renewcommand\section{\clearpage\stdsection}%every section new page


\begin{document}
%\preprint{ }
%\title{Transitivity of implication and causal structure}
\title{The Inflation DAG Technique for Causal Inference with Hidden Variables}
\author{Elie Wolfe}
\email{ewolfe@perimeterinstitute.ca}
\affiliation{Perimeter Institute for Theoretical Physics, Waterloo, Ontario, Canada, N2L 2Y5}
\author{Robert W. Spekkens}
\email{rspekkens@perimeterinstitute.ca}
\affiliation{Perimeter Institute for Theoretical Physics, Waterloo, Ontario, Canada, N2L 2Y5}
\author{Tobias Fritz}
\email{tobias.fritz@mis.mpg.de}
\affiliation{Perimeter Institute for Theoretical Physics, Waterloo, Ontario, Canada, N2L 2Y5}
\affiliation{Max Planck Institute for Mathematics in the Sciences, Leipzig, Germany}
\date{\today}


\begin{abstract}
\sout{The fundamental problem of causal inference is to infer from a given probability distribution over observed variables, what causal structures, possibly incorporating hidden variables, could have given rise to that distribution.}


The fundamental problem of causal inference is to infer if an observed probability distribution is compatible with some causal structure hypothesis, possibly incorporating hidden variables.
Given a particular causal structure, it is therefore valuable to derive incompatibility witnesses, i.e. criteria whose violation certifies the incompatibility of the violating distribution with the given causal structure.
The problem of causal inference via incompatibility witnesses comes up in many fields. Special incompatibility witnesses are Bell inequalities (which distinguish non-classical from classical distributions) and Tsirelson inequalities (which distinguish quantum from post-quantum distributions), and Pearl's instrumental inequality. All of these are limited to very specific causal structures. Analogues of such inequalities for more-general causal structures, i.e., necessary criteria for either classical or quantum distributions to be realizable from the structure, are highly sought after. 

We here introduce a technique for deriving such incompatibility witnesses, applicable to any causal structure. It consists of first \textit{inflating} the causal structure and then translating weak constraints on the inflated structure into stronger constraints on the original structure. Moreover, we show how our technique can be tuned to yield either classical witnesses (i.e., that may have quantum violations), or post-classical witnesses (i.e., that hold even in the context of general probability theories), depending on whether or not the inflation implicitly broadcasts the value of a hidden variable. Concretely, we derive polynomial inequalities for the so-called Triangle scenario, and we show how all Bell inequalities also follow from our method. %analyze Pearl's instrumental inequality from our perspective. 
Furthermore, given both a causal structure and a specific probability distribution, our technique can be used to efficiently witness their incompatibility, without requiring explicit inequalities. The inflation technique is therefore both relevant and practical for general causal inference tasks with hidden variables.


\end{abstract}
\maketitle
%In Ref.~\cite{WoodSpekkens}, the standard proof of Bell's theorem is presented in the language of causal inference.  In particular, the CHSH inequality emerges as a special case of what Pearl calls an ``instrumental inequality''.  Hardy's proof of Bell's theorem is quite different from the standard proof and the following question naturally arises: is there a generic tool for classical causal inference of which the Hardy argument can be considered a special case when applied to the M-shaped causal structure of the Bell experiment?

%To try and answer this question, we apply Hardy-type reasoning to the triangle causal structure, that is, the one with three observed variables, each pair of which have a common cause.  We show that this sort of reasoning does indeed facilitate causal inference in the case of the triangle causal structure, thereby lending some evidence to the notion that this style of argument has the potential to be generalized into a generic tool for classical causal inference.

\section{Introduction}

Given a probability distribution over some random variables, the problem of \tblue{causal inference} is to determine a plausible set of causal relations between these variables that could have generated the given distribution. This type of problem comes up in a wide variety of statistics applications, from sussing out biological pathways to enabling machine learning \cite{pearl2009causality,spirtes2011causation,studeny2005probabilistic,koller2009probabilistic}. A related problem is to start with a given set of causal relations, and then to determine the set of all probability distributions that these relations can generate on random variables that live on the nodes of the causal structure. Taking both of these starting points together, one can also start with \emph{both} a given distribution and a candidate causal structure and ask whether the two are compatible, in the sense that the causal structure could in principle have generated the given distribution.

%\tblue{infeasibility criteria}, i.e., observable constraints such that the their violation implies the invalidity of the hypothesis as an explanation for observational data. 

In the simplest setting, a causal structure hypothesis consists of a directed acyclic graph (DAG) which only contains the observed variables as nodes. In this case, obtaining a verdict on the compatibility of a given distribution with the causal structure is simple: the compatibility holds if and only if the distribution is Markov with respect to the DAG. The compatible causal structures can be determined algorithmically solely from the distribution, i.e.~without having an \emph{a priori} hypothesis~\cite{pearl2009causality}.

A significantly more difficult case is when one considers a causal structure hypothesis which consists of a DAG with \tblue{latent} nodes, so that the set of observable variables is only a subset of the nodes of the DAG. This case occurs e.g.~in situations where one needs to deal with the possible presence of unobserved confounders, as in experiment design. While the conditional independence relations implied by the causal structure on the observable nodes are still necessary conditions for compatibility of a distribution with the causal structure hypothesis, these equations are generally no longer sufficient. Finding necessary and sufficient conditions is a difficult open problem. Determining plausible candidate causal structures from the given distribution is even more difficult, but this can be reduced to the previous problem e.g.~by enumerating all DAGs (say with a fixed number of latent nodes) and checking one at a time for compatibility with the distribution. Due to the possibility of making this reduction, and because \sout{the}\purp{pairwise} compatibility determination is already difficult enough, we will focus on the case when a DAG (with latent nodes) is given as a description of the causal structure. We then consider the problem of deciding whether a given distribution can arise from this causal structure, as well as that of describing the entire set of these distributions for a fixed number of values per variable.

Historically, the insufficiency of the conditional independence relations for causal inference in the presence of latent variables was first noted by Bell in the context of the hidden variable problem of quantum physics~\cite{bell1964einstein}. Roughly speaking, Bell derived an inequality that any distribution compatible with the spacetime causal structure must satisfy, and found this inequality to be violated by distributions generated from suitably entangled quantum states\footnote{This incompatibility has subsequently become known as \emph{quantum nonlocality}~\cite{Brunner2013Bell}. Although the term suggests the existence of nonlocal interactions, in the sense that the actual causal structure may be different from the hypothesized one, this interpretation is at odds with the fact that no nonlocal interactions have been observed in nature, implying that their presence would require fine-tuning~\cite{WoodSpekkens}. A less problematic alternative conclusion from Bell's theorem is the impossibility to model quantum physics in terms of the usual notions of ``classical'' probability theory.}. Later on, Pearl derived another inequality, the \tblue{instrumental inequality}~\cite{pearl1995instrumental}, which witnesses the impossibility to model certain three-variable distribution by the \emph{instrumental scenario} causal structure hypothesis, with applications to experiment design. More recently, \citet{steudel2010ancestors} have derived inequalities which formalize an extended version of Reichenbach's principle by witnessing the impossibility to model a given distribution on $n$ variables in terms of a causal structure without a common (hidden) ancestor for any subset of at least $c$ nodes, for any $n,c\in\mathbb{N}$. Also, motivated by Bell's inequality, many other Bell-type inequalities have been used in quantum physics~\cite{Brunner2013Bell}, but the foundational role of causal structure has only recently been appreciated~\cite{WoodSpekkens,fritz2012bell,pusey2014gdag,BeyondBellII}.


%In contexts other than quantum theory, the latent nodes in causal structures are generally taken to represent hidden variables. This is not fully general, however, so we apply the retronym\footnote{Retronym (noun): a modification of an original term to distinguish it from a later development \cite{retronym}.} ``classical", as in classical causal structure and classical causal inference. The classical distributions of a given causal structure are defined as those which arise from it while restricting the latent nodes to be arbitrary (classical) random variables. Quantum distributions, by contrast, are those which are realizable if the latent nodes in the causal structure are allowed to be quantum systems. We hereafter take all causal structures and probability distributions to be classical, except where explicitly stated otherwise.

%From a physics perspective, therefore, tightly characterizing the set of observable probability distributions realizable from a causal structure is critical, in order to recognize and exploit the existence of distributions that can be realized quantumly but not classically. Few techniques are known for bounding this set of distributions which are simultaneously practical and applicable to general causal structures. Celebrated examples include the use of conditional independence relations (easy) \cite{pearl2009causality,spirtes2011causation,studeny2005probabilistic,koller2009probabilistic} and entropic inequalities (more advanced) \cite{fritz2013marginal,chaves2014novel,chaves2014informationinference}. In the presence of hidden variables, these criteria only rarely provide a tight characterization, and frequently fail to witness the non-classicality of quantum distributions.% Indeed, all the causal scenarios we shall consider here are instances where conventional causal compatibility criteria are found to be insufficient 

%Distinguishing quantum from classical correlations has historically been achieved through the use of Bell inequalities \cite{bell1966lhvm,GisinFramework2012,scarani2012device,Brunner2013Bell,BancalDIApproach}. Bell inequalities, however, are limited to very special causal scenarios involving \emph{only one} latent common cause variable, i.e.~Bell scenarios. A Bell scenario is also very special in that its realizable distributions admit characterization by a finite set of linear inequalities (after conditioning on the setting variables), i.e.~its realizable distributions comprise a convex polytope \cite{GisinFramework2012,FritzDuality}. %The nonconvexity of distributions realizable from a general structure is explicitly evidenced here. 
%Entirely new techniques, therefore, are required to derive quantum-sensitive incompatibility witnesses for more general causal scenarios \cite{fritz2012bell,pusey2014gdag,BeyondBellII}. 

We here introduce a new technique, applicable to any causal structure, for deriving incompatibility witnesses. This technique allows for, but is not limited to, the derivation of polynomial inequalities. As far as we know, our method is the first systematic tool for doing causal inference with latent variables that goes beyond observable conditional independence relations and does not assume any bounds on the number of values of each latent variable. While our method can be used to systematically generate necessary conditions for compatibility with a given causal structure hypothesis, we do not know whether these inequalities are also sufficient, and we currently have conflicting evidence on this question. On the one hand, our method rederives all Bell-type inequalities (\cref{sec:Bellscenarios}), but on the other hand we have not yet been able to obtain Pearl's instrumental inequality from our method.

% We show how the no-broadcasting theorem governing quantum theory \cite{NoCloningQuantum1996,NoCloningGeneral2006} can be exploited in order to derive specifically quantum-sensitive infeasibility criteria.
Our witnesses are generally based on the \emph{broadcasting} of the values of a hidden variable, i.e.~the assumption that its value can be copied and broadcast at will. The no-broadcasting theorem from quantum theory shows that this is not valid in the non-classical case, and from our perspective this is the reason for the existence of quantum violations of Bell inequalities. Moreover, our technique can also be applied in order to derive criteria that must be satisfied by all distributions that can be generated with latent nodes that are states in quantum theory or any other general probabilistic theory, simply by not assuming the possibility of broadcasting.
% like Tsirelson inequalities \cite{Tsirelson1980,Brunner2013Bell}, i.e. constraints which are satisfied by all distributions realizable from a given quantum causal structure. 

%\section{Notation}
\section{Causal models and causal inference}\label{sec:definitions}


%We need to define several notions.

%A causal inference problem is one whose input is some feature of observed statistical data---a sample of the probability distribution over the observed variables or some coarse-grained properties thereof, such as a list of conditional independence relations---and whose output is a verdict on (or probabilistic inference about) whether a given causal hypothesis can explain the observed data.   

A \tblue{causal model} consists of a pair of objects: a \tblue{causal structure} and a set of \tblue{causal parameters}.  We define each in turn.

The causal structure specifies a directed acyclic graph (DAG).  Recall that a DAG $G$ consists of a set of nodes and directed edges (i.e., ordered pairs of nodes), which we denote by $\nodes{G}$ and $\edges{G}$ respectively.  Each node corresponds to a random variable and a directed edge between two nodes corresponds to there being a direct causal influence from one variable to the other.   Our terminology for the causal relations between the nodes in a DAG is the standard one. The parents of a node $X$ in a given graph $G$ are defined as those nodes which have directed edges originating at them and terminating at $X$, i.e. $\Pa[G]{X} = \{\:Y\:|\:Y\to X\:\}$.  Similarly the children of a node $X$ in a given graph $G$ are defined as those nodes which have have directed edges originating at $X$ and terminating at them, i.e. $\Ch[G]{X} = \{\:Y\:|\: X\to Y\:\}$. If $\bm{U}$ is a set of nodes, then we put $\Pa[G]{\bm{U}} := \bigcup_{X\in\bm{U}} \Pa[G]{X}$ and $\Ch[G]{\bm{U}} := \bigcup_{X\in\bm{U}} \Ch[G]{X}$.  The \tblue{ancestors} of a set of nodes $\bm{U}$, denoted $\An[G]{\bm{U}}$, are defined as those nodes which have a directed \emph{path} to some node in $\bm{U}$, including the nodes in $\bm{U}$ themselves. Equivalently (dropping the $G$ subscript), $\An{\bm{U}} := \bigcup_{n\in\mathbb{N}} \mathsf{Pa}^n(\bm{U})$, where $\mathsf{Pa}^n(\bm{U})$ is inductively defined via $\mathsf{Pa}^0(\bm{U}) := \bm{U}$ and $\mathsf{Pa}^{n+1}(\bm{U}) := \mathsf{Pa}(\mathsf{Pa}^n(\bm{U}))$.  The causal structure also incorporates a distinction between the nodes of the DAG, namely, between those that are observed, denoted $\SmallNamedFunction{ObservedNodes}{G}$ and those that are latent, denoted $\SmallNamedFunction{LatentNodes}{G}$.

The set of causal parameters specify, for each node, the conditional probability distribution over the values of the random variable associated to that node, given the values of the variables associated to the node's parents.  (In the case of root nodes, the parents are the null set and the conditional probability distribution is simply a probability distribution.)
 %Specifically, for each root node, the model specifies a probability distribution over the values of the associated random variable, and for each non-root node, it specifies a conditional probability distribution over the values of the random variable associated to that node, given the values of the variables associated to its parents in the DAG.  
 We will denote a conditional probability distribution over a variable $Y$ given a variable $X$ by $\pfunc{Y|X}$, while the particular conditional probability of the variable $X$ taking the value $x$ given that the variable $Y$ takes the values $y$ is denoted $\pdf[Y|X]{y|x}$.  
% We will denote a conditional probability distribution over a variable $X$ given a set of variables $\bm{U}$ by $\pfunc{A|\bm{U}}$, while the particular value of this 
%Denoting the conditional probability distribution associated to a node $A$ by $\pdf{A|\Pa[G]{A}}$, we have
Therefore, a given set of causal parameters, denoted $F_G$, has the form
\begin{align}
F_G \equiv \{ \pfunc{A|\Pa[G]{A}} : A \in \SmallNamedFunction{Nodes}{G} \}.
\end{align}

%A causal model is defined as a pair of such objects: for a DAG $G$ and a set of causal parameters is $F_G$, the causal model is $M = (G,F_G)$. 
A DAG $G$ and a set of causal parameters $F_G$ defines a causal model, which we denote by $M = (G,F_G)$. In this case,
 $\SmallNamedFunction{DAG}{M}$ and $\SmallNamedFunction{Params}{M}$ denote $G$ and $F_G$ respectively.
 %the DAG and the set of parameters of $M$ respectively. 
%The functional parameters specify, for each root node, how the variable corresponding to that node is statistically distributed, and for each node that has parents in the DAG, the precise manner in which it causally depends on its parents.  If a DAG is denoted $G$ and the parameter values thereon are denoted $\mathcal{P}(G)$, then the corresponding causal model is the pair $C = (G,\mathcal{P}(G))$.  
%We shall also make use of the notation $\SmallNamedFunction{DAG}{C}$ for $G$ and $\SmallNamedFunction{ParamVals}{C}$ for $\mathcal{P}(G)$. \purp{I want to switch notation: A model should be indicated by $\mathcal{M}$, a DAG by $\mathcal{G}$, and a parameters as $\mathcal{F}$. I then want to talk about sets of parameters as  $\bm{\mathcal{F}}$ and sets of models as $\bm{\mathcal{M}}$. Thus a model is a pair $(\mathcal{G},\bm{\mathcal{F}})$, and a hypothesis $\mathcal{H}$ is a pair $(\mathcal{G},\bm{\mathcal{M}})$.}


%In a causal model, the localized systems are random variables, and the parameters are conditional probabilities
%\footnote{This is the definition of a {\em classical}  causal model, which will be the primary focus of this article. Nonetheless, there is a quantum generalization of the notion of a causal model that will feature in our discussions at the end of this article.  In a {\em quantum} causal model\cite{leifer2013conditionalstates}, the localized systems are types of quantum systems (specified by Hilbert space dimensionality), and the parameters are quantum operations.  Specifically, for each root node, the model specifies a unit-trace positive operator on the Hilbert space for that system and for each non-root node, the model specifies a trace-preserving completely-positive linear map from the operators on the Hilbert space of the parents of the system to the operators on the Hilbert space of the system.  Certain nodes in a quantum causal model may be considered classical, in which case all states and operations must be diagonal in some fixed basis on that system.  One can also generalize the notion of a causal model further still to the case of generalized probabilistic theories, as was done in Ref.~\cite{pusey2014gdag}.}.

%Specifically, for each node, the model specifies a conditional probability distribution over the values of the random variable associated to that node, given the values of the variables associated to its parents in the DAG.  (In the case of root nodes, the parents are the null set and the conditional probability distribution is simply a probability distribution.)
%Denoting the conditional probability distribution associated to a node $A$ in the causal model $C$ by $\pdf[C]{A|\Pa[G]{A}}$, we have
%\begin{align}
%\mathcal{P}(G) \equiv \{\pdf[C]{A|\Pa[G]{A}}: A \in \SmallNamedFunction{Nodes}{G}\}.
%\end{align}
As is standard, a causal model $M$ specifies a joint distribution over all variables in the DAG, namely,
\begin{align}
\pdf{\SmallNamedFunction{Nodes}{G}} = \Pi_{A\in \SmallNamedFunction{Nodes}{G}} \pfunc{A|\Pa[G]{A}},
\end{align}
(typically called the Markov condition), 
%The subset of nodes of $G$ that are observed is denoted $\SmallNamedFunction{ObservedNodes}{G}$ and the rest are denoted $\SmallNamedFunction{LatentNodes}{G}$. 
and the joint distribution over the observed variables is obtained from the joint distribution over all variables by marginalization over the latent variables 
\begin{align}
\pdf{\SmallNamedFunction{ObservedNodes}{G}} =  \sum_{\{X :X \in\SmallNamedFunction{LatentNodes}{G}\}} \pdf{\SmallNamedFunction{Nodes}{G}}.
\end{align}

In addition to the notion of a causal model, it is useful to have a notion of a \tblue{causal hypothesis}, which we define to be a set of causal models.  We shall consider only causal hypotheses for which the models in the set share the same DAG, and we denote such a causal hypothesis by $H_G$ where $G$ is the associated DAG.
% consistent with a single DAG. Specifically, if $G$ is a DAG and $S$ is a set of parameter values consistent with $G$ (i.e. a set of possibilities for the conditional probability distributions at each node of $G$),\purp{Wait, are the defining a hypothesis positively or negatively? This seems mixed up to me.} then $G$ and $S$ define a causal hypothesis, which we denote by $H_{G,S}$, and which we define formally as the following set of causal models:
%\begin{align}
%H_{G,S} \equiv \{ C \mid \SmallNamedFunction{DAG}{C}=G, \SmallNamedFunction{ParamVals}{C} \in S\}.
%\end{align}
% (is this standard usage of the term "model"?  In physics, a model typically does not include a specification of the boundary conditions, or does it?)
The most simple sort of causal hypothesis is one that includes the full set of causal models associated to a DAG $G$, which we denote by $H_G^{\textrm{full}}$.  This is simply the hypothesis of a particular DAG, with no restriction on the set of parameters that supplement it.   Examples of more refined hypotheses are provided below.

%We denote  the set of {\em all} parameter values consistent with $G$ by $S_{\textsf{full}}$, so that $H_{G,{S_{\textsf{full}}}}$ is simply the causal hypothesis that the DAG is $G$, that is, This is the hypothesis that there exists {\em some} set of parameter values supplementing $G$ such that the observed data can be explained by the resulting causal model. 


With these notions in hand, we can finally specify the sort of causal inference problem for which one can usefully apply our technique, namely, those that come in the form of a decision problem, taking as input (i) a specification of observational data and (ii) a specification of a causal hypothesis, and outputting whether or not they are compatible.  In broad strokes, the inflation DAG technique is a way of mapping such a causal inference problem onto a new one where the observational and causal inputs of the new problem are determined by the observational and causal inputs of the original problem in such that if there is compatibility between the observational and causal inputs of the original problem, then there is compatibility between the observational and causal inputs of the new problem.  The technique is useful because, as we show, simple witnesses of inconsistency in the new problem yield nontrivial witnesses of inconsistency in the original problem. 

%\begin{itemize}
%\item a specification of {\em observational data}
%\item a specification of a {\em causal hypothesis}
%\end{itemize}
%and that outputs a verdict about their compatibility.

The inflation DAG technique can accommodate many different forms for the observational data and the causal hypothesis appearing in the original problem.  It also imposes a restriction on the forms for the observational data and the causal hypothesis appearing in the new problem.  It is therefore useful to pause and consider the range of possibilities for the two inputs of a causal decision problem. 

The possibilities for the form of the observational data includes:
\begin{enumerate}
\item a specification of the joint distributions over all of the observed variables.
\item a specification of a region in the space of possible joint distributions over all of the observed variables
\item a specification of the conditional independence relations that hold among the observed variables
\item a specification of the marginals of the joint distribution for certain subsets of the observed variables
\end{enumerate}
In addition to this sort of variety, one can imagine that the statistical dependences among a set of variables may be specified not by a joint distribution but by covariance matrices or the values of entropic quantities such as mutual information.  Combinations of these possibilities are also possible.

The specification of the joint distribution, (1), is the most informative form the observational data can take.  Specifying a region, (2), provides a means of expressing uncertainty about the joint distribution.  Specifying conditional independences, (3), is the form of observational data that has been most thoroughly exploited in the development of tools for causal inference.   If one specifies marginals, as in (4), then the causal inference problem becomes a version of the marginals problem (described in the introduction), but where the space of joint distributions from which the marginals may arise is constrained by the causal hypothesis.  
%We refer to this as a causally-constrained marginals problem. 

The inflation DAG technique can be applied for {\em any} of these types of observational data.   Nonetheless. the particular concrete applications of the inflation DAG technique that we will describe in detail in this article will consider causal inference problems wherein the joint distribution is specified, i.e., data of type (1).  The new causal inference problem to which this original problem is mapped by inflation, however, is one where the data concerns marginals, i.e., data of type (4).

%The most informative sort of observational data is a specification of the joint distributions over all of the observed variables.  There are many possibilities for the data being less informative.  It may merely specify an interval in the space of possible joint distributions, reflecting uncertainty about the joint distribution for instance.   Alternatively, one might merely have a description of the conditional independence relations that hold among the observed variables.  Or one might merely have a specification of the marginals of the joint distribution for certain subsets of the observed variables.  In this last case, the causal inference problem corresponds to a version of the marginals problem, but where the joint distribution is constrained by the causal hypothesis.  Such a causally-constrained version of the marginals problem will be central to our inflation DAG technique. 

It is also useful to list a variety of possibilities for the form that the causal hypothesis may take:
\begin{enumerate}
\item the full set of causal models for a particular DAG $G$
\item the full set of causal models for a particular DAG $G$ excluding those that yield conditional independence relations beyond those implies by d-separation in $G$
\item the set of causal models for a particular DAG $G$ wherein there are constraints on the manner in which particular nodes causally depend on their parents 
\item the set of causal models for a particular DAG $G$ wherein the manner in which a particular node causally depends on its parents is constrained to be equivalent, under some mapping between nodes, to the manner in which another node causally depends on its parents. 
\end{enumerate}
In all cases, we have left implicit the trivial constraint that the causal hypothesis must include only those DAGs for which the set of observed nodes coincides with the set of variables described in the observational data. 

The motivation for considering hypotheses of the form of (2) rather than (1) is the principle that a causal explanation should not be fine-tuned [reference].  If, for some observational data, a causal hypothesis of type (1) is consistent but one of type (2) is not, then it means that the conditional independences in the observational data are not a consequence of the causal structure, but rather are a consequence of the choice of parameters.  Such an explanation can be criticized on the grounds that it is fine-tuned.  Finding a causal hypothesis of type (2) that is consistent with the observational data implies that one has a non-fine-tuned explanation of that data.  The fine-tuning issue will come up in Sec. ? where we discuss quantum causal models. 


%On the side of the causal hypothesis, it is most common to consider the hypothesis of a particular DAG, with no restriction on the set of parameters that supplement it.  This is just the full set of causal models associated to a DAG $G$, which we denote by $H_G^{\rm full}$.  Our results will primarily concern this sort of hypothesis.    
%Nonetheless, our technique is also applicable to  causal hypotheses that are more fine-grained.One example is to exclude from the causal models consistent with a DAG all of those which generate conditional independence relations that do not follow from d-separation criteria applied to the DAG.  If such a causal hypothesis  is found to be consistent with some observational data, then this consistency is not a result of fine-tuning. 

%Another example of a fine-graining relative to the causal hypothesis of a DAG is to incorporate constraints on how certain nodes in the DAG functionally depend on their parents.  
An example of a causal hypothesis of type (3) is the hypothesis of an additive noise model: if an observed variable $Y$ has an observed variable $X$ and a latent variable $U$ as parents, then the noise is deemed additive if $Y=\alpha X + \beta U$ for some scalars $\alpha$ and $\beta$ [provide references].  Clearly, the conditional probability distributions $\pfunc{Y|XU}$ that can be achieved in such an additive noise model are a subset of the valid conditional distributions.  More general constraints on the conditional distributions have also been explored [provide references].  
%(Note that constraints on the probability distributions over the latent variables can be understood as a kind of constraint on the causal dependences.)
%Another conceivable constraint on the parameters is a constraint on the form of the probabilities distributions over the latent variables.  

A causal hypothesis of type (4) involves a novel and unusual sort of constraint, which has not, to our knowledge been studied previously.  We include it on our list because it is the sort of causal hypothesis that appears in the new causal problem that is defined by our inflation DAG technique.  Insofar as it is usually assumed that the manner in which one node in a DAG causally depends on its parents should be completely independent of the manner in which another depens on its parents---sometimes described as the assumption of {\em autonomy} of different causal mechanisms [Pearl]---it is quite conceivable that hypotheses of type (4) have {\em no} significance besides the role that they play in the inflation DAG technique.\footnote{\color{red} We could say something here about how this suggests that one does better to think of the inflation DAG in terms of counterfactuals and twin diagrams. \color{black}.}

%Our inflation DAG technique will require consideration of a novel and unusual sort of causal hypothesis, wherein the conditional probability distributions associated to distinct nodes are sometimes required to be equivalent. 
%constraint on the sets of parameters of the model. \purp{What is difference between causal dependencies and parameters?}

%\sout{Finally, the verdict about the compatibility of the observational input and the causal hypothesis could come in different forms.  The simplest possibility is a specification of whether the observational input is consistent with the causal hypothesis.  However, the notion of compatibility at play might be more refined than this, for instance, the algorithm might specify a confidence level with which one can reject the given causal hypothesis as an explanation of the observational input.}\purp{Replace with ``we consider causal inference to be about deciding consistency only..."}

%The most informative sort of observational input is a specification of the joint distributions over all of the observed variables.  Nonetheless, causal inference problems often proceed with a less informative sort of observational input.  For instance, one might merely have a description of the conditional independence relations that hold among the obeerved variables.  Another example of how the observational input might be less informative is if it merely specifies the marginals of the joint distribution for certain subsets of the observed variables.  (Indeed, the latter sort of observational input will be relevant for our inflation DAG technique.)   The observational input might also be less informative in another sense, namely, that it is merely a {\em finite sample} of some joint distribution.

%In broad strokes, the inflation DAG technique is a way of mapping a given causal inference problem onto a new causal inference problem where the observational and causal inputs of the latter problem are determined by the observational and causal inputs of the original problem.   
%In particular, under the inflation map, the observational data becomes less informative \purp{say more precisely?} while the causal hypothesis becomes more fine-grained, in a manner that we shall now make explicit.

\section{Inflation: a tool for causal inference}

We now introduce the notion of \tblue{an inflation of a causal model}.  %Let $C_G$ denote a causal model associated to a DAG $G$.  
If the original causal model is associated to a DAG $G$, then a nontrivial inflation of this model is associated to a different DAG, $G'$.  
%Below, we define a particular map from a DAG $G$ to a DAG $G'$ 
%An inflation of this model is another causal model, denoted $C_{G'}$ and associated to a different DAG, $G'$.  
We refer to $G'$ as an inflation of $G$.  There are many possible choices of $G'$ for a given $G$ (specified below), hence many possible inflations of a given DAG.  We denote the set of these by $\SmallNamedFunction{Inflations}{G}$.   The choice of an element $G'\in \SmallNamedFunction{Inflations}{G}$ is the only freedom in the inflation of a causal model.  Once a choice is made, the set of parameters of the inflation model is fixed uniquely by the the set of parameters of the original model.  We denote the function on models associated with such an inflation by $\SmallNamedFunction{Inflation}{G\to G'}{M}$ (specified below).

%Once an element $G'\in \SmallNamedFunction{Inflations}{G}$ is specified, however, the parameters of the causal model $C_{G'}$ are fixed by the parameters of the causal model $C_{G}$.  We denote this function by $C_(G') = \SmallNamedFunction{Inflation}{C_(G)}$.

% $\mathcal{P'}(G')$ are fixed uniquely by $\mathcal{P}(G)$.  We call the parameters $\mathcal{P'}(G')$ the inflationary image of the parameters $\mathcal{P}(G)$, and write $\mathcal{P'}(G') = \SmallNamedFunction{Inflation}{\mathcal{P}(G)}$.  Let $C=(G,\mathcal{P}(G))$ denote the original causal model and $C'=(G',\mathcal{P'}(G'))$ an inflation thereof.   We say that the DAG $G'$ is an inflation of the DAG $G$.  There are many possible choices of $G'$ for a given $G$, hence many possible inflations of a given causal model.  We denote this set by $\SmallNamedFunction{Inflations}{G}$.   Once an element $G'\in \SmallNamedFunction{Inflations}{G}$ is specified, however, the parameters $\mathcal{P'}(G')$ are fixed uniquely by $\mathcal{P}(G)$.  We call the parameters $\mathcal{P'}(G')$ the inflationary image of the parameters $\mathcal{P}(G)$, and write $\mathcal{P'}(G') = \SmallNamedFunction{Inflation}{\mathcal{P}(G)}$.
 %We now describe concretely what an inflation is for the DAG and for the parameters in turn.  

We begin by defining the condition under which a DAG $G'$  is an inflation of a DAG $G$.  This requires some preliminary definitions. 

The \tblue{subgraph} of $G$ induced by restricting attention to the set of nodes $\bm{V}\subseteq \SmallNamedFunction{Nodes}{G}$ will be denoted $\subgraph[G]{\bm{V}}$.
It consists of the nodes $\bm{V}$ and the edges between pairs of nodes in $\bm{V}$ per the original graph. Of special importance to us is the 
\tblue{ancestral subgraph} of $\bm{V}$, denoted $\ansubgraph[G]{\bm{V}}$, which is the minimal subgraph containing the full ancestry of $\bm{V}$, $\ansubgraph[G]{\bm{V}}\coloneqq\subgraph[G]{\An[G]{\bm{V}}}$. 

Inflation involves a sort of copying operation on nodes of the DAG.  Specifically, every node of $G'$ can be understood as a copy of a node of $G$.    If $A$ denotes a node in $G$ that has copies in $G'$, then we denote these copies by $A_1,\ldots, A_k$, and the variable that indexes the copies is termed the \tblue{copy-index}.  When two objects (e.g.~nodes, sets of nodes, DAGs, etc\ldots) are the same up to copy-indices, then we use $\sim$ to indicate this.  For instance, we have $A_i\sim A_j\sim A$.   This copying operation must also preserve the causal structure of the DAG, in a manner that is formalized by the following definition. 
%, and a set of variables in $G'$ that differ only by copy-index are called a {\em copy set}.  
\begin{definition}
%The necessary condition on $G'$ for being an inflation of $G$ is that
The DAG $G'$ is said to be an inflation of the DAG $G$ if and only if  for every node $A_i$ in $G'$, the ancestral subgraph of $A_i$ in $G'$ is equivalent, under removal of the copy index, to the ancestral subgraph of $A$ in $G$,
\begin{align}\label{eq:definflationDAG}
G' \in\SmallNamedFunction{Inflations}{G} \quad\text{ iff }\quad \forall A_i\in \SmallNamedFunction{Nodes}{G'}:\; \ansubgraph[G']{A_i}\sim\ansubgraph[G]{A}.
\end{align}
\end{definition}
%Given this notational convention, we can formalize the condition for $G'$ to be an inflation of $G$ as follows:

To illustrate the notion of the inflation of a DAG, we consider the DAG of \cref{fig:TriMainDAG}, which is called the {\em Triangle scenario} (for obvious reasons) and which has been studied by many authors [\citealp{pusey2014gdag}~(Fig.~E\#8), \citealp{WoodSpekkens}~(Fig.~18b), \citealp{fritz2012bell}~(Fig.~3), \citealp{chaves2014novel}~(Fig.~6a), \citealp{Chaves2015infoquantum}~(Fig.~1a), \citealp{BilocalCorrelations}~(Fig.~8), \citealp{steudel2010ancestors}~(Fig.~1b), \citealp{chaves2014informationinference}~(Fig.~4b)]
%As an example, consider the Triangle scenario [\citealp{pusey2014gdag}~(Fig.~E\#8), \citealp{WoodSpekkens}~(Fig.~18b), \citealp{fritz2012bell}~(Fig.~3), \citealp{chaves2014novel}~(Fig.~6a), \citealp{Chaves2015infoquantum}~(Fig.~1a), \citealp{BilocalCorrelations}~(Fig.~8), \citealp{steudel2010ancestors}~(Fig.~1b), \citealp{chaves2014informationinference}~(Fig.~4b)]. The associated DAG, the shape of which explains the name, is depicted here in \cref{fig:TriMainDAG}. 
%The Triangle scenario is a correlation scenario in the sense of \citet{fritz2012bell}; see especially Sec. 2.3 there. 
Different inflations of the Triangle scenario are depicted in \cref{fig:TriFullDouble,fig:Tri222,fig:simpleinflation,fig:simplestinflation,fig:TriDagSubA2B1C1}.

\begin{figure}[h]
\centering
\begin{minipage}[b]{0.23\linewidth}
\centering
\includegraphics[scale=1]{TriDagRaw.pdf}
\caption{The causal structure of the Triangle scenario.}\label{fig:TriMainDAG}
\end{minipage}
\hfill
\begin{minipage}[t]{0.38\linewidth}
\centering
\includegraphics[scale=1]{TriDagFull222.pdf}
\caption{An inflation DAG of the Triangle scenario where each latent node has been duplicated, resulting in four copies of each observable node.}\label{fig:TriFullDouble}
\end{minipage}
\hfill
\begin{minipage}[b]{0.35\linewidth}
\centering
\includegraphics[scale=1]{TriDagSub222.pdf}
\caption{Another inflation of the Triangle scenario consisting, also notably $\ansubgraph[(\textrm{\cref{fig:TriFullDouble}})]{A_1 A_2 B_1 B_2 C_1 C_2}$.}\label{fig:Tri222}
\end{minipage}
\end{figure}

\begin{figure}[hb]
\centering
\begin{minipage}[t]{0.3\linewidth}
\centering
\includegraphics[scale=1]{broadcastingexamplenohighlight.pdf}
\caption{A simple inflation of the Triangle scenario, also notably $\ansubgraph[(\cref{fig:Tri222})]{A_1 A_2 B_1 C_1}$.}\label{fig:simpleinflation}
\end{minipage}\hfill
\begin{minipage}[t]{0.275\linewidth}
\centering
\includegraphics[scale=1]{nobroadcastingexamplenohighlight.pdf}
\caption{An even simpler inflation of the Triangle scenario, also notably $\ansubgraph[(\cref{fig:simpleinflation})]{A_2 B_1 C_1}$. }\label{fig:simplestinflation}
\end{minipage}
\hfill
\begin{minipage}[t]{0.325\linewidth}
\centering
\includegraphics[scale=1]{TriDagSubA2B1C1.pdf}
\caption{Another representation of \cref{fig:simplestinflation}. Despite not containing the original scenario, this is a valid inflation per \cref{eq:definflationDAG}.}\label{fig:TriDagSubA2B1C1}
\end{minipage}
\end{figure}

We now turn to specifying the function $\SmallNamedFunction{Inflation}\;_{G\to G'}$, that is, to specifying how the set of parameters of a causal model transform under inflation. 
\purp{One we've said how DAG inflated, then we need TWO things: How models (and hypothesis) inflate, and also how observational data inflates. Best to word intro to subject in a way that suggests we'll be getting around to both topics sequentially?}
%Next, we turn to specifying how the parameters in the inflation of a causal model are determined from those in the original model.
%condition under which a set of parameters $\mathcal{P'}(G')$ are the inflationary image of the set of parameters $\mathcal{P}(G)$. 
\purp{PHYSICS needed here. Why mathematical abstraction? Explain that due to physical construct-ability, therefore we expect that models on the original DAG can be used to build a model on the inflation DAG...}
\begin{definition}
Consider causal models $M$ and $M'$ where $\SmallNamedFunction{DAG}{M}=G$ and $\SmallNamedFunction{DAG}{M'}=G'$ and such that $G'$ is an inflation of $G$.   The causal model $M'$ is the $G\to G'$ inflation of $M$, that is, $M' = \SmallNamedFunction{Inflation}{M}$,  if and only if for every node $A_i$ in $G'$, the manner in which $A_i$ depends causally on its parents within $G'$ must be the same as the manner in which $A$ depends causally on its parents within $G$.  Noting that $A_i \sim A$ and that $\Pa[G']{A_i} \sim \Pa[G]{A}$ (given Eq.~\eqref{eq:definflationDAG}), one can formalize this condition as:
\begin{align}\label{eq:funcdependences}
%M' = \SmallNamedFunction{Inflation}{M}_{G\to G'}  \quad \text{iff}\quad 
 \forall A_i \in \SmallNamedFunction{Nodes}{G'}:\; \pfunc{A_i| \Pa[G']{A_i}}=\pfunc{A|\Pa[G]{A}},
\end{align}
%where Eq.~\eqref{eq:definflationDAG} guarantees that $A_i \sim A$ and $\Pa[G']{A_i} \sim \Pa[G]{A}$.  
\end{definition}

%A causal model specifies the causal structure and the autonomous causal mechanisms.  
To sum up then, inflation is a mapping to a new causal model wherein each given variable in the original DAG may have counterparts in the inflation DAG and where the manner in which variables depend causally on their parents in the inflation DAG is given by the manner in which their counterparts depend causally on their parents in the original DAG.   Note that the operation of modifying a DAG and equipping the modified version with conditional probability distributions that mirror those of the original also appears in the \emph{do calculus} of~\citet{pearl2009causality}. \fxnote{adhesivity and non-Shannon-type ineqs}
%The copying of variables under inflation implies that causal mechanisms in the inflation DAG are not necessarily autonomous, but rather often coincide with one another.


%defines a novel sort of constraint on the parameters of a causal model.  We pause to describe this constraint in general terms, before introducing the notion of inflation.  The sorts of causal structures that can arise in our inflation scheme have the property that necessarily there will be certain pairs of variables which have the same ancestral subgraph.  The constraint on the parameters is that the two conditional probability distributions describing how each variable of the pair depends on its parents are quantitatively equal to one another.  (To our knowledge, this sort of constraint has not been considered before in causal inference.)

%The inflation DAG technique that we introduce allows questions about the viability of one causal hypothesis for certain observed data to be turned into questions about the viability of a different sort of causal hypothesis for an inflation of the observed data.

%Because we have defined inflation as a map on causal models, 
Because causal hypotheses are sets of causal models, and the inflation map acts on causal models, it has an induced action on causal hypotheses.   Let $H_G^{\textsf{full}}$ denote the causal hypothesis of a DAG $G$, that is, the set of all models consistent with that DAG.  The induced action of the inflation map has the notable feature that $\SmallNamedFunction{Inflation}{H_G^{\textsf{full}}} \ne H_{G'}^{\textsf{full}}$ because even if there is no restriction on the set of parameters supplementing $G$, inflation imposes a restriction on the set of parameter supplementing $G'$.  Specifically,   if nodes $A_i$ and $A_j$ on $G'$ are copies of a single node $A$ on $G$,then the set of parameters on $G'$ are constrained to satisfy $\pfunc{A_i| \Pa[G']{A_i}}=\pfunc{A_j|\Pa[G']{A_j}}$.  This is the sense in which the inflation map takes a causal hypothesis of type (1) and maps it to a causal hypothesis of type (4).

%Note that even if $S=S_{\textsf{full}}$, so that the original causal hypothesis puts no constraints on the parameter values, one generally has $S'\ne S'_{\textsf{full}}$, that is, the inflationary image of the full set of parameter values on $G$ is not the full set of parameter values on $G'$. Rather, $S'$ incorporates a nontrivial restriction on the parameters consistent with $G'$, namely, that if nodes $A_i$ and $A_j$ on $G'$ are copies of a single node $A$ on $G$,then the parameter values on $G'$ are constrained to satisfy $\pdf{A_i| \Pa[G']{A_i}}=\pdf{A_j|\Pa[G']{A_j}}$.

%Finally, we can begin to spell out how the inflation technique is useful for causal inference.  
%First, we emphasize the broad notion of causal inference that we have in mind here. 



%[Note that the distinction between observed nodes and latent nodes in the DAG $G$ is inherited by the DAG $G'$.]


To see how inflation is relevant for causal inference, we must explain how the distributions that can be achieved in the inflation model are constrained by those in the original model.  In what follows, we assume that $G' \in \SmallNamedFunction{Inflations}{G}$ and that $C' = \SmallNamedFunction{Inflation}{C}$.

Note, first of all, that for any sets of nodes $\bm{U}\in \SmallNamedFunction{Nodes}{G'}$ and   $\subsim{\bm{U}}\in \SmallNamedFunction{Nodes}{G}$,
\begin{align}\label{eq:coincidingdistrodef}
\text{if }\quad \ansubgraph[G']{\bm{U}}\sim\ansubgraph[G]{\subsim{\bm{U}}}\quad\text{then}\quad \pdf[C']{\bm{U}}=\pdf[C]{\subsim{\bm{U}}}.
\end{align}
This follows from the fact that the probability distributions over $\bm{U}$ and $\subsim{\bm{U}}$ depend only on their ancestral subgraphs and the parameters defined thereon, which by the definition of inflation are the same for $\bm{U}$ and for $\subsim{\bm{U}}$.

It is useful to have a name for a set of nodes in the inflation DAG, $\bm{U} \subseteq \SmallNamedFunction{Nodes}{G'}$, such that one can find a corresponding set in the original DAG, $\subsim{\bm{U}} \subseteq \SmallNamedFunction{Nodes}{G}$, which has an equivalent ancestral subgraph.   We say that such subsets of the nodes of $G'$ are injectable into $G$ and we call them the \tblue{injectable sets},
% into $\SmallNamedFunction{Nodes}{G}$,
\begin{align}\label{eq:definjectable}
\bm{U}\in\SmallNamedFunction{InjectableSets}{G'} \quad\text{ iff }\quad \exists \subsim{\bm{U}}\subseteq \SmallNamedFunction{Nodes}{G}: \ansubgraph[G']{\bm{U}}\sim\ansubgraph[G]{\subsim{\bm{U}}}.
\end{align}
In \cref{fig:Tri222}, for example, the set $\brackets{A_1 B_1 C_1}$ is injectable because its ancestral subgraph is equivalent up to copy-indices to the ancestral subgraph of $\brackets{A B C}$ in the original DAG (which is just the full DAG), and the set $\brackets{A_2 C_1}$ is injectable because its ancestral subgraph is equivalent to that of $\brackets{ A C}$ in the original DAG. 

Note that it is clear that a set of nodes in the inflation DAG can only be injectable if it contains at most one copy of any node from the original DAG.  Similarly, it can only be injectable if its ancestral subgraph also contains at most one copy of any node from the original DAG.  
Thus, in \cref{fig:Tri222}, $\brackets{A_1 A_2 C_1}$ is not injectable because it contains two copies of $A$, and $\brackets{A_2 B_1 C_1}$ is not injectable because its ancestral subgraph contains two copies of $Y$. 

%The relation between nodes of the inflation DAG $G'$ to nodes on the DAG $G$ that will be of primary interest to us is not injectability  actually slightly weaker than the notion of injectability.  
The sets of nodes of $G'$ that will be of primary interest to us are not the injectable sets per se, but sets satisfying a slightly weaker constraint.
%We call this notion {\em pre-injectability}.  
To define the latter sorts of sets, which we term {\em pre-injectable}, we must first introduce some additional terminology. 

We refer to a pair of nodes which do not share any common ancestor as being \tblue{ancestrally independent}, for which we invent the notation $X\aindep Y$. Generalizing to sets, $\bm{U}\aindep \bm{V}$ indicates that no node in $\bm{U}$ shares a common ancestor with any node in $\bm{V}$, 
\begin{align}
\bm{U}\aindep \bm{V} \quad \text{iff} \quad \An{\bm{U}}\cap\An{\bm{V}}=\emptyset.
\end{align}
%i.e.~$\An{\bm{U}}\cap\An{\bm{V}}=\emptyset$. 
%It is possible for more than two sets to be ancestrally independent: the notation ${\bm{U}\aindep \bm{V}\aindep \bm{W}}$ should be understood as indicating that the ancestors of $\bm{U}$,$\bm{V}$, and $\bm{W}$ comprise three distinct non-overlapping sets, i.e.~$\bm{U}\aindep \bm{V}$ and $\bm{V}\aindep \bm{W}$ and $\bm{U}\aindep \bm{W}$.
Ancestral independence is equivalent to $d$-separation by the empty set~\cite{pearl2009causality,spirtes2011causation,studeny2005probabilistic,koller2009probabilistic}. 


A set of nodes $\bm{U}$ in the inflation DAG $G'$ will be called \tblue{pre-injectable} whenever it is a union of injectable sets with disjoint ancestries. 
\begin{align}\label{eq:defpreinj}
\bm{U}\in\SmallNamedFunction{PreInjectableSets}{G'} \quad\text{ iff }\quad  \exists \{ \bm{U}_i \in \SmallNamedFunction{InjectableSets}{G'} \} \quad \text{s.t.}\quad \bm{U}=\bigcup_i \bm{U}_i  \quad\text{and} \quad  \forall i\ne j: \bm{U}_i \aindep \bm{U}_j.
%\bm{U}\in\SmallNamedFunction{PreInjectableSets}{G'} \quad\text{ iff }\quad  \exists \{ \bm{U}_i \}_i \quad \text{s.t.}\quad \bm{U}=\bigcup_i \bm{U}_i  \quad\text{and} \quad \forall i 
%: \bm{U}_i \in \SmallNamedFunction{InjectableSets}{G'}\quad \text{and}\quad \forall i,j
%: \bm{U}_i \aindep \bm{U}_j.
\end{align}
%Equivalently, a set of nodes is pre-injectable if and only if its (weakly) connected components are injectable. 
Note that every injectable set is a trivial example of a pre-injectable set.

Because ancestral independence in the DAG implies statistical independence for any probability distribution compatible with the DAG, it follows that  if 
$\bm{U}$ is a pre-injectable set and $\bm{U}_1,\bm{U}_2,\ldots,\bm{U}_n$ are the ancestrally independent components of $\bm{U}$, then 
\begin{align}%\label{eq:preinjfactor}
%\forall \bm{U}\in\SmallNamedFunction{PreInjectableSets}{G'}:	
\pdf[C']{\bm{U}} = \pdf[C']{\bm{U}_1} \pdf[C']{\bm{U}_2}  \cdots \pdf[C']{\bm{U}_n}.
\end{align}
Furthermore, because each injectable set of variables $\bm{U}_i$ satisfies Eq.~\eqref{eq:coincidingdistrodef}, it follows that joint distributions on pre-injectable sets in the inflation model can be expressed in terms of distributions defined on the original causal model,
\begin{align}\label{eq:preinjfactor}
%\forall \bm{U}\in\SmallNamedFunction{PreInjectableSets}{G'}:	
\pdf[C']{\bm{U}} = \pdf[C]{\subsim{\bm{U}}_1} \pdf[C]{\subsim{\bm{U}}_2}  \cdots \pdf[C]{\subsim{\bm{U}}_n}.
\end{align}
This latter property implies that the observable probability distribution over any pre-injectable set in the inflation model is fully specified by the \sout{parameters} \purp{observable probability distribution} \sout{in the original causal model} \purp{per the original observable data}.  Indeed, this relation is what motivates us to consider the pre-injectable sets.

Consider a causal inference problem where the observational input is a probability distribution over a set $\bm{O}$ of observed variables, denoted $\pdf{\bm{O}}$ and the causal hypothesis is $H_{G,S}$ where $\SmallNamedFunction{ObservedNodes}{G}=\bm{O}$.  Suppose that one seeks only to determine whether the causal hypothesis is consistent with the observational input.  
%Then the following condition is necessary and sufficient for such consistency
This holds if and only if
\begin{align}
\exists C \in H_{G,S} : \pdf[C]{\SmallNamedFunction{ObservedNodes}{G}}= \pdf{\bm{O}}.
\end{align}

%Consider a causal inference problem where the input data is a joint distribution over a set of observed variables.  A causal model is only a candidate for a causal explanation of this joint distribution if it posits a DAG $G$ for which $\SmallNamedFunction{ObservedNodes}{G}$ corresponds to the set of observed variables, so we assume this henceforth. The causal hypothesis $H_{G,S}$ is consistent with data $\pdf{\SmallNamedFunction{ObservedNodes}{G}}$ if and only if there exists a causal model satisfying the constraints of the hypothesis, $C \in H_{G,S}$, such that $\pdf[C]{\SmallNamedFunction{ObservedNodes}{G}} = \pdf{\SmallNamedFunction{ObservedNodes}{G}}$. 



%Suppose the original causal inference problem takes as its inputs  a probability distribution over a set $\bm{O}$ of observed variables, denoted $\pdf{\bm{O}}$, and a causal hypothesis, denoted $H_{G,S}$, and seeks to determine whether these are consistent.  

Every inflation $G\to G'$
%DAG $G' \in \SmallNamedFunction{Inflations}{G}$
 defines a new causal inference problem for which the decision regarding consistency is the same as for the original problem.  

The causal hypothesis of the new causal inference problem is simply the image under a $G\to G'$ inflation of the causal hypothesis of the original causal inference problem, and one seeks to evaluate its consistency with observational data that is determined by the observational data of the original in the following way.
%: for all pre-injectable sets on $G'$ consisting entirely of observed nodes on $G'$, the distribution over the set is given by the observed distribution over the corresponding nodes on $G$. 

%We can formalize this as follows. 
Let  $\bm{O}$ denote the image of $\subsim{\bm{O}}$ under the $G \to G'$ inflation, and consider the subsets of $\bm{O}$ that are pre-injectable relative to the $G \to G'$ inflation, denoted $\SmallNamedFunction{PreInjectableSets}{\bm{O}}$.  Recall that for every set of nodes $\bm{U} \in \SmallNamedFunction{PreInjectableSets}{\bm{O}}$ there is a partition $\bm{U} = \bigcup_i \bm{U}_i$ where the $\bm{U}_i$ are subsets of $\bm{O}$ that are injectable relative to the $G \to G'$ inflation and that are ancestrally independent.  Recall also that, by the definition of a $G\to G'$ inflation, for any set of nodes $\bm{U}_i \subseteq \bm{O}$ that is injectable, one can find a set of nodes $\subsim{\bm{U}}_i \subseteq \subsim{\bm{O}}$ such that $\forall  i: \bm{U}_i \sim \subsim{\bm{U}}_i$.  The observational data of the new causal inference problem is that $\forall \bm{U} \in \SmallNamedFunction{PreInjectableSets}{\bm{O}}: \pdf{\bm{U}} = \pdf{\subsim{\bm{U}_1}}\pdf{\subsim{\bm{U}_2}} \dots \pdf{\subsim{\bm{U}_n}}$, where $\pdf{\subsim{\bm{U}_i}}$ is the marginal of $\pdf{\subsim{\bm{O}}}$ on $\subsim{\bm{U}_i}$.

Therefore the observational input to the new causal inference problem is not the distribution on $\bm{O}$, but rather a specification of the marginals of this distribution on each of the pre-injectable sets of $\bm{O}$ under the $G\to G'$ inflation.

Summarizing, we have:
\begin{lemma}
The causal hypothesis $H_{G,S}$ is consistent with the observational data $\pdf{\subsim{\bm{O}}}$ if and only if the image of this causal hypothesis under a $G\to G'$ inflation, denoted $H_{G',S'}$, is consistent with the observational data that $\forall \bm{U} \in \SmallNamedFunction{PreInjectableSets}{\bm{O}}: \pdf{\bm{U}} = \pdf{\subsim{\bm{U}_1}}\pdf{\subsim{\bm{U}_2}} \dots \pdf{\subsim{\bm{U}_n}}$.
\label{mainlemma}
\end{lemma}

It follows that any consistency conditions that one can derive for the new causal inference problem immediately yield consistency conditions for the original causal inference problem.  Indeed, any standard tool of causal inference can be applied to the new problem and conditions derived therefrom provide novel conditions for the original problem.  Any causal inference tool, therefore, can potentially have its power augmented by combining it with inflation.
%The new problem has a rather different form than the old, 



\example{: \tred{Perfect correlation cannot arise from the Triangle scenario.}}\par\smallskip\nobreak

Consider the following causal inference problem.  The observational data is a joint distribution over three binary-outcome variables, $P^{\text{obs}}_{A B C}$, where the marginal on each variable is uniform and the three are perfectly correlated,
\begin{align}\label{eq:ghzdistribution1}
P^{\text{obs}}_{A B C}=\frac{[000]+[111]}{2},\quad\text{i.e.}\quad P^{\text{obs}}_{A B C}(a b c)=\begin{cases}\tfrac{1}{2}&\text{if }\; a\eql b\eql c, \\ 0&\text{otherwise}.\end{cases}
\end{align}
The causal hypothesis is the set of causal models associated to the DAG of the triangle scenario, depicted in \cref{fig:Tri222}.  The problem is to decide whether the observational data is consistent with this causal hypothesis.

To solve this problem, we consider an inflated causal inference problem, namely the problem induced by the particular inflation depicted in Fig.~\ref{fig:TriDagSubA2B1C1}.  

The sets $\brackets{A_2 C_1}$ and $\brackets{B_1 C_1}$ are injectable (hence trivially pre-injectable). The set $\brackets{A_2 B_1}$ is also pre-injectable because the singleton sets $\brackets{A_2}$ and $\brackets{B_1}$ are each injectable and $A_2 \aindep B_1$.
%and they ancestrally independent.  
Therefore, the inflated observational data for our new causal inference problem includes
\begin{align}\begin{split}\label[eqs]{eq:GHZobsinf}
%\p[A_2 C_1]{a c}=\p{A C}{a c} &= \frac{[00]+[11]}{2}\label{j1}\\
%\p[B_1 C_1]{b c}=\p[B C]{b c} &= \frac{[00]+[11]}{2}\label{j2}\\
%\p[A_2 B_1]{a b}=\p[A]{a}\p[B]{b} &= \frac{[0]+[1]}{2}\otimes\frac{[0]+[1]}{2} \label{j3}.
&P^{\text{obs-inf}}_{A_2 C_1} =P^{\text{obs}}_{A C}= \frac{[00]+[11]}{2}\\
&P^{\text{obs-inf}}_{B_1 C_1} =P^{\text{obs}}_{B C}= \frac{[00]+[11]}{2}\\
&P^{\text{obs-inf}}_{A_2 B_1} =P^{\text{obs}}_A \otimes P^{\text{obs}}_B= \frac{[0]+[1]}{2}\otimes\frac{[0]+[1]}{2}. 
\end{split}\end{align}
The question is whether such constraints are consistent with the causal hypothesis that the DAG is that of \cref{fig:TriDagSubA2B1C1}, where the parameters supplementing the DAG are unrestricted but for that the latent variables $Y_1$ and $Y_2$ are identically distributed. 

It is not difficult to see that the answer to the question is negative.  And this verdict can be rendered without even appealing to the form of the inflation DAG (Note, however, that the inflation DAG has played a role in defining the new causal inference problem insofar as it has dictated the form that the inflated observational data should take).  It suffices to make use of results concerning the marginals problem.  
%From Eq.~\eqref{eq:ghzdistribution1} one deduces that the marginal $\pdf{A C}$ described perfect correlation between $A$ and $C$ and similarly for the marginal $\pdf{B C}$.  But then, by Eqs.~\eqref{j1} and \eqref{j2}, one deduces that there is perfect correlation between $A_2$ and $C_1$ and perfect correlation between $B_1$ and $C_1$.  Meanwhile, Eq.~\eqref{j3} dictates that $A_2$ and $B_1$ should be uncorrelated.  But 

There is no joint distribution $P_{A_2 B_1 C_1}$ having such marginal as given in \cref{eq:GHZobsinf}. The only joint distribution that exhibits perfect correlation between $A_2$ and $C_1$ and between $B_1$ and $C_1$ also exhibits perfect correlation between $A_2$ and $B_1$.  

We have therefore certified that perfect correlations among $A$, $B$ and $C$ is not consistent with the Triangle causal structure, recovering the seminal result of \citet{steudel2010ancestors}.


%Let us ask, is it possible for the three original-scenario observable variables $\{A,B,C\}$ to be random but perfectly correlated? We call this the GHZ-type distribution, after an eponymous tripartite entangled quantum state \cite{greenberger1989going,3Qubits2Ways}. So $P_{\text{GHZ}}\parenths{A B C}$ is given by
%\begin{align}\label{eq:ghzdistribution1}
%p_{\text{GHZ}}\parens{a b c}=\frac{[000]+[111]}{2}=\begin{cases}\tfrac{1}{2}&\text{if }\; a=b=c, \\ 0&\text{otherwise}.\end{cases}
%\end{align}
%We assume uniform binary variables for the sake of concreteness, but the argument is general. Let's temporarily (falsely) assume $P_{\text{GHZ}}$ to be compatible with the triangle scenario. Since $\brackets{A_2 C_1}$ is an injectable set we have $\pdf{A_2 C_1}=\pdf{A C}$, and therefore $P_{\text{GHZ}}$ implies that $A_2$ and $C_1$ are perfectly correlated. Similarly, since $\brackets{B_1 C_1}$ is an injectable set we have $\pdf{B_1 C_1}=\pdf{A C}$, and therefore $B_1$ and $C_1$ must be perfectly correlated, and by extension $B_1$ and $A_2$ are perfectly correlated as well. On the other hand, $A_2$ and $B_1$ must be statistically independent, as they do not share any common ancestor. The only way for two variables to be \emph{both} perfectly correlated and independent is by being deterministic. This is not the case in $P_{\text{GHZ}}$, and thus we have certified that $P_{\text{GHZ}}$ is not realizable from the Triangle causal structure, recovering the seminal result of \citet{steudel2010ancestors}.


\example{: \tred{The W-type distribution cannot arise from the Triangle scenario.}}\par\smallskip\nobreak

Consider another causal inference problem concerning the triangle scenario, namely, that of determining whether the triangle DAG can explain a joint distribution $P^{\text{W}}_{A B C}$ of the form
\begin{align}\label{eq:wdistribution1}
P^{\text{W}}_{A B C}=\frac{[100]+[010]+[001]}{3},\quad\text{i.e.}\quad P^{\text{W}}_{A B C}(a b c)=\begin{cases}\tfrac{1}{3}&\text{if }\; a\cramp{+}b\cramp{+}c\eql 1, \\ 0&\text{otherwise}.\end{cases}
\end{align}
We call this the W-type distribution\footnote{Because its correlations are reminiscent of those one obtains for the quantum state appearing in Ref. \cite{3Qubits2Ways}, and which is called the W state.}. To our knowledge, the fact that the Triangle causal structure is inconsistent with the W-type distribution has not been demonstrated previously.

To prove the inconsistency of  $P^{\text{W}}$ with the Triangle causal structure, we consider the inflation DAG of \cref{fig:Tri222}. 
The sets $\{A_2 C_1\}$, $\{B_2 A_1\}$, $\{A_2 B_1\}$ and $\{ A_1 B_1 C_1\}$ are injectable.  The set $\{ A_2 B_2 C_2\}$ is pre-injectable because the singleton sets $\{A_2\}$,  $\{B_2\}$ and $\{C_2\}$ are injectable and \emph{all} ancestrally independent. It follows that we must consider the inflated observational data
\begin{align}
&P^{\text{W-inf}}_{A_2 C_1}&&=P^{\text{W}}_{A C} &&= \frac{[10]+[01]+[00]}{3} &&&&&&&&\label{W1}\\
&P^{\text{W-inf}}_{B_2 A_1}&&=P^{\text{W}}_{B A} &&= \frac{[10]+[01]+[00]}{3} &&&&&&&&\label{W2}\\
&P^{\text{W-inf}}_{C_2 B_1}&&=P^{\text{W}}_{C B} &&= \frac{[10]+[01]+[00]}{3} &&&&&&&&\label{W3}\\
&P^{\text{W-inf}}_{A_1 B_1 C_1}&&=P^{\text{W}}_{A B C}  &&= \frac{[100]+[010]+[001]}{3} &&&&&&&&\label{W4}\\
&P^{\text{W-inf}}_{A_2 B_2 C_2}&&=P^{\text{W}}_{A}\otimes P^{\text{W}}_{B}\otimes P^{\text{W}}_{C} &&= \frac{[1]+2\cdot[0]}{3}\otimes\frac{[1]+2\cdot[0]}{3}\otimes\frac{[1]+2\cdot[0]}{3} &&&&&&&&\label{W5}.
\end{align}
Eq.~\eqref{W1} %together with the form of $P_{\text{W}}$ 
implies that $C_1\eql 0$ whenever $A_2 \eql 1$. Similarly, $A_1\eql 0$ whenever $B_2\eql 1$ and $B_1\eql 0$ whenever $C_2\eql 1$. Eq.~\eqref{W5} %and the form of $P_{\text{W}}$ 
implies that $A_2$,$B_2$, and $C_2$ are uncorrelated and uniformly distributed, which means that sometimes they all take the value 1. 
%, that is, $A_2=1$ and $B_2=1$ and $C_2=1$.  
Summarizing, we have
\begin{align*} 
&A_2 \eql 1 \implies C_1\eql 0\\
&B_2\eql 1 \implies A_1\eql 0\\
&C_2 \eql 1 \implies B_1 \eql 0\\
\text{and sometimes}
%\footnote{In fact, this occurs with probability 1/8, though it is sufficient for the argument to note that it is nonzero.}
 \quad &A_2 \eql 1, B_2 \eql 1, C_2 \eql 1.
\\
\shortintertext{But these constraints clearly imply that}
\text{sometimes} \quad &A_1 \eql 0, B_1 \eql 0, C_1 \eql 0,
\end{align*}
which contradicts Eq.~\eqref{W4}.% and the fact that $P_{\text{W}}$ has no weight on $[000]$.

Again, we have reached our verdict here simply by noting that the distributions defined in Eqs.~\eqref{W1}-\eqref{W5} cannot arise as the marginals of a single distribution on $A_1, B_1, C_1, A_2, B_2, C_2$.  Specifically, we have leveraged an approach to the marginal problem inspired by Hardy's version of Bell's theorem \cite{L.Hardy:PRL:1665,Mansfield2012}, see \cref{sec:TSEM} for further discussion of Hardy-type paradoxes and their applications.



%$\p{A_2\eql B_2\eql C_2\eql 1}=\nicefrac{1}{8}$ according to $P_{\text{W}}$. But that would imply $\p{A_1\eql B_1\eql C_1\eql 0}\geq\nicefrac{1}{8}$, which contradicts $P_{\text{W}}$, hence $P_{\text{W}}$ cannot arise from the Triangle scenario.

The W-type distribution is difficult to witness as unrealizable using conventional causal inference techniques.
\begin{compactenum}
\item There are no conditional independence relations between the observable nodes of the Triangle scenario. %, so the distribution is observationally Markov with respect to the DAG. 
\item Shannon-type entropic inequalities cannot detect this distribution as not allowed by the Triangle scenario~\cite{fritz2013marginal,chaves2014novel,chaves2014informationinference}. 
\item Moreover, \emph{no} entropic inequality can witness the W-type distribution as unrealizable. \citet{weilenmann2016entropic} have constructed an inner approximation to the entropic cone of the Triangle causal structure, and the W-distribution lies inside this. In other words, a distribution with the same entropic profile as the W-type distribution \emph{can} arise from the Triangle scenario.
\item The newly-developed method of covariance matrix causal inference due to \citet{kela2016covariance}, which gives tighter constraints than entropic inequalities for the Triangle scenario, also cannot detect inconsistency with the W-type distribution.
\end{compactenum}
\par\noindent But the inflation technique can, and does so very easily.

\example{: \tred{The PR-box cannot arise from the Bell scenario.}}\par\smallskip\nobreak

Consider the causal structure associated to the Bell \cite{bell1964einstein,Brunner2013Bell,bell1966lhvm,CHSHOriginal} scenario [\citealp{pusey2014gdag}~(Fig.~E\#2), \citealp{WoodSpekkens}~(Fig.~19), \citealp{chaves2014novel}~(Fig.~1), \citealp{BeyondBellII}~(Fig.~1), \citealp{wolfe2015nonconvexity}~(Fig.~2b), \citealp{steeg2011relaxation}~(Fig.~2)], depicted here in \cref{fig:NewBellDAG1}. The observable variables are $\brackets{A,B,X,Y}$, and $\Lambda$ is the latent common cause of $A$ and $B$. 

\begin{figure}[ht]
\centering
\begin{minipage}[t]{0.45\linewidth}
\centering
\includegraphics[scale=1]{BellDagRaw.pdf}
\caption{The causal structure of the a bipartite Bell scenario. The local outcomes of Alice's and Bob's experimental probing is assumed to be a function of some latent common cause, in addition to their independent local experimental settings.}\label{fig:NewBellDAG1}
\end{minipage}
\hfill
\begin{minipage}[t]{0.45\linewidth}
\centering
\includegraphics[scale=1]{BellDagCopy.pdf}
\caption{An inflation DAG of the bipartite Bell scenario, where both local settings variables have been duplicated.}\label{fig:BellDagCopy1}
\end{minipage}
\end{figure}

%P^{\text{obs-inf}}_{A_2 C_1}

We consider the distribution ${P^{\text{obs}}_{A B X Y} = P^{\text{PR}}_{A B | X Y} \otimes P^{\text{free}}_{X} \otimes P^{\text{free}}_{Y}}$, where $P^{\text{free}}_{X}$ and $P^{\text{free}}_{Y}$ are arbitrary full-support distributions\footnote{In the literature on the Bell scenario, these variables are known as ``settings''. Generally, we may think of endogenous observable variables as settings, coloring them light green in the DAG figures. Settings variables are natural candidates for conditioning on.} over the binary variables $X$ and $Y$, and
\begin{align}\begin{split}\label{eq:PRbox}
% & p_{\text{PR}}\parens{a b |x y}=\frac{[00|00]+[11|00]+[00|10]+[11|10]+[00|01]+[11|01]+[01|11]+[10|11]}{8}\\
P^{\text{PR}}_{A B | X Y}\parens{a b |x y}=\begin{cases}\tfrac{1}{2}&\text{if }\; \SmallNamedFunction[2]{mod}{a\cramp{+}b}\eql x\cramp{\cdot} y, \\ 0&\text{otherwise}.\end{cases}
\end{split}\end{align}
The correlations described by this distribution are known as PR-box correlations after Popescu and Rohrlich, and they are well-known to be inconsistent with the the Bell scenario \cite{PROriginal,PRUnit}. Here we prove this inconsistency using the inflation technique. 

We use the inflation of the Bell DAG shown in \cref{fig:BellDagCopy1}.

We begin by recognizing that $\{A_1 B_1 X_1 Y_1\}$, $\{A_2 B_1 X_2 Y_1\}$, $\{A_1 B_2 X_1 Y_2\}$, and $\{A_2 B_2 X_2 Y_2\}$ are all injectable sets, and that $\{X_1 X_2 Y_1 Y_2\}$ is a pre-injectable set because the singleton sets $\{X_1\}$, $\{X_2\}$, $\{Y_1\}$, and $\{Y_2\}$ are all injectable and ancestrally independent of one another.  It follows that the observable data inflates according to
\begin{align}\begin{split}\label[eqs]{eq:PRinflated}\begin{array}{l}
P^{\text{obs-inf}}_{A_1 B_1 X_1 Y_1}=P^{\text{obs}}_{A B X Y}\\%\label{PR1}\\
P^{\text{obs-inf}}_{A_2 B_1 X_2 Y_1}=P^{\text{obs}}_{A B X Y}\\%\label{PR2}\\
P^{\text{obs-inf}}_{A_1 B_2 X_1 Y_2}=P^{\text{obs}}_{A B X Y}\\%\label{PR3}\\
P^{\text{obs-inf}}_{A_2 B_2 X_2 Y_2}=P^{\text{obs}}_{A B X Y}\\%\label{PR4}\\
P^{\text{obs-inf}}_{X_1 X_2 Y_1 Y_2}=P^{\text{free}}_X \otimes P^{\text{free}}_X \otimes P^{\text{free}}_Y \otimes P^{\text{free}}_Y%\label{PR5}.
\end{array}\quad\text{and hence}\quad\begin{matrix}
&P^{\text{obs-inf}}_{A_1 B_1 |X_1 Y_1}=P^{\text{PR}}_{A B |X Y}\\%\label{PR1b}\\
&P^{\text{obs-inf}}_{A_2 B_1 |X_2 Y_1}=P^{\text{PR}}_{A B |X Y}\\%\label{PR2b}\\
&P^{\text{obs-inf}}_{A_1 B_2 |X_1 Y_2}=P^{\text{PR}}_{A B |X Y}\\%\label{PR3b}\\
&P^{\text{obs-inf}}_{A_2 B_2 |X_2 Y_2}=P^{\text{PR}}_{A B |X Y}%\label{PR4b}.
\end{matrix}.\end{split}\end{align}

Given the assumption that the distributions $P^{\text{free}}_{X}$ and $P^{\text{free}}_{Y}$ are full support, it follows from \cref{eq:PRinflated} that sometimes $\{X_1, X_2, Y_1, Y_2\}=\{0, 1, 0, 1\}$.  
%$X_1$, $X_2$, $Y_1$, and $Y_2$ are all independent variables.
% $P_{\text{PR}}$ is given as a conditional distribution because $X$ and $Y$ are settings variables.
For these values, \cref{eq:PRinflated} specifies perfect correlation between $A_1$ and $B_1$.  Similarly, it the inflated observational data also specified perfect correlation between $A_1$ and $B_2$, perfect correlation between $A_2$ and $B_1$, and perfect \emph{anti}correlation between $A_2$ and $B_2$. Those four requirements are not mutually compatible, however: since perfect correlation is transitive, the first three properties entail perfect correlation between $A_2$ and $B_2$. 

The mathematical structure of this proof parallels that of standard proofs of the inconsistency of PR-box correlations with the Bell structure.  Standard proofs focus on a set of variables $\{A_0, A_1, B_0, B_1\}$ where $A_0$  is the value of $A$ when $X=0$, and similarly for the others.  The fact that there must be a joint distribution over these variables can be inferred from the structure of the Bell DAG and the fact that one can assume without loss of generality that the causal dependencies are deterministic (a result known as Fine's theorem \cite{FineTheorem}).  It is then sufficient to note that the marginals given by the PR-box correlations do not arise from any joint distribution.  Nonetheless, our proof is conceptually distinct insofar as the variables to which we apply the marginals problem are not conditioned on a setting value.  And we do not require Fine's theorem.  

Many causal inference techniques can be used to derive sufficient conditions for the inconsistency of the causal hypothesis $H_{G',S'}$ with the observational data $\forall \bm{U} \in \SmallNamedFunction{PreInjectableSets}{\bm{O}}: \pdf{\bm{U}} = \pdf{\subsim{\bm{U}_1}}\pdf{\subsim{\bm{U}_2}} \dots \pdf{\subsim{\bm{U}_n}}$. We will call sets of such conditions \tblue{incompatibility witnesses}.  

\color{black}
\section{Example applications of the inflation technique}\label{sec:examplebaddistributions}

Here are some examples of causal incompatibility witnesses for the Triangle scenario which we can derive by considering the inflation DAG of~\cref{fig:TriDagSubA2B1C1}.

For technical convenience, let us assume that all observed variables take values in $\{-1,+1\}$. By virtue of the existence of a joint distribution of $A_2$, $B_1$ and $C_1$, we can conclude~\cite{pitowsky_boole_1994,Pitowsky1989,kellerer_marginal_1964,leggett_garg_1985,araujo_cycle_2013},
\begin{equation}
	\label{eq:polymonogamyraw}
	\langle A_2 C_1\rangle + \langle B_2 C_1 \rangle - \langle A_2 B_1 \rangle \leq 1.
\end{equation}
A consequence of the manner in which observational data is inflated per \cref{fig:TriDagSubA2B1C1}, i.e. the non-distribution-specific relations which appear as part of \cref{eq:GHZobsinf}, we can therefore conclude that if an observable distribution is compatible with the Triangle scenario DAG it must satisfy
\begin{equation}
	\label{eq:polymonogamy}
	\langle A C\rangle + \langle B C\rangle \leq 1 + \langle A\rangle \langle B\rangle,
\end{equation}
which we think of as a sort of monogamy inequality: it is impossible for $C$ to be strongly correlated with both $A$ and $B$, unless $A$ and $B$ are both strongly biased.

Alternatively, we could also assume variables with any number of outcomes and start with the inequality~\cite{fritz2013marginal}
\begin{align}
	I(A_2 : C_1) + I(C_1 : B_1) - I(A_2 : B_1) \leq H(B_1),	
\end{align}
which also simply follows from the existence of a joint distribution of all variables. Again, implicit per \cref{eq:GHZobsinf} is that that the third term vanishes regardless of what the observable distribution might be, purely as a consequence of \emph{how} observable distributions are inflated pursuant to \cref{fig:TriDagSubA2B1C1}. We therefore derive
\begin{align}\label{eq:monogomyofcorrelations}
	I(A : C) + I(C : B) \leq H(B) ,
\end{align}
which is the original entropic monogamy inequality derived for the Triangle scenario in~\cite{fritz2012bell}. Our rederivation in terms of inflation is essentially the proof of~\citet{pusey2014gdag}.

%To see how this distribution is incompatible with \cref{eq:tritrivial1}, note that for three \emph{identically distributed} (but not independent) binary variables a further special case of \cref{eq:tritrivial1} is
%\begin{align*}\begin{split}
%&\hspace{-6ex}3\p{A\cramp{=}1}\leq 1+\p{A\cramp{=}1}^2+2\p{A\cramp{=}B\cramp{=}1}.
%\end{split}\end{align*}
%For the W-distribution ${\p{A\cramp{=}B\cramp{=}C\cramp{=}0}=0}$, and also ${\p{A\cramp{=}1,B\cramp{=}1}=0}$, yet ${\p{A\cramp{=}1}=\nicefrac{1}{3}}$. As ${(\nicefrac{1}{3})^3\nleq 0}$ we have proven that the W-type distribution is incompatible with the Triangle scenario.
%Earlier works have already shown that the GHZ-type distribution is incompatible with the classical Triangle scenario \cite{steudel2010ancestors,fritz2012bell,chaves2014novel}. Interestingly, however, the entropic monogamy relation $I\parens{A:B}+I\parens{A:C}\leq H\parens{A}$ which rejects the GHZ-type distribution has been shown also to hold if the hidden shared resources are non-classical, even using generalized probabilistic theories \cite[Cor. 24]{pusey2014gdag}. 


\bigskip

Slightly more involved but otherwise analogous considerations can be applied to the inflation DAG of~\cref{fig:Tri222}, where in particular we have the pre-injectable sets $\{A_1 B_1 C_1\}$, $\{A_1 B_2 C_2\}$, $\{A_2 B_1 C_2\}$, $\{A_2 B_2 C_1\}$ and $\{A_2 B_2 C_2\}$, resulting in the factorization relations
\begin{align}\begin{split}\label{eq:tri222fac}
	P_{A_1 B_1 C_1} &= P_{A B C}, \\
	P_{A_1 B_2 C_2} &= P_{A B} \otimes P_{C}, \\
	P_{B_1 C_2 A_2} &= P_{B C} \otimes P_{A}, \\
	P_{A_2 C_1 B_2} &= P_{A C} \otimes P_{B}, \\
	P_{A_2 B_2 C_2} &= P_{A} \otimes P_{B} \otimes P_{C} .
\end{split}\end{align}
In this case, let us assume binary variables with values in $\{0,1\}$. Now we can again use the existence of a joint distribution over all six observable variables, which implies e.g.~the inequality
\begin{align}\label{eq:FritzF3raw}
	P_{A_2 B_2 C_2}(111) \leq P_{A_1 B_1 C_1}(000) + P_{A_1 B_2 C_2}(111) + P_{B_1 C_2 A_2}(111) + P_{A_2  C_1 B_2}(111),
	% Mermin: \langle A_1 B_1 C_1 \rangle - \langle A_2 B_2 C_1 \rangle - \langle A_2 B_1 C_2 \rangle - \langle A_1 B_2 C_2 \rangle \leq 2.
\end{align}
which one can show to be valid for every joint distribution of all six variables simply by checking that it holds on every deterministic assignments of values, from which the general case follows by linearity. Applying the above factorization relations turns this into the polynomial inequality
\begin{equation}\label{eq:FritzF3}
	P_{A}(1) P_{B}(1) P_{C}(1) \leq P_{ABC}(000) + P_{AB}(11) P_C(1) + P_{BC}(11) P_A(1) + P_{AC}(11) P_B(1),
	% Mermin-poly: \langle A B C\rangle \leq 2 + \langle A B\rangle \langle C\rangle + \langle B C \rangle \langle A\rangle + \langle C A\rangle \langle B\rangle,
\end{equation}
which is yet another necessary condition for compatibility with the Triangle scenario causal structure, and now takes genuine three-way correlations into account. 
%&\qquad
% \underbrace{\begin{array}{l}
% \p{a_1 b_2 c_2}\to \p{c_2} \p{a_1 b_2} \\
% \p{a_2 b_1 c_2}\to \p{a_2} \p{b_1 c_2} \\
% \p{a_2 b_2 c_1}\to \p{b_2} \p{a_2 c_1} \\
% \p{a_2 b_2 c_2}\to \p{a_2} \p{b_2} \p{c_2} \\
%\end{array}}_{\text{via ancestral independence}}   \quad \text{ and }\quad
%\underbrace{\begin{array}{l}
%\p{a_1 b_2} \to \p{a b} \\
%\p{a_2 c_1} \to \p{a c} \\
%\p{b_1 c_2} \to \p{b c} \\
%\p{\n{a}_1 \n{b}_1 \n{c}_1}\to \p{\n{a} \n{b} \n{c}}  \\
%\end{array}\,.}_{\text{via injectable sets}}
%\end{align}
A consequence of this inequality is again that the W-type distribution per \cref{eq:wdistribution1}
%\begin{align}\label{eq:wdistribution}
%{P_{\text{W}}}\parenths{A B C} \coloneqq\quad p_{\text{W}}\parens{a b c}=\frac{[100]+[010]+[001]}{3}=\begin{cases}\tfrac{1}{3}&\text{if }\; a+b+c=1 \\ 0&\text{otherwise}\end{cases}
%\end{align}
is found to be inconsistent with the Triangle scenario, since the right-hand side vanishes but the left-hand side does not.
% \cref{eq:FritzF3} requires the use of a broadcasting inflation, and therefore does not hold in the context of general probability theories.
%, where $a,b,c\in\brackets{0,1}$.
%The W-distribution states that the in any event in which $A,B,C$ are observed, precisely one of them will be found to equal $1$ while the other two will equal $0$. The identity of the variable which takes the value $1$ is uniformly random. In informal but intuitive notation, the W-type distribution is ${\nicefrac{1}{3}[100]+\nicefrac{1}{3}[010]+\nicefrac{1}{3}[001]}$.
%To see how this distribution is incompatible with \cref{eq:FritzF3}, note that for three \emph{identically distributed} (but not independent) binary variables a further special case of \cref{eq:FritzF3} is
%\begin{align*}\begin{split}
%&\hspace{-6ex}\p{A\cramp{=}1}^3\leq \p{A\cramp{=}B\cramp{=}C\cramp{=}0} + 3\times\p{A\cramp{=}B\cramp{=}1}\p{C\cramp{=}1}.
%\end{split}\end{align*}
%For the W-distribution ${\p{A\cramp{=}B\cramp{=}C\cramp{=}0}=0}$, and also ${\p{A\cramp{=}1,B\cramp{=}1}=0}$, yet ${\p{A\cramp{=}1}=\nicefrac{1}{3}}$. As ${(\nicefrac{1}{3})^3\nleq 0}$ we have proven that the W-type distribution is incompatible with the Triangle scenario.



\cref{sec:38ineqs} provides a list of further polynomial inequalities that we have derived for the Triangle scenario using the method developed in the following section.

\section{Deriving polynomial inequalities systematically}
\label{sec:ineqs}

\purp{T: A lot of the things in this section \emph{also} need to be done when checking for satisfiability only, such as identifying the pre-injectable sets and writing down all the constraints.}

We have defined causal inference as a decision problem, namely testing the consistency of some observational data with some some causal hypothesis. We have shown that this decision can sometimes be negatively answered by proxy, namely by demonstrating inconsistency of \emph{inflated} data with an \emph{inflated} hypothesis. But the inflation technique is more useful than that: we can also derive explicit \emph{inequalities} that a distribution on observable nodes needs to satisfy in order to be compatible with the causal structure. % Any such constraint is also an implicit consequence of the original hypothesis, and hence a relevant infeasibility criterion.

The ``big'' problem, therefore, is rather straightforward: We seek to derive inequalities which from the inflation hypothesis and translate them back into inequalities constraining the original distribution. Deriving inequalities for inflation models, however, is just a special instance of generic causal inference: Given some causal hypothesis, what can we say about how it constrains possible observable marginal distributions? Any technique for deriving incompatibility witnesses is therefore relevant when using the inflation technique. Interestingly, weak constraints from the inflation hypothesis often translate into stronger constraints pursuant to the original hypothesis.

In the discussion that follows, we continue to assume that the original hypothesis is nothing more than supposing the causal structure to be given by the original DAG. Furthermore we presume that the joint distribution over all original observable variables is accessible. Moreover, we limit our attention to deriving polynomial inequalities in terms of probabilities, although inflation could also be used to derive entropic inequalities (in particular non-Shannon-type inequalities, as in~\cref{sec:NonShannon}).

% The potential of using inflation as tool for deriving entropic inequalities is considered separately in \cref{sec:NonShannon}.

In what follows we consider three different strategies for constraining possible marginal distributions from the inflation hypothesis. 
\begin{compactitem}
\item The full nonlinear strategy attempts to leverage different kinds of constraints which follow from the inflation hypothesis. This strategy yields the strongest incompatibility witnesses, but relies on computationally-difficult nonlinear quantifier elimination.
\item An intermediate strategy asks only if the various marginal distributions are compatible with \emph{any} joint distribution, without regard to the specific causal structure of the inflation DAG whatsoever. Solving the marginal problem amounts to determining all the facets of the \tblue{marginal polytope}, for which we discuss algorithms in~\cref{sec:projalgorithms}. The resulting incompatibility witnesses are nevertheless still polynomial inequalities at the level of the original distribution.
\item Since computing all facets of the marginal polytope is computationally costly, one can try to derive instead only a collection of linear inequalities which bound it. One strategy for doing so is based on possibilistic Hardy-type paradoxes, which we connect to the hypergraph transversal problem. This strategy requires the least computational effort, but is limited in that it only yields polynomial inequalities of a very particular form.
\end{compactitem}

In the narrative below the marginal problem is discussed first; the nonlinear strategy is presented as supplementing the marginal problem with additional constraints. The most computationally efficient strategy is presented as a relaxation of the marginal problem, and is discussed separate from the other two strategies, namely in \cref{sec:TSEM}.

Preliminary to every strategy, however, is the identification of the pre-injectable sets.


\topic{\tred{Identifying the Pre-Injectable Sets}}\label{step:findpreinjectable}

To identify the pre-injectable sets, we first identify the \emph{injectable} sets. To this end, it is useful to construct an auxiliary undirected graph from the inflation DAG which we call the \tblue{injection graph}. Let the nodes of the injection graph be the observable nodes in the inflation DAG. Two nodes $A_i$ and $B_j$ are adjacent if  $\An{A_i B_j}$ does not contain two distinct nodes that are equivalent up to copy index. The injectable sets are then precisely the cliques\footnote{A clique is a subset of nodes such that every node in the subset is adjacent to every other node in the subset.} in this graph, per \cref{eq:definjectable}. Note that it is usually necessary to enumerate \emph{all} the nonempty cliques, i.e.~not only the maximal ones.

Determining the pre-injectable sets from there can be done via constructing another graph that we call the \tblue{independence graph}. Its nodes are the injectable sets, and we connect two of these by an edge if their ancestral subgraphs are disjoint. Then by definition, the pre-injectable are precisely the cliques in this graph. Taking the union of all the injectable sets represented by such a clique results in a pre-injectable set. Since it is sufficient to only consider the maximal pre-injectable sets, one can eliminate all those pre-injectable sets that are contained in other ones, as a final step.

%Let us also define the \tblue{ancestral dependence graph}, in which two nodes are adjacent if they share a common ancestor, and its complement the \tblue{ancestral independence graph}, in the ancestrally independent nodes are adjacent. To ascertain the factorization of a node set $\bm{U}$ into ancestrally-independent partitions one considers the subgraph on  $\bm{U}$ of the ancestral dependence graph: the ancestrally-independent partitions are identically the distinct connected components of that subgraph. By examining the injection graph and the ancestral dependence graph, therefore, one is able to quickly determine all injectable sets and all ancestral independence relations.

%It is also useful to define another auxiliary graph, the \tblue{pre-injection graph} in which a pair of nodes $A_i$ and $B_j$ are connected if either $\An{A_i B_j}$ or if $A_i\aindep B_j$. The pre-injection graph is identically the union of the injection graph with the ancestral independence graph. Any clique in the pre-injection graph is \emph{not} necessarily a pre-injectable set, but every pre-injectable set must correspond to a clique in the pre-injection graph, per \cref{eq:defpreinjectable}. Moreover, maximal-size pre-injectable sets must correspond to maximal cliques in the pre-injection graph. This makes the pre-injection graph a handy tool for determining the pre-injectable sets. We start by enumerating all maximal cliques in the pre-injection graph to obtain candidate pre-injectable sets. Each candidate set is then factored into ancestrally-independent partitions by means of the ancestral dependence graph. A candidate set is a legitimately pre-injectable if and only if all of its ancestrally-independent partitions are themselves injectable. Isolating the genuine pre-injectable sets from the candidates is therefore quite easy, especially since the complete set of injectable sets is already known.

\begin{figure}[t]
\centering
\begin{minipage}[b]{0.4\linewidth}
\centering
\includegraphics[scale=1]{injectiongraph222.pdf}
\caption{The auxiliary injection graph corresponding to the inflation DAG in \cref{fig:Tri222}, wherein a pair of nodes are adjacent iff they are pairwise injectable.}\label{fig:injection222}
\end{minipage}
%\hfill
%\begin{minipage}[b]{0.3\linewidth}
%\centering
%\includegraphics[scale=1]{ancestraldependancegraph222.pdf}
%\caption{An auxiliary ancestral dependance graph corresponding to the inflation DAG in \cref{fig:Tri222}, wherein a pair of nodes are adjacent iff they share a common ancestor.}\label{fig:dependances222}
%\end{minipage}
%\hfill
\hfill
\begin{minipage}[b]{0.5\linewidth}
\centering
\includegraphics[scale=0.25]{simplicialcomplex.pdf}
\caption{The simplicial complex of pre-injectable sets for the inflation of~\cref{fig:Tri222}. The 5 faces correspond to the maximal pre-injectable sets, namely $\{A_1 B_1 C_1\}$, $\{A_1 B_2 C_2\}$, $\{A_2 B_1 C_2\}$, $\{A_2 B_2 C_1\}$ and $\{A_2 B_2 C_2\}$.}\label{fig:simplicialcomplex222}
\end{minipage}
\end{figure}

For example, applying these prescriptions to the inflation DAG in \cref{fig:Tri222} results in the injection graph of~\cref{fig:injection222} and maximal pre-injectable sets as follows:
\begin{align}\label{eq:basicsetup222}
{\underbrace{\begin{matrix}
A_2\aindep B_1\\
A_2\aindep C_2\\
B_2\aindep A_2\\
B_2\aindep C_1\\
C_2\aindep A_1\\
C_2\aindep B_2
\end{matrix}}_{\substack{\text{adjacent nodes in}\\\text{the injection graph}}}}
\qquad\qquad
{\underbrace{\begin{matrix}
\\ \\
\brackets{A_1},\:\brackets{B_1},\:\brackets{C_1},\\
\brackets{A_2},\:\brackets{B_2},\:\brackets{C_2},\\
\brackets{A_1 B_1},\:\brackets{A_1 C_1},\:\brackets{B_1 C_1},\\
\brackets{A_1 B_2}.\:\brackets{A_2 C_1},\:\brackets{B_1 C_2},\\
\brackets{A_1 B_1 C_1}
\end{matrix}}_{\substack{\text{injectable sets}}}}
%\qquad\quad
%{\underbrace{\begin{matrix}
%\\ \\
%\brackets{A_2}\aindep \brackets{B_1 B_2 C_2}\\
%\brackets{B_2}\aindep \brackets{A_2 C_1 C_2}\\
%\brackets{C_2}\aindep \brackets{A_1 A_2 B_2}\\
%\brackets{A_2}\aindep \brackets{B_2}\aindep \brackets{C_2}
%\end{matrix}}_{\substack{\text{maximal}\\\text{ancestral}\\\text{independencies}}}}
%\qquad\quad
%{\underbrace{\begin{matrix}
%\brackets{ A_1 B_1 }\\
%\brackets{ B_1 C_1 }\\
%\brackets{ A_1 C_1 }\\
%\brackets{ A_2 C_1 }\\
%\brackets{ B_2 A_1 }\\
%\brackets{ C_2 B_1 }\\
%\end{matrix}}_{\substack{\text{pairwise}\\\text{injectable}\\\text{sets}}}}
\qquad\qquad
{\underbrace{\begin{matrix}
\\
\brackets{A_1 B_1 C_1} \\
\brackets{A_1 B_2,\, C_2} \\
\brackets{B_1 C_2,\, A_2} \\
\brackets{C_1 A_2,\, B_2} \\
\brackets{A_2,\, B_2,\, C_2}
\end{matrix}}_{\substack{\text{maximal}\\\text{pre-injectable}\\\text{sets}}}}
\end{align}
where we have placed commas in the maximal pre-injectable sets in order to indicate how each of these arises as a disjoint union of injectable sets with disjoint ancestry, which one can read off from the corresponding clique in the independence graph. This results in the distributions on the pre-injectable sets being given in terms of distribution on the original DAG via
\begin{align}\label{eq:preinjectableuses222}
\begin{split}
&\forall{a b c}:\; \begin{cases}
\pdf[A_1 B_1 C_1]{a b c} = \pdf[A B C]{a b c} \\
\pdf[A_1 B_2 C_2]{a b c} = \pdf[C]{c}\pdf[A B]{a b}\\%\pdf{C}\pdf{A B}\\
\pdf[A_2 B_1 C_2]{a b c} = \pdf[A]{a}\pdf[B C]{b c}\\
\pdf[A_2 B_2 C_1]{a b c} = \pdf[B]{b}\pdf[A C]{a c}\\
\pdf[A_2 B_2 C_2]{a b c} = \pdf[A]{a}\pdf[B]{b}\pdf[C]{c}
\end{cases}\end{split}
\end{align}
\cref{eq:preinjectableuses222} is an equivalent restatement of \cref{eq:tri222fac}. Having identified the pre-injectable sets (and how to use them), we next consider various ways to invoke constraints on the distributions over the pre-injectable sets.


\topic{\tred{Constraining Possible Distributions over Pre-Injectable Sets via the Marginal Problem}}\label{step:marginalsproblem}

The most trivial constraint on possible marginal probabilities, regardless of causal structure,  
%on the pre-injectable set
is simply the \emph{existence of some joint probability distribution} from which the marginal distributions can be recovered through marginalization. This isn't really causal inference---as no nontrivial causal hypothesis is considered---but rather more of a preliminary sanity check. If the marginal distributions are not \tblue{consistent}, then the answer to ``Can these marginal distributions be explained by this particular causal hypothesis?'' is automatically ``No''. This type of problem is known as a \tblue{marginal problem}. For some of the long history of marginal problems and further references, see~\cite{fritz2013marginal}; for a more recent account using the language of presheaves, see~\cite{abramsky_contextuality_2011}.

More precisely, a \emph{marginal problem} consists of a finite set $\bm{X}$ containing the variables to be considered together with a collection of subsets $\bm{U}_1,\ldots,\bm{U}_n$ called \tblue{contexts}, as well as a specification of the (finite) number of outcomes that each variable is allowed to take. It may help to draw the visualize contexts in terms of the simplicial complex that they generate, as in~\cref{fig:Tri222}. Now every joint distribution $P_{\bm{X}}$ restricts to a collection of marginal distributions $(P_{\bm{U}_1},\ldots,P_{\bm{U}_n})$. The marginal problem is concerned with the converse: given a collection marginal distributions $(P_{\bm{U}_1},\ldots,P_{\bm{U}_n})$, under what conditions can one find a joint distribution $\hat{P}_{\bm{X}}$ which has all the given distributions as marginals, as $P_{\bm{U}_i} = \hat{P}_{\bm{U}_i}$ for all $i$?

There is a simple necessary condition: in order for $\hat{P}_{\bm{X}}$ to exist, the marginals clearly must be consistent, in the sense that marginalizing $P_{\bm{U}_i}$ and $P_{\bm{U}_j}$ to the variables in $\bm{U}_i\cap\bm{U}_j$ results in the same distribution\footnote{This is difficult to express explicitly in our notation, since $P_{\bm{U}_i\cap \bm{U}_j}$ could refer to either marginal.}. As we have already seen in the previous section, in many cases this is not sufficient\footnote{Depending on how the contexts intersect with one another, this \emph{may} be sufficient. A criterion for when this occurs has been found by~\citet{vorobev_extension_1960}.}. So what are the necessary and sufficient conditions?

To answer this question, it helps to realize two things:
\begin{itemize}
	\item Every joint distribution $P_{\bm{X}}$ is a convex combination of deterministic assignments of values to all variables (delta distributions), and conversely.
	\item The map $P_{\bm{X}}\to (P_{\bm{U}_1},\ldots,P_{\bm{U}_n})$ is linear.
\end{itemize}
Hence the image of the map $P_{\bm{X}}\to (P_{\bm{U}_1},\ldots,P_{\bm{U}_n})$ is exactly the convex hull of the deterministic assignments of values at the level of marginals. Since there are only finitely many such deterministic assignments, this convex hull is a polytope called the \tblue{marginal polytope}~\cite{kahle_marginal_2010}. Together with the above set of equations, the facet inequalities of the marginal polytopes form necessary and sufficient conditions for the marginal problem to have a solution.

Thus solving the marginal problem is an instance of a facet enumeration problem, or equivalently a linear quantifier elimination problem;~\cref{sec:projalgorithms} gives an overview of how to solve this in practice. Given a facet of the marginal polytope---or any other linear inequality that bounds it---we can construct a polynomial inequality for our original causal inference problem by plugging in the factorization relations of Lemma~\ref{mainlemma}. Doing the same with the equations of coinciding submarginals shows that these are trivially satisfied, and thus it is only the inequalities that are of interest to us.

% quantifier-free form in terms of inequalities such that satisfaction of all such inequalities is necessary and sufficient for marginal compatibility. An efficient algorithm to solve the marginal problem is given in \cref{sec:projalgorithms}. The marginal problem comes up in a variety of applications, and has been studied extensively; see~\cite{fritz2013marginal} for further references.

As an example, here's how the marginal problem can be phrased as a linear quantifier elimination problem in the case of the five three-variable marginal distributions corresponding to the pre-injectable sets in \cref{eq:preinjectableuses222}.
% For simplicity, we assume that all observable variables are binary.
%\footnote{If the observables are not binary, then the resulting  binary-outcome inequalities are necessary for marginal compatibility, in that they should hold for any course-graining of the observational data into two classes, but the binary-outcome inequalities are no longer sufficient.}.
% In order for the five pre-injectable sets in \cref{eq:preinjectableuses222} to be marginally compatible there must exist 64 nonnegative joint probabilities, i.e. satisfying
The putative joint distribution is nonnegative,
\begin{align}\label{eq:nonnegativity}
\forall{a_1 a_2 b_1 b_2 c_1 c_2}:\; \pdf[A_1 A_2 B_1 B_2 C_1 C_2]{a_1 a_2 b_1 b_2 c_1 c_2} \geq 0,
\end{align}
and is required to reproduce the given marginal distributions via
\begin{align}\label[eqs]{eq:marginalequalities222}
\begin{split}
&\forall{a_1 b_1 c_1}:\;\pdf[A_1 B_1 C_1]{a_1 b_1 c_1} = \sum\nolimits_{a_2 b_2 b_2}\pdf[A_1 A_2 B_1 B_2 C_1 C_2]{a_1 a_2 b_1 b_2 c_1 c_2},\\
&\forall{a_1 b_2 c_2}:\;\pdf[A_2 B_2 C_2]{a_1 b_2 c_2} = \sum\nolimits_{a_2 b_1 c_1}\pdf[A_1 A_2 B_1 B_2 C_1 C_2]{a_1 a_2 b_1 b_2 c_1 c_2},\\
&\forall{a_2 b_1 c_2}:\;\pdf[A_2 B_1 C_2]{a_2 b_1 c_2} = \sum\nolimits_{a_1 b_2 c_1}\pdf[A_1 A_2 B_1 B_2 C_1 C_2]{a_1 a_2 b_1 b_2 c_1 c_2},\\
&\forall{a_2 b_2 c_1}:\;\pdf[A_2 B_2 C_1]{a_2 b_2 c_1} = \sum\nolimits_{a_1 b_1 c_2}\pdf[A_1 A_2 B_1 B_2 C_1 C_2]{a_1 a_2 b_1 b_2 c_1 c_2},\\
&\forall{a_2 b_2 c_2}:\;\pdf[A_2 B_2 C_2]{a_2 b_2 c_2} = \sum\nolimits_{a_1 b_1 c_1}\pdf[A_1 A_2 B_1 B_2 C_1 C_2]{a_1 a_2 b_1 b_2 c_1 c_2}.
\end{split}
\end{align}
For example in the case of binary variables, we therefore have 64 inequalities and 40 equalities, although the latter are not all independent. Doing the facet enumeration means eliminating 64 unknowns from those inequalities and equalities, namely any $\p[A_1 A_2 B_1 B_2 C_1 C_2]{\underline{\phantom{xxxxx}}}$, and to thereby compute the inequalities that constrain the marginal probabilities.
%We coin the term gedankenprobability to denote any probability pertaining to a \emph{not}-pre-injectable set of inflation-DAG variables. The gedankenprobabilities evoke thought experiments, because any \purp{black box implementation of the original causal structure % \emph{in-principle}.


Linear quantifier elimination is already widely used in causal inference to derive entropic inequalities \cite{fritz2013marginal,chaves2014novel,chaves2014informationinference}. In that task, however, the quantifiers being eliminated are those entropies which refer to hidden variables. By contrast, the probabilities we consider here are exclusively in terms of observable variables right from the very start (although not in terms of the original observables). The unknowns we eliminate are the not-pre-injectable joint probabilities, which are, at least on first look, quite different from probabilities involving hidden variables;~\cref{sec:Bellscenarios} will partly elucidate the relation.

Note that the marginal problem can be solved by computing \emph{either} a convex hull \emph{or} by eliminating quantifiers from linear inequalities. The derivation of entropic inequalities, by contrast, is strictly a linear quantifier elimination task, and cannot be recast as a convex hull problem. As such, while convex hull enumerations tools \emph{are} useful to derive polynomial inequalities via the inflation technique, they are \emph{not} available for the derivation of entropic inequalities; see \cref{sec:projalgorithms} for further details.

% When solving the marginal problem is too difficult, one may consider solving a relaxation of it, instead. One extremely computationally amenable relaxation of the marginal problem is to enumerate probabilistic Hardy-type paradoxes. This is discussed later on in  \cref{sec:TSEM}.

%\topic{\tred{Constraining Possible Distributions over Pre-Injectable Sets via Hardy Paradoxes}}

%Although solving the marginal problem can be highly optimized, it can still prove computationally difficult. It is therefore sometimes useful to consider relaxations of the marginals problem. The \emph{full} marginals problem is to find inequalities on the marginal distributions such that the inequalities are satisfied \emph{if and only if} the given marginal distributions can be extended. It is much easier to generate necessary-but-insufficient inequalities, i.e. satisfied by all compatible marginal distributions but such that no-violation does not certify marginal compatibility. We have identified a technique for rapidly generating such quantifier-free inequalities by restricting the search to inequalities of a very particular form. We found this alternative technique — trading generality for speed — to be extraordinarily practical. The type of inequalities that we consider are given by a certain class of tautologies in classical propositional logic, see \cref{sec:TSEM} for further details.

\topic{\tred{Constraining Possible Distributions over Pre-Injectable Sets via Conditional Independence Relations}}

The marginal problem asks about the existence of \emph{any} joint distribution which recovers the given marginal distributions. In causal inference, however, there are plenty of other constraints on the sorts of joint distributions which are consistent with some causal hypothesis. The minimal constraint embedded in any causal hypothesis is the idea of causal structure. Thus it is natural to supplement the marginal problem with additional constraints, motivated by causal structure, constraining the hypothetical distribution over observable variables of the inflation DAG.

The most familiar causally-motivated constraints on a joint distribution are \tblue{conditional independence relations}, say among observable variables. Conditional independence relations are inferred by $d$-separation; if $\bm{X}$ and $\bm{Y}$ are $d$-separated in the (inflation) DAG by $\bm{Z}$, then we infer the conditional independence $\bm{X}\bot\bm{Y}|\bm{Z}$. The $d$-separation criterion is explained at length in~\cite{pearl2009causality,studeny2005probabilistic,WoodSpekkens,pusey2014gdag}, so we elect not to review it here.

Every conditional independence relation can be translated into a nonlinear constraint on probabilities, as $\bm{X}\bot\bm{Y}|\bm{Z}$ implies $\p{\bm{x}\bm{y}|\bm{z}}=\p{\bm{x}|\bm{z}}\p{\bm{y}|\bm{z}}$ for all $\bm{x}$, $\bm{y}$, and $\bm{z}$. As we generally prefer to work with unconditional probabilities, we rewrite this as follows: If $\bm{X}$ and $\bm{Y}$ are $d$-separated by $\bm{Z}$, then $\p[\bm{X}\bm{Y}\bm{Z}]{\bm{x}\bm{y}\bm{z}}\p[\bm{Z}]{\bm{z}}=\p[\bm{X}\bm{Z}]{\bm{x}\bm{z}}\p[\bm{Y}\bm{Z}]{\bm{y}\bm{z}}$ for all $\bm{x}$, $\bm{y}$, and $\bm{z}$. Such nonlinear constraints can be incorporated as further restrictions on the sorts of joint distributions consistent with the inflation DAG, supplementing the basic nonnegativity of probability constraints of the marginal problem discussed above.

For example, in \cref{fig:Tri222} we find that $A_1$ and $C_2$ are $d$-separated by $\{A_2 B_2\}$, and so one might incorporate the family of nonlinear equalities $\p[A_1 A_2 B_2 C_2]{a_1 a_2 b_2 c_2}\p[A_2 B_2]{a_2 b_2}=\p[A_1 A_2 B_2]{a_1 a_2 b_2}\p[A_2 B_2 C_2]{a_2 b_2 c_2}$ for all $a_1$, $a_2$, $b_2$ and $c_2$. Every probability that appears in an equation like this must occur as a marginal as well, i.e.~it can be written as a sum of various joint probabilities, as in
\begin{align}
%\begin{array}{lll}
\forall{a_2 b_2}:\;\pdf[A_2 B_2]{a_2 b_2} = \sum\nolimits_{a_1 b_1 c_1 c_2}\pdf[A_1 A_2 B_1 B_2 C_1 C_2]{a_1 a_2 b_1 b_2 c_1 c_2}.
%\forall{a_1 a_2 b_2 c_2}:&\;\pdf[A_1 A_2 B_2 C_2]{a_1 a_2 b_2 c_2} \to& \sum\nolimits_{b_1 c_1}\pdf[A_1 A_2 B_1 B_2 C_1 C_2]{a_1 a_2 b_1 b_2 c_1 c_2},\\
%\forall{a_1 a_2 b_2}:&\;\pdf[A_1 A_2 B_2]{a_1 a_2 b_2} \to\quad& \sum\nolimits_{b_1 c_1 c_2}\pdf[A_1 A_2 B_1 B_2 C_1 C_2]{a_1 a_2 b_1 b_2 c_1 c_2}
%\end{array}
\end{align}
In particular, the number of unknown quantities to be eliminated is still the same, but now the system of equations and inequalities is nonlinear. 
%Note also that incorporating such constraints also increases the number of quantifier which must be eliminated, as additional non-injectable probabilities are now featured in the equalities corresponding to conditional independence which do not appear in the unconstrained marginal problem. 

Many modern computer algebra systems have functions capable of tackling nonlinear quantifier elimination symbolically\footnote{For example \textit{Mathematica$^{_{\textit{\tiny\texttrademark}}}$}'s \href[pdfnewwindow]{http://reference.wolfram.com/language/ref/Resolve.html}{\texttt{Resolve}} command, \textit{Redlog}'s \href[pdfnewwindow]{http://www.redlog.eu/documentation/reals/rlqe.php}{\texttt{rlposqe}}, or \textit{Maple$^{_{\textit{\tiny\texttrademark}}}$}'s \href[pdfnewwindow]{http://maplesoft.com/support/help/Maple/view.aspx?path=RegularChains/SemiAlgebraicSetTools/RepresentingQuantifierFreeFormula}{\texttt{RepresentingQuantifierFreeFormula}}, etc.}. 
%One might then hope to use such software systems to rid the hybrid inequalities of the gedankenprobabilities. 
Currently, however, it is generally not practical to perform nonlinear quantifier elimination on large polynomial systems with many quantifiers. It may help to exploit results on the particular algebraic-geometric structure of these particular systems~\cite{garcia_bayesian_2005}. But also without using quantifier elimination, the nonlinear constraints can be easily accounted for numerically. Upon substituting numerical values for all the injectable probabilities, the former quantifier elimination problem is converted to simpler existence problem: Do there exist joint probabilities that satisfy the full set of linear and nonlinear constraints numerically? Most computer algebra systems can resolve such \emph{satisfiability} questions quite easily\footnote{For example \textit{Mathematica$^{_{\textit{\tiny\texttrademark}}}$} \href[pdfnewwindow]{http://reference.wolfram.com/language/Experimental/ref/ExistsRealQ.html}{\texttt{Reduce\`{}ExistsRealQ}} function. Specialized satisfiability software such as SMT-LIB's \href[pdfnewwindow]{http://smtlib.cs.uiowa.edu/solvers.shtml}{\texttt{check-sat}} \cite{BarFT-SMTLIB} are particularly apt for this purpose.
%One can also exploit the fact than any nonlinear optimizer will return an error when a set of constraints cannot be satisfied. Nonlinear optimizers include \textit{Maple$^{_{\textit{\tiny\texttrademark}}}$}'s \href[pdfnewwindow]{http://www.maplesoft.com/support/help/Maple/view.aspx?path=Optimization/NLPSolveMatrixForm}{\texttt{NLPSolve}}, \textit{Mathematica$^{_{\textit{\tiny\texttrademark}}}$}'s \href[pdfnewwindow]{http://reference.wolfram.com/language/ref/message/NMinimize/nsol.html}{\texttt{NMinimize}}, and dozens of free and commercial optimizers for \href[pdfnewwindow]{http://ampl.com/products/solvers/all-solvers-for-ampl}{\textit{AMPL}} and/or \href[pdfnewwindow]{https://neos-server.org/neos/solvers/index.html\#nco}{\textit{GAMS}}
}.

It is also possible to use a mixed strategy of linear and nonlinear quantifier elimination, such as \citet{ChavesPolynomial} advocates. The explicit results of~\cite{ChavesPolynomial} are therefore consequences of any inflation DAG, achieved by applying a mixed quantifier elimination strategy. \purp{T: I don't see where the ``therefore'' comes from. Are Rafael's results a special case of inflation? If so, we should explain this in one or two sentences, but maybe not here.}

\topic{\tred{Constraining Possible Distributions over Pre-Injectable Sets via Coinciding Marginals}}

Even if the original hypothesis does not constrain possible causal models beyond $d$-separation, The inflation hypothesis is (in all nontrivial cases) more than just that: every inflation model also satisfies $\pdf{A_i| \Pa{A_i}}=\pdf{A_j|\Pa{A_j}}$, per \cref{eq:funcdependences}.
% Consequently, the distributions over different injectable sets must occasionally coincide, i.e. $\pdf{\bm{X}}=\pdf{\bm{Y}}$ whenever both $\bm{X}$ and $\bm{Y}$ are injectable, and $\subsim{\bm{X}}=\subsim{\bm{Y}}$.
This implies cases that distributions over certain sets of nodes must coincide in any inflation model. For example, $P_{\bm{X}} = P_{\bm{Y}}$ certainly holds whenever $\bm{X}$ and $\bm{Y}$ are injectable and $\subsim{\bm{X}} = \subsim{\bm{Y}}$, but adding this constraint to the quantifier elimination problem does not help as both sides of this equations are already determined by the original distribution. However, this type of equation also follows in some cases when $\bm{X}$ and $\bm{Y}$ are not injectable. For example, $P_{A_1 A_2 B_1}=P_{A_1 A_2 B_2}$ follows from \cref{fig:Tri222} and the inflation hypothesis, even though $\brackets{A_1 A_2 B_1}$ and $\brackets{A_1 A_2 B_2}$ are not injectable sets.

Hence equations such as $\forall{a_1 a_2 b}:\;\p[A_1 A_2 B_1]{a_1 a_2 b} = \p[A_1 A_2 B_2]{a_1 a_2 b}$ are also consequnces of the inflation hypothesis, and may be incorporated into either linear or nonlinear quantifier eliminations in order to derive stronger incompatibility witnesses. The details of how to recognize coinciding distributions beyond the obvious coincidences implied by injectable or pre-injectable sets and under what conditions they may yield tighter inequalities are discussed in \cref{sec:coincidingdetails}.



%One may also substitute numeric values for all the pre-injectable probabilities appearing in the marginals problem. Upon doing so, the quantifier elimination problem is converted to a quantifier existence problem: Do there exist gedankenprobabilities that satisfy the resulting system of inequalities? Such \emph{satisfiability} questions can be resolved quite rapidly, especially when the quantifiers are linear \cite{Korovin2012ImplementingCRA,Bobot2012SimplexSAT}. Note that real-world data with uncertainties can also be incorporated into these satisfiability questions. Instead of asserting that a particular probability is equal to a given \emph{value}, one can incorporate new inequalities which constrain the experimentally-known probabilities to lie in given \emph{intervals}. Assigning probabilities to intervals as opposed to numeric values results in further free parameters in the system, but the problem nevertheless remains one of \emph{universal} existential closure, and can be resolved with extreme efficiency.


%One useful alternative to linear quantifier elimination is to identify representative probability distributions which are incompatible with the (unprojected) constraints; \citet{ChavesNoSignalling} use this technique, for example. %That technique essentially translates the elimination problem to a satisfiability problem, and moreover an ultra-efficient \emph{linear} quantifier existence problem at that! In the context of polytope projection, these representative probability distributions correspond to extreme rays of the so-called ``projection cone" \cite{jones2004equality,Jones2008,BalasProjectionCone}.

%A final alternative to linear quantifier elimination is to restrict one's consideration to quantifier-free inequalities with a particular form. We found this alternative technique — trading generality for speed — to be extraordinarily practical. The subtype of causal criteria which can be most rapidly recognized are those which follow from certain tautologies in classical propositional logic, see \cref{sec:TSEM} for further details.




%\purp{It is interesting to compare the technique for deriving polynomial inequalities here to that in in Ref. \cite{ChavesPolynomial}. One the one hand, our initial set of linear inequalities is much stronger, as we work with the inflated DAG whereas \citet{ChavesPolynomial} considers only the original DAG. This allows us to analyze scenarios which Ref. \cite{ChavesPolynomial} cannot, namely those without any observable conditional independence relations, such as the Triangle scenario. On the other hand, Ref. \cite{ChavesPolynomial} purportedly incorporates any kind of observable conditional independence relation, whereas we account only for \emph{unconditional} independence relations, per \cref{step:fac}. A careful examination of Ref. \cite{ChavesPolynomial}, however, reveals that only unconditional independence relations are utilized in all the example there. To that extent, therefore, all the explicit results in Ref. \cite{ChavesPolynomial} are implied by the inflation DAG technique.}

%We note that our polynomial inequalities are related to those introduced recently by \citet{ChavesPolynomial}, in that our inequalities subsume the explicit results of Ref. \cite{ChavesPolynomial}. \purp{NEEDS CONFIRMATION!} \citet{ChavesPolynomial} exploits conditional independence at the observable level only, whereas our technique is applicable to scenarios even without any observable CI relations, such as the Triangle scenario.

\purp{T: yes, this should be moved, to some place where also the second to last paragraph goes. Maybe end of introduction, a paragraph on related work?}
As far as we can tell, our inequalities are not related to the nonlinear incompatibility witnesses which have been derived specifically to constrain classical networks \cite{TavakoliStarNetworks,RossetNetworks,TavakoliNoncyclicNetworks}, nor to the nonlinear inequalities which account for interventions to a given causal structure \cite{kang2007polynomialconstraints,steeg2011relaxation}.


\section{Deriving Hardy-type constraints for the marginal problem}\label{sec:TSEM}

In the literature on Bell inequalities, it has been noticed that incompatibility with the Bell scenario DAG can sometimes be witnessed by only looking at which probabilities are zero and which ones are nonzero. In other words, instead of considering the \emph{probability} of a composite outcome, the inconsistency of some marginal distributions can be evident from considering only the \emph{possibility} or \emph{impossibility} of each composite outcome. This is due to~\citet{L.Hardy:PRL:1665}, and hence such \tblue{possibilistic constraints} are also known as \tblue{Hardy-type paradoxes}, in the sense that the ``paradox'' is said to occur whenever the constraint is violated. For more background on Hardy's paradoxes, see Refs.~\cite{Garuccio95,CabelloHardyInequality,Braun08,Mancinska14,LSW}; a partial classification of Hardy-type paradoxes in Bell scenarios can be found in Ref.~\cite{Mansfield2012}

The usual presentation of a Hardy-type paradox takes the following form: Even though events $E_1, \ldots, E_n$ never occur (are not possible), some other event $E_0$ \emph{does} occur sometimes (is possible). However, the logical relations between these events are such that whenever $E_0$ occurs, then also at least one of $E_1,\ldots,E_n$ occurs, and this is in contradiction with the previous statement.

\newcommand{\smiley}{\textrm{\Smiley}}
\newcommand{\frowny}{\textrm{\Frowny}}

The following example illustrates a Hardy-type paradox. Suppose a plague has suddenly wiped out a population of rats. Three kinds of autopsies are performed on different samples of the dead rats. Autopsies checking specifically for heart and brain disease find that every rat suffered from at least one of those conditions %, but never from \emph{both} conditions,
i.e. the event $[\mathrm{Brain}\eql\smiley,\mathrm{Heart}\eql\smiley]$ is found to be \emph{not possible}. In a slightly different vein, suppose further that autopsies specifically checking for brain and lung diseases find that every dead rat afflicted by brain disease also suffered from lungs disease, i.e. the event $[\mathrm{Brain}\eql\frowny,\mathrm{Lungs}\eql\smiley]$ is not possible. Logic (and the absence of sampling bias) then dictates that an autopsy for heart and lung disease will never find both heart and lungs simultaneously healthy. If some medical examiner checking rats heart and lung conditions then finds that \emph{it is possible} for a rat to have $[\mathrm{Heart}\eql\smiley,\mathrm{Lungs}\eql\smiley]$, this would be considered an occurrence of a Hardy-type paradox.

% Hardy-type paradoxes are predicated on assuming the existence of a solution to the marginal problem. Thus, ensuring the observational data avoids any Hardy-type paradoxes is a necessary condition for solving the marginal problem. \citet[Section~III.C]{LSW} discuss the use of possibilistic constraints to certify the incompatibility of marginal distributions in contexts of than Bell scenarios.

The possibilistic constraint which is equivalent to a Hardy-type paradox is as follows,
\begin{align}
    \SmallNamedFunction{Never}{E_1} \land \ldots \land \SmallNamedFunction{Never}{E_n} \implies \SmallNamedFunction{Never}{E_0},
\end{align}
although it can be expressed more fundamentally in conjuctive form, i.e.
\begin{align}
    \SmallNamedFunction{}{E_0} \implies \SmallNamedFunction{}{E_1} \lor \ldots \lor \SmallNamedFunction{}{E_n}.
\end{align}
Any possibilistic constraint can be immediately translated into a stronger probabilistic one, as noted by \citet{Mansfield2012}. The probabilistic variant states than \emph{whenever} the event $E_0$ occurs than at least one of the events $E_1 .. E_N$ should also occur. Applying the union bound to the probability of the right-hand side we obtain
\begin{align}
\p{E_0}\leq \sum\limits_{j=1}^n{\p{E_j}}.
\end{align}

In order to derive causal incompatibility witnesses via consistency of the marginal distributions, therefore, we may consider the task of \tblue{enumerating} Hardy-type paradoxes, each of which we can then express as an inequality in terms of probabilities. In the following, we explain how to determine \emph{all} such constraints for \emph{any} marginal problem.


To start with a simple example, suppose that we are in a marginal scenario where the pairwise joint distributions of three variables $A$, $B$ and $C$ are given. %, and these two variables are binary with values in $\{0,1\}$. 
One Hardy-type possibilistic constraint which we would want to enumerate is
\begin{align}
%    \bracks{\mgreen{A \eql a}, \mgreen{C \eql c}} \implies \bracks{\mgreen{A \eql a}, B \eql b} \bigvee \bracks{B \eql \n{b}, \mgreen{C \eql c}}
   \bracks{\mgreen{A \eql 1}, \mgreen{C \eql 1}} \implies \bracks{\mgreen{A \eql 1}, B \eql 1} \lor \bracks{B \eql 0, \mgreen{C \eql 1}},
\end{align}
resulting in the probabilistic constraint
\begin{align}\label{eq:trivmarginalconstraint}
	P_{AC}(\mgreen{1 1}) \leq P_{AB}(\mgreen{1} 1) + P_{BC}(0 \mgreen{1}).
\end{align}
\cref{eq:trivmarginalconstraint} is a necessary condition for the existence of a joint distribution of $P_{A B}$, $P_{A C}$, and $P_{B C}$; it is equivalent to \cref{eq:polymonogamyraw} and therefore also implies the polynomial inequality~\cref{eq:polymonogamy}.

%Applying this constraint, together with the inflation technique to the Triangle scenario, results in
%\[
%	\p[AC]{a c} \leq \p[A]{a}\p[B]{b} + \p[BC]{\n{b} c},
%\]
%which is equivalent to~\cref{eq:polymonogamy}.
%
%\purp{I'm in the middle of switching the notation here to lowercase probabilities. ~EW}

We outline the general procedure using a slightly more sophisticated example. Consider the marginal scenario of~\cref{fig:simplicialcomplex222}, where the contexts are $\{A_1 B_1 C_1\}$, $\{A_1 B_2 C_2\}$, $\{A_2 B_1 C_2\}$, $\{A_2 B_2 C_1\}$ and $\{A_2 B_2 C_2\}$, pursuant to \cref{eq:basicsetup222}.
%\begin{align*}
%	\{A_1 B_1 C_1\},\\
%	\{A_1 B_2 C_2\},\\
%	\{A_2 B_1 C_2\},\\
%	\{A_2 B_2 C_1\},\\
%	\{A_2 B_2 C_2\}.
%\end{align*}
Now a possibilistic constraint on this marginal problem consists of a logical implication with one joint outcome as the \tblue{antecedent} and a disjunction of joint outcomes as the \tblue{consequent}. In the following, we explain how to generate \emph{all} such implications which are tight in the sense that their right-hand sides are minimal.

First we fix the antecedant by choosing some context and a composite outcome for it. In order to generate all possibilistic constraints, one will have to perform this procedure for \emph{every} context as the antecedent and every choice of joint outcome thereupon. For the sake of concreteness we take $\bracks{\mgreen{A_2 \eql 1}, \mgreen{B_2 \eql 1}, \mgreen{C_2 \eql 1}}$ to be the fixed antecedent.

The consequent will be a conjunction of composite outcomes in marginal contexts, with the additional property that all outcomes of variables that also occur in the antecedent carry the same outcome. For the implication to be valid, the consequent must further be such that \tred{for any \emph{joint} composite outcome which extends the antecedent's marginal composite outcome, also at least one of the marginal composite outcomes in the consequent must occur.}

To formally determine all valid consequents, we first consider two hypergraphs. %Take the set of nodes to be the disjoint union over all the contexts of all the joint outcomes which are compatible with~\eqref{eq:lhshardy}\footnote{In the left-hand side context there will thus be only one node, and one can also omit this one node without changing the result.}. 
The nodes in the first hypegraph correspond to every possible composite outcome for every possible context.
The hyperedges correspond to every possible joint outcome of all variables. A hyperedge (joint composite outcome) contains a node (marginal composite outcome) iff the hyperedge is an extension of the node; for example the hyperedge $\bracks{A_1 \eql 0, \mgreen{A_2 \eql 1}, B_1 \eql 0, \mgreen{B_2 \eql 1}, C_1 \eql 1, \mgreen{C_2 \eql 1}}$ is an extension of the node $\bracks{A_1 \eql 0,  \mgreen{B_2 \eql 1}, \mgreen{C_2 \eql 1}}$. In our example following \cref{fig:simplicialcomplex222}, this initial hypergraph has $5\cdot 2^3 = 40$ nodes and $2^6 = 64$ hyperedges.

The second hypergraph is a sub-hypergraph of the first one. We delete from the first graph all nodes and hyperedges which contradict the outcomes supposed by the antecedent. For example, the node $\bracks{\mgreen{A_2 \eql 1}, \mred{B_2 \eql 0}, C_1 \eql 1}$ contradicts the antecedent $\bracks{\mgreen{A_2 \eql 1}, \mgreen{B_2 \eql 1}, \mgreen{C_2 \eql 1}}$. We also delete the node corresponding to the antecedent itself. In our example, this final resulting hypergraph has $2^3 + 3\cdot 2^1 = 14$ nodes and $2^3 = 8$ hyperedges.

All valid (minimal) consequents are (minimal) \tblue{transversals} of this latter hypergraph. A transversal is a set of nodes which has the property that it intersects every hyperedge in at least one node. In order to get implications which are as tight as possible, it is sufficient to enumerate only the minimal transversals. Doing so is a well-studied problem in computer science with various natural reformulations and for which manifold algorithms have been developed~\cite{eiter_dualization_2008}.

%It is then possible right-hand sides of the implications are precisely the \tblue{transversals} of this hypergraph, i.e.~the sets of nodes which have the property that they intersect every hyperedge in at least one node. In order to get implications which are as tight as possible, it is sufficient to enumerate only the \tblue{minimal transversals}. Doing so is a well-studied problem in computer science with various natural reformulations and for which manifold algorithms have been developed~\cite{eiter_dualization_2008}. We expect that this enumeration of minimal transversals will be computationally much more tractable than the linear quantifier elimination, even if one does it for every possible left-hand side of the implication.

In our example, it is not hard to check that the right-hand side of
\begin{align}\begin{split}\label{eq:F3implicationform}
	\bracks{\mgreen{A_2 \eql 1}, \mgreen{B_2 \eql 1}, \mgreen{C_2 \eql 1}} \quad\Longrightarrow\quad &\bracks{A_1 \eql 0, B_1 \eql 0, C_1 \eql 0} \lor \bracks{A_1 \eql 1, \mgreen{B_2 \eql 1}, \mgreen{C_2 \eql 1}} \\
	\lor\: & \bracks{\mgreen{A_2 \eql 1}, B_1 \eql 1, \mgreen{C_2 \eql 1}} \lor \bracks{\mgreen{A_2 \eql 1}, \mgreen{B_2 \eql 1}, C_1 \eql 1}
\end{split}\end{align}
is such a minimal transversal: every assignment of values to all variables which extends the assignment on the left-hand side satisfies at least one of the terms on the right, but this ceases to hold as soon as one removes any one term on the right. 

We convert these implications into inequalities in the usual way, by replacing ``$\Rightarrow$'' by ``$\leq$'' at the level of probabilities and the disjunctions by sums. For example the possibilistic constraint \cref{eq:F3implicationform} translates to the probabilistic constraint
\begin{align}\label{eq:F3rawprobform}
%    P_{A_2 B_2 C_2}\parens{\mgreen{a_2} \mgreen{b_2} \mgreen{c_2}} \leq \p{\n{a_1} \n{b_1} \n{c_1}}+\p{a_1 \mgreen{b_2 c_2}}+\p{\mgreen{a_2} b_1 \mgreen{c_2}}+\p{\mgreen{a_2 b_2} c_1}
    P_{A_2 B_2 C_2}\parens{\mgreen{1} \mgreen{1} \mgreen{1}} \leq P_{A_1 B_1 C_1}\parens{1 1 1}+P_{A_1 B_2 C_2}\parens{a_1 \mgreen{1 1}}+P_{A_2 B_1 C_2}\parens{\mgreen{1} 1 \mgreen{1}}+P_{A_2 B_2 C_1}\parens{\mgreen{1 1} 1}
\end{align}
%\[
%	P_{A_2 B_2 C_2}(111) \leq P_{A_1 B_1 C_1}(000) + P_{A_1 B_2 C_2}(111) + P_{A_2 B_1 C_2}(111) + P_{A_2 B_2 C_1}(111).
%\]
\cref{eq:F3rawprobform} is equivalent to \cref{eq:FritzF3raw}; therefore applying it the inflation depicted in~\cref{fig:Tri222} recovers \cref{eq:FritzF3}.

Inequalities resulting from hypergraph transversals are generally weaker than those that result from completely solving the marginal problem. Nevertheless, many Bell inequalities are of this form---such as the CHSH inequality which follows in this way from Hardy's original implication. So it seems that this method is still sufficiently powerful to generate plenty of interesting inequalities. At the same time, it should be significantly easier to perform in practice than the full-fledged linear (let alone nonlinear) quantifier elimination, even if one does it for every possible antecedent.

In conclusion, linear quantifier elimination is the preferable tool for deriving inequalities whenever it is computationally tractable; but whenever it is not, then enumerating hypergraph transversals presents a good alternative. %: \cref{eq:trisimplestunmapped} here corresponds to Eq. (2-4) in~\cite{Pitowsky1989}%, and \cref{eq:bellcondeq} here corresponds to Eq. (30) in Ref. \cite{Ghirardi08}

%One can also come up with a \emph{satisfiability} version of a marginal problem at the possibilistic level. Possibilistic satisfiability is the following: for every joint marginal outcome with positive probability, one needs to find an assignment of values to all variables which extends the given marginal outcomes and such that the restriction to every context also has positive probability. This is very easy to do: one can simply enumerate all deterministic assignments of values to all observables, discard all those that have a marginal with zero probability, and then check whether the remaining assignments are enough to generate all the joint marginal outcomes with positive probability.

% We note that the connection between classical propositional logic and linear inequalities has been used previously in the task of causal inference. Noteworthy examples of works deriving causal infeasibility criteria via classical logic are \citet{Pitowsky1989} and \citet{Ghirardi08}, see also Refs. \cite{pitowsky_boole_1994,kellerer_marginal_1964,leggett_garg_1985,araujo_cycle_2013}. Novel to this work, however, is the use of the hypergraph transversals problem to formally enumerate relevant logical implications. 

\section{Bell scenarios and inflation}
\label{sec:Bellscenarios}


To further illustrate the power of our inflation DAG approach, we now demonstrate how to recover all Bell inequalities~\cite{Brunner2013Bell,bell1966lhvm,CHSHOriginal} via our method. To keep things simple we only discuss the case of a bipartite Bell scenario with two values for both ``settings'' and ``outcome'' variables here, but the case of more parties and/or more values per variable is totally analogous.
%It is critical that our method should be able to derive these seminal criteria, as Bell inequalities have been a foundational component of quantum information theory for the last half century \cite{scarani2012device,BancalDIApproach}.

The causal structure associated to the Bell \cite{bell1964einstein,Brunner2013Bell,bell1966lhvm,CHSHOriginal} experiment [\citealp{pusey2014gdag}~(Fig.~E\#2), \citealp{WoodSpekkens}~(Fig.~19), \citealp{chaves2014novel}~(Fig.~1), \citealp{BeyondBellII}~(Fig.~1), \citealp{wolfe2015nonconvexity}~(Fig.~2b), \citealp{steeg2011relaxation}~(Fig.~2)] is depicted here in \cref{fig:NewBellDAG1}. The observable variables are $A,B,X,Y$, and $\Lambda$ is the latent common cause of $A$ and $B$. In a Bell scenario, one traditionally works with the conditional distribution $P_{AB|XY}$, to be understood as an array of distributions indexed by the possible values of $X$ and $Y$, instead of with the original distribution $P_{ABXY}$, which is what we do.

In the Bell scenario DAG, the maximal pre-injectable sets are
\begin{align}\begin{split}
	\label{eq:bellcontexts}
&\brackets{A_1 B_1 X_1 X_2 Y_1 Y_2} \\
&\brackets{A_1 B_2 X_1 X_2 Y_2 Y_2} \\
&\brackets{A_2 B_1 X_1 X_2 Y_2 Y_2} \\
&\brackets{A_2 B_2 X_1 X_2 Y_2 Y_2} ,
\end{split}\end{align}
where notably every maximal pre-injectable set contains all ``settings'' variables $X_1$ to $Y_2$. The marginal distributions on these pre-injectable sets are then specified by the original observable distribution via
\begin{align}\begin{split}&\forall{a b x_1 x_2 y_1 y_2}:\; \begin{cases}
	P_{A_1 B_1 X_1 X_2 Y_1 Y_2}(a b x_1 x_2 y_1 y_2)  = P_{A B X Y}(a b x_1 y_1) P_X(x_2) P_Y(y_2), \\
	P_{A_1 B_2 X_1 X_2 Y_1 Y_2}(a b x_1 x_2 y_1 y_2)  = P_{A B X Y}(a b x_1 y_2) P_X(x_2) P_Y(y_1), \\
	P_{A_2 B_1 X_1 X_2 Y_1 Y_2}(a b x_1 x_2 y_1 y_2)  = P_{A B X Y}(a b x_2 y_1) P_X(x_1) P_Y(y_2), \\
	P_{A_2 B_2 X_1 X_2 Y_1 Y_2}(a b x_1 x_2 y_1 y_2)  = P_{A B X Y}(a b x_2 y_2) P_X(x_1) P_Y(y_1), \\
\hspace{2.5pc}	P_{X_1 X_2 Y_1 Y_2}(x_1 x_2 y_1 y_2)  = P_X(x_1) P_X(x_2) P_Y(y_1) P_Y(y_2).
\end{cases}\end{split}\end{align}
%where it should be understood implicitly that the inequalities hold for all values of $\{a b x_1 x_2 y_1 y_2\}$.
%The last equations follows from each of the others by making use of the Markov condition $P_{XY} = P_X P_Y$ for the original distribution, it nevertheless
By dividing each of the first four equations by the fifth, we obtain
\begin{align}\begin{split}
	\label{eq:bellfactor}
	\forall{a b x_1 x_2 y_1 y_2}:\; \begin{cases}
	P_{A_1 B_1 | X_1 X_2 Y_1 Y_2}(a b | x_1 x_2 y_1 y_2)  = P_{A B | X Y}(a b | x_1 y_1), \\
	P_{A_1 B_2 | X_1 X_2 Y_1 Y_2}(a b | x_1 x_2 y_1 y_2)  = P_{A B | X Y}(a b | x_1 y_2), \\
	P_{A_2 B_1 | X_1 X_2 Y_1 Y_2}(a b | x_1 x_2 y_1 y_2)  = P_{A B | X Y}(a b | x_2 y_1), \\
	P_{A_2 B_2 | X_1 X_2 Y_1 Y_2}(a b | x_1 x_2 y_1 y_2)  = P_{A B | X Y}(a b | x_2 y_2).
\end{cases}\end{split}\end{align}
The existence of a joint distribution over all six variables---i.e.~the existence of a solution to the marginal problem---implies in particular
%If we then impose marginal compatibility according the the marginal problem we find that the (minimal!) consequence of the inflation hypothesis is 
\begin{align}
	\forall{a b x_1 x_2 y_1 y_2}: \quad P_{A_1 B_1 | X_1 X_2 Y_1 Y_2}(a b | x_1 x_2 y_1 y_2)  =  \sum\nolimits_{a',b'} P_{A_1 A_2 B_1 B_2 X_1 X_2 Y_1 Y_2}(a a' b b'|x_1 x_2 y_1 y_2),
\end{align}
and similarly for the other three conditional distributions under consideration. For consistency with the causal hypothesis, therefore, the original distribution must satisfy in particular
\begin{align}\begin{split}\label{eq:finalBellstep}\forall{a b}:\; \begin{cases}
	P_{A B | X Y}(a b | 0 0)  =  \sum\nolimits_{a',b'} P_{A_1 A_2 B_1 B_2 X_1 X_2 Y_1 Y_2}(a a' b b'|0101) \\
	P_{A B | X Y}(a b | 1 0)  =  \sum\nolimits_{a',b'} P_{A_1 A_2 B_1 B_2 X_1 X_2 Y_1 Y_2}(a' a b b'|0101) \\
	P_{A B | X Y}(a b | 0 1)  =  \sum\nolimits_{a',b'} P_{A_1 A_2 B_1 B_2 X_1 X_2 Y_1 Y_2}(a a' b' b|0101) \\
	P_{A B | X Y}(a b | 1 1)  =  \sum\nolimits_{a',b'} P_{A_1 A_2 B_1 B_2 X_1 X_2 Y_1 Y_2}(a' a b' b|0101)
\end{cases}\end{split}\end{align}
The possibility to write the conditional probabilities in the Bell scenario in this form is equivalent to the existence of a latent variable model, as noted in Fine's Theorem~\cite{FineTheorem}. Thus, if an inflation model exists with the required marginals, then a latent variable model of the original distribution exists as well (and conversely, trivially). Hence the inflation DAG of~\cref{fig:BellDagCopy1} provides necessary and sufficient conditions for the consistency of the original observed distribution with the Bell scenario causal structure.

%The last equations turns out to be rather useful to write down explicitly: putting $x_1 = y_1 = 0$ and $x_2 = y_2 = 1$ and dividing the first equation by the last one results in
%\begin{align*}
%	P_{A B X Y}(a b | 0 0)  =  \sum_{a',b'} P_{A_1 A_2 B_1 B_2 X_1 X_2 Y_1 Y_2}(aa'bb'|0101).
%\end{align*}
%Similarly, $P_{A B X Y}(a b | 0 1)$, $P_{A B X Y}(a b | 1 0)$ and $P_{A B X Y}(a b | 1 1)$ can also be written as marginals of a conditional distribution. 
%By Fine's Theorem~\cite{FineTheorem}, this implies the existence of a hidden-variable model. Conversely, if a hidden-variable model exists, then the existence of the inflation model implies the existence of a solution to the marginal problem.

% In conclusion, we therefore find in the case of the inflation DAG~\cref{fig:BellDagCopy1}, the inflation method yields necessary and sufficient infeasibility criteria for the Bell causal structure, i.e. the Bell/CHSH inequalities, just by requiring marginal compatibility of the pre-injectable sets.
% More generally, we can use Fine's theorem to show that applying the marginal problem to suitable inflation DAGs can reproduce \emph{all} Bell inequalities, for any standard Bell scenario, no matter how many parties or settings or possible outcomes. 

Moreover, it is possible to describe the marginal polytope over the pre-injectable sets of~\cref{eq:bellcontexts}, due to the fact that the ``settings'' variables $X_1$ to $Y_4$ occur in all four contexts. This description is easier to state for the marginal \emph{cone}, by which we mean the convex cone spanned by the marginal polytope, i.e.~the convex cone consisting of all nonnegative linear combinations of deterministic assignments of values, or equivalently the convex cone of all measures on the set of joint outcomes. This cone lives in $\oplus_{i=1}^4 \mathbb{R}^{2^6} = \oplus_{i=1}^4 (\mathbb{R}^2)^{\otimes 6}$, where each tensor factor has basis vectors corresponding to the two possible outcomes of each variable, and the direct summands enumerate the four contexts. Now the marginal cone is precisely the set of all nonnegative linear combinations of the points
\begin{align*}
	(e_{A_1} & \otimes e_{B_1} \otimes e_{X_1} \otimes e_{X_2} \otimes e_{Y_1} \otimes e_{Y_2}) \\
	\oplus\: (e_{A_1} & \otimes e_{B_2} \otimes e_{X_1} \otimes e_{X_2} \otimes e_{Y_1} \otimes e_{Y_2}) \\
	\oplus\: (e_{A_2} & \otimes e_{B_1} \otimes e_{X_1} \otimes e_{X_2} \otimes e_{Y_1} \otimes e_{Y_2}) \\
	\oplus\: (e_{A_2} & \otimes e_{B_2} \otimes e_{X_1} \otimes e_{X_2} \otimes e_{Y_1} \otimes e_{Y_2}),
\end{align*}
where all six variables range over their deterministic outcomes. Since the last four tensor factors occur in every direct summand in exactly the same way, the resulting marginal cone is linearly isomorphic to the cone generated by all vectors of the form
\[
	\left[ (e_{A_1} \otimes e_{B_1}) \oplus (e_{A_1} \otimes e_{B_2}) \oplus (e_{A_2} \otimes e_{B_1}) \oplus (e_{A_2} \otimes e_{B_2})\right] \otimes \left[ e_{X_1} \otimes e_{X_2} \otimes e_{Y_1} \otimes e_{Y_2}\right]
\]
in $\mathbb{R}^{2^2}\otimes \mathbb{R}^{2^4}$. Now since the first four variables in the first tensor factor vary completely independently of the latter four variables in the second tensor factor, the resulting cone will be precisely the tensor product~\cite{namioka_tensor_1969} of two cones: first, the cone generated by all vectors of the form
\[
	(e_{A_1} \otimes e_{B_1}) \oplus (e_{A_1} \otimes e_{B_2}) \oplus (e_{A_2} \otimes e_{B_1}) \oplus (e_{A_2} \otimes e_{B_2}),
\]
and second the one spanned by all $e_{X_1} \otimes e_{X_2} \otimes e_{Y_1} \otimes e_{Y_2}$. While the latter cone is simply the standard positive cone of $\mathbb{R}^8$, the former cone is the cone generated by the ``local polytope'' or ``Bell polytope'' that is traditionally used in the context of Bell scenarios~\cite[Sec.~II.B]{Brunner2013Bell}. Standard results on tensor products of cones and polytopes~\cite{bogart_hom_2013} therefore imply that our marginal polytope is the tensor product of the Bell polytope, corresponding to the $A_1$ to $B_2$ part, with a simplex corresponding to the $X_1$ to $Y_2$ ``settings'' part. This implies that the facets of our marginal polytope are precisely the pairs consisting of a facet of the Bell polytope and a facet of the simplex. For example, in this way we obtain one version of the CHSH inequality as a facet of the marginal polytope,
\[
	\sum_{a,b,x,y} (-1)^{a + b + xy} P_{A_x B_y X_1 X_2 Y_1 Y_2}(a b 0 1 0 1) \leq 2 P_{X_1 X_2 Y_1 Y_2}(0101).
\]
Upon using~\cref{eq:bellfactor}, this becomes
\begin{align*}
	\sum_{a,b} (-1)^{a + b} \big( & P_{A B X Y}(ab00)P_X(1)P_Y(1) + P_{A B X Y}(ab01)P_X(1)P_Y(0) \\
	& + P_{A B X Y}(ab10)P_X(0)P_Y(1) - P_{A B X Y}(ab11)P_X(0)P_Y(0) \big) \leq P_X(0)P_X(1)P_Y(0)P_Y(1),
\end{align*}
so that dividing by the right-hand side results in essentially the conventional form of the CHSH inequality,
\[
	\sum_{a,b} (-1)^{a + b} \left( P_{AB|XY}(ab|00) + P_{AB|XY}(ab|01) + P_{AB|XY}(ab|10) - P_{AB|XY}(ab|11) \right) \leq 2.
\]
In conclusion, the inflation DAG technique is powerful enough to get a precise characterization of all distributions consistent with the Bell causal structure, and our technique for generating polynomial inequalities while solving the marginal problem recovers all Bell inequalities.

\begin{comment}
Analysis of the inflated DAG in \cref{fig:BellDagCopy1} shows that  
\begin{align}\label{eq:bellypolymapped}
 \p{a b x y}\p{\n{x}} \p{\n{y}}
&\leq
 \p{a \n{b} x \n{y}}\p{\n{x}}\p{y} +  \p{\n{a} b \n{x} y}\p{x} \p{\n{y}}+\p{a b \n{x} \n{y}}\p{x} \p{y}
\\
\shortintertext{by virtue of the unmapped (but factored) inequality}
\label{eq:bellypolyunmapped}
 \p{a_1 b_1 x_1 y_1}\p{\n{x}_2} \p{\n{y}_2}
&\leq
 \p{a_1 \n{b}_2 x_1 \n{y}_2}\p{\n{x}_2} \p{y_1} +\p{\n{a}_2 b_1 \n{x}_2 y_1}\p{x_1}\p{\n{y}_2}+  \p{a_2 b_2 \n{x}_2 \n{y}_2}\p{x_1} \p{y_1}
\end{align}
where \emph{every} probability appearing in \cref{eq:bellypolyunmapped} maps to the original scenario, hence yielding \cref{eq:bellypolymapped}. To derive the usual Bell inequalities from \cref{eq:bellypolymapped} we switch to conditional probabilities via ${\p{a b x y}\to\p{a b | x y}\p{x y}=\p{a b | x y}\p{x}\p{y}}$ which, after dividing both sides of \cref{eq:bellypolymapped} by $\p{x}\p{\n{x}}\p{y}\p{\n{y}}$, yields
\begin{align}\label{eq:chwithneg}
&\p{a b | x y}
\leq
{\p{a \n{b} | x \n{y}} + \p{\n{a} b | \n{x} y}+\p{a b | \n{x} \n{y}}}
\\
\hspace{-\mathindent}\text{or, equivalently,}\quad
\label{eq:chwithoutneg}
&{\p{a b | x y}+\p{a b | x \n{y}} + \p{a b | \n{x} y}}
\leq
{\p{a|\n{x}}+\p{b|\n{y}} + \p{a b | \n{x} \n{y}}}
\end{align}
which is precisely the Clauser-Horne (CH) inequality \cite{CHInequality} for the Bell scenario. Note that to obtain \cref{eq:chwithoutneg} from \cref{eq:chwithneg} we implicitly made use of the no-signalling assumptions, namely $\p{a|x y}=\p{a|x}$ and $\p{b| x y}=\p{b | y}$. The CH inequality is the \emph{unique} Bell inequality (up to permutations) for the Bell scenario if $\brackets{A,B,X,Y}$ are all binary, and hence the CH inequality is a necessary and sufficient criterion to ascertain if correlations are compatible with that Bell scenario variant.

The causal structure of a Bell scenario can also be formulated directly in terms of conditional random variables. For example, the conditional-structure interpretation of \cref{fig:BellDagCopy1} is \cref{fig:BellConditionalDAG}. 

\begin{figure}[t]
\centering
\begin{minipage}[t]{0.45\linewidth}
\centering
\includegraphics[scale=1]{BellDagConditionForm.pdf}
\caption{The causal structure of the Bell scenario expressed in a form which makes use of conditional random variables.}\label{fig:BellConditionalDAG}
\end{minipage}
%\hfill
%\begin{minipage}[t]{0.45\linewidth}
%\centering
%\includegraphics[scale=1]{BellDagCopy.pdf}
%\caption{An inflation DAG of the Bell scenario, where both local settings variables have been duplicated.}\label{fig:BellDagCopy}
%\end{minipage}
\end{figure}

The Bell inequalities are then self-evident from \cref{fig:BellConditionalDAG} without the need for an inflation DAG. The conditional-structure formulation innately implies its own inaccessible gedankenprobabilities, such as $\{\p{a|x , a|\n{x}}$, $\p{a|x , \n{a}|\n{x}},...\}$ etc. By eliminating these gedankenprobabilities from the set of inequalities generated by $0\leq \pdf{A|x , A|\n{x} , B|y , B|\n{y}}$ we obtain
\begin{align}\label{eq:bellcondeq}
0
\leq
{\p{a|\n{x}} + \p{b|\n{y}} + \p{a|x , b|y} -\p{a|x ,b|\n{y}} -\p{a|\n{x} , b|y} -\p{a|\n{x} , b|\n{y}}}\,,
\end{align}
for example. It should be clear that \cref{eq:bellcondeq} is equivalent to \cref{eq:chwithoutneg}.
\end{comment}

%We can perform quantifier elimination on the set of (polynomial) inequalities given by $0\leq \pdf{A|x , A|\n{x} , B|y , B|\n{y}}$. The use of a conditional-structure DAG innately implies certain inaccessible gedankenprobabilities, such as $\{\p{a|x , a|\n{x}}$, $\p{a|x , \n{a}|\n{x}},...\}$ etc. 
%\purp{Rob says kill next two paragraphs?}

%Conditional-structure gedankenprobabilities are somewhat different from the inflation DAG kind, in that they reference multiple counterfactual events, such as ``What is the probability that Alice would choose to visit the museum \emph{IF} (given that) it's a rainy day in Maryland \emph{AND} that Alice would choose to go the beach \emph{IF} (given that) it's sunny in Maryland?". By contrast, unconditional gedankenprobabilities which live on an inflation DAG reference multiple heterofactual (for lack of a better word) events, such as ``What is the probability that Alice-copy-\#1 chooses to visit the museum \emph{AND} that it's raining in Maryland-copy-\#1 \emph{AND} that Alice-copy-\#2 chooses to go the beach \emph{AND} that it's sunny in Maryland-copy-\#2?". 

%Joint counterfactual probabilities are experimentally inaccessible, just the same as joint heterofactual probabilities are. Suppose one could establish both Alice's propensity for going to the museum when it rains and her propensity for going to the beach when it's sunny. Even so, neither the joint counterfactual probability nor the joint heterofactual probability can be established from that limited data. For example, the value of the hidden variable ${\Lambda\cramp{=}\lambda}$ may influence Alice's willingness to get out of bed at all, or determine if she is on-call as a volunteer EMT on a particular day, or $\lambda$ might encode if Alice is travelling out-of-state. If we could measure $\p{a|x , \n{a}|\n{x}},...\}$ we might learn that Alice's likelihood of visiting the museum if it rains in Maryland is highly correlated with her likelihood of visiting the beach when it's sunny in Maryland. Or we might learn that those two counterfactual probabilities are relatively statistically independent. The ``hidden-ness" of the classical variable corresponding to the latent node shields the gedankenprobabilities from being determined. 


\section{Quantum Causal Inference and the No-Broadcasting Theorem}\label{sec:classicallity}

In the causal inference problems with latent nodes that we have considered so far, the latent nodes correspond to unobserved random variables. This describes things that come up in \emph{classical} physics (and things outside of physics). In \emph{quantum} physics, however, the latent nodes may instead carry \emph{quantum systems}. Whenever this is allowed, we say that the DAG represents a \tblue{quantum causal structure}. Some quantum causal structures are famously capable of generating distributions over the observable variables that would not be possible classically.

The set of quantumly realizable distributions is superficially quite similar to the classical subset \cite{pusey2014gdag,fritz2012bell}. For example, classical and quantum distributions alike respect all conditional independence relations implied by the common underlying causal structure \cite{pusey2014gdag}. It is an interesting problem to find quantum distributions that are not realizable classically, or to show that there are no such distributions on a given DAG.

However, this is by no means an easy task. For example, recent work has found that quantum causal structure also implies many of the entropic inequalities that hold classically~\cite{pusey2014gdag,Chaves2015infoquantum,ChavesNoSignalling}. To date, no quantum distribution has been found to violate a Shannon-type entropic inequality on observable variables derived from the Markov conditions on all nodes \cite{chaves2012entropic,fritz2012bell}. Fine-graining the scenario by conditioning on root variables (``settings'') leads to a different kind of entropic inequality, and these have proven somewhat quantum-sensitive \cite{braunstein1988entropic,SchumacherInequality,chaves2014novel}. Such inequalities are still limited, however, in that they only apply in the presence of observable root nodes\footnote{Rafael Chaves and E.W.~are exploring the potential of entropic analysis based on conditioning on non-root observable nodes. This generalizes the method of entropic inequalities, and might be capable of providing much stronger entropic witnesses.}, and they still fail to witness certain distributions as classically impossible~\cite{chaves2014novel,fritz2012bell}.

We hope that polynomial inequalities derived from broadcasting inflation DAGs will provide an additional tool for witnessing certain quantum distributions as non-classical. For example due to the results of~\cref{sec:Bellscenarios}, it seems conceivable that these inequalities will be much stronger and provide much tighter constraints than entropic inequalities.

It is worth pondering how it is possible that some of the inequalities that can be derived via inflation---such as Bell inequalities---have quantum violations, i.e.~why one cannot expect them to be valid for all quantum distributions as well. The reason for this is that duplicating an outgoing edge in a DAG during inflation amounts to \tblue{broadcasting} the value of the random variable. For example while the information about $X$ in~\cref{fig:TriMainDAG} was ``sent'' to $A$ and $C$, the information about $X_1$ is sent to $A_1$ \emph{and} $A_2$ and $C$ in the inflation \cref{fig:simpleinflation}. Since quantum theory satisfies a no-broadcasting theorem~\cite{NoCloningQuantum1996,NoCloningGeneral2006}, one cannot expect such broadcasting to be possible quantumly. More generally, there is an analogous no-broadcasting theorem in the regime of epistemically restricted general probabilistic theories (GPTs) \cite{SpekkensToyTheory,NoCloningGeneral2006,Barnum2012GPT,Janotta2014GPT}, so that the same statement applies in many theories other than quantum theory. As a consequence, a quantum or general probabilistic causal model on the original DAG does generally not inflate to a ``quantum inflation model'' or ``general probabilistic inflation model'' on the inflation DAG. 

Some inflations, such as the one of~\cref{fig:simplestinflation}, do not require such broadcasting. By remove $A_1$ from the broadcasting inflation of \cref{fig:simpleinflation} we obtain the non-broadcasting inflation of \cref{fig:simplestinflation}. In \cref{fig:simplestinflation} the channel from $X$ to $A$ is merely \emph{redirected} from it original configuration in \cref{fig:TriMainDAG}; there is no broadcasting of information required.

\begin{definition}
	$G'\in\SmallNamedFunction{Inflations}{G}$ is \tblue{non-broadcasting} if every latent node in $G'$ has at most one copy of each $A\in\SmallNamedFunction{Nodes}{G}$ among its children.
\end{definition}

It follows that every quantum causal model can be inflated to a non-broadcasting DAG, so that one obtains a quantum and general probabilistic analogue of Lemma~\ref{mainlemma} in the non-broadcasting case. Constraints derived from non-broadcasting inflations are therefore valid also for quantum and even general probabilistic distributions. In the specific case of the entropic monogamy inequality for the Triangle scenario, i.e. \cref{eq:monogomyofcorrelations} here, this was originally noticed in Ref.~\cite{pusey2014gdag}. Another example is \cref{eq:polymonogamy}, which was derived from the non-broadcasting inflation of \cref{fig:TriDagSubA2B1C1}. \cref{eq:polymonogamy} too, therefore, is a necessary criterion for compatibility with the Triangle scenario even when the latent nodes are allowed to carry quantum or general probabilistic systems.
% This confirms our numerical computations, which indicated that~\eqref{eq:polymonogamy} does not have any quantum violations. The same is true for monogamy of correlations, per \cref{eq:monogomyofcorrelations}.
Since the perfect-correlation distribution considered in~\cref{eq:ghzdistribution1} violates both of these inequalities, it evidently cannot be generated within the Triangle scenario even with quantum or general probabilistic states on the hidden nodes. This was also pointed out in Ref.~\cite{pusey2014gdag}.

%\begin{align}\label{eq:nonbroadcastinginflationDAG}
%G'\in\SmallNamedFunction{NonBroadcastingInflations}{G} \quad\text{ iff }\quad \forall{\text{latent }A_i\in G'}\; \Ch[G']{A_i} \text{ is an irredundant set.}
%\end{align}

On the other hand, by intentionally using broadcasting in an inflation DAG, we can specifically try to witness certain quantum or general probabilistic distributions as non-classical. This is exactly what happens in Bell's theorem.

%  We also find it useful to define the notion of a non-broadcasting subset of nodes within some larger broadcasting inflation DAG.
% Let's define any pair of redundant nodes which share a latent parent to be a \tblue{fundamental broadcasting pair}. An inflation DAG is non-broadcasting if it does not contain any fundamental broadcasting pairs. Similarly, a set of nodes $\bm{U}$ is a \tblue{non-broadcasting set} iff $\An[G']{\bm{U}}$ is free of any fundamental broadcasting pairs.

%A set of nodes $\bm{U}$ is a \tblue{non-broadcasting set} iff $\ansubgraph[G']{\bm{U}}$ is a non-broadcasting inflation DAG. Any inference about the original DAG which can be made by referencing exclusively to non-broadcasting sets hold in both the classical and quantum paradigms. Broadcasting inflation DAGs are therefore especially useful for deriving criteria which distinguish quantum and classical probability distributions, but we anticipate them to be valuable for broader causal inference tasks as well.

% It is worth emphasizing that broadcasting the values of hidden variables %which are predicated on multiple counterfactual or heterofactual events 
%are strictly classical constructs. If the latent node in the Bell scenario in \cref{fig:NewBellDAG1} is allowed to be a quantum resource $\mathcal{H}^{d_A\otimes d_B}$, for example, then broadcasting gedankendistributions such as $\pdf{A|x , A|\n{x},...}$ or $\pdf{A_1,A_2,...}$ are \tblue{physically prohibited} if the quantum state is suitably entangled.

% More precisely, quantum states are governed by a no-broadcasting theorem \cite{NoCloningQuantum1996,NoCloningGeneral2006}: If half the state is sent to Alice and she performs some measurement on it, she fundamentally perturbs the state by measuring it. Post-measurement, that half of the state cannot be ``re-sent" to Alice, that she might re-measure it using a different measurement setting. As a consequence of the no-broadcasting theorem, in the inflation DAG picture a quantum state which was initially available to a single party cannot be distributed both to Alice-copy-\#1 and Alice-copy-\#2 in the way a classical hidden variable could be. 

% This means that considerations on inflation DAGs cannot be used to derive quantum causal infeasibility criteria whenever a gedankenprobability presupposes the ability to broadcast a latent node's system. Broadcasting and non-broadcasting sets of variables are distinguished per \cref{eq:nonbroadcastinginflationDAG}.

%When analyzing GPT causal structures one may not assume that a joint probability distribution over observable variables is universally nonnegative if the set of observable variables has simultaneous meaning only under broadcasting of latent variables. This is in direct contrast to \cref{step:generateineqs} in \cref{sec:mainalgorithm}. 

%Not every inflation requires broadcasting, however, and hence not every gedankenprobability is physically prohibited by quantum theory. 
%An inflation DAG is a \tblue{non-broadcasting inflation} whenever the children of every individual node in the inflation DAG $G'$ map injectively to the corresponding children of the mapped node in the original DAG $G$, i.e. $dmap\parenths*{\Ch[G']{X}}\subseteq\Ch[G]{dmap\parenths*{X}}$ for all $X\in\nodes{G'}$. 
% \cref{fig:TriDagSubA2B1C1} is an example of a non-broadcasting inflation.

Even when using broadcasting inflation DAGs, it may still be possible to derive inequalities valid for quantum distributions if one appropriately the nonnegativity inequalities in the marginal problem, e.g. such as \cref{eq:nonnegativity}. The modification would replace demanding nonnegativity of the full joint distribution with instead demanding the nonnegativity of only quantum-physically-meaningful marginal probability distributions. 

Even when using broadcasting inflation DAGs, it may still be possible to derive inequalities valid for quantum distributions if one appropriately modifies \cref{sec:ineqs} to generate a different initial set of nonnegativity inequalities. This new set should capture the nonnegativity of only quantum-physically-meaningful marginal probability distributions. Indeed, a quantum causal model on the original DAG can potentially be inflated to a quantum inflation model on the inflation DAG in terms of the logical broadcasting maps of \citet{Coecke2011}. From this perspective, a broadcasting inflation DAG is an abstract logical concept, as opposed to a feasible physical construct. However, this would result in a joint distribution over all observable variables that may have some negative probabilities, and one cannot expect~\cref{eq:nonnegativity} to hold in general. But one can still try to reformulate the marginal problem so as to refer only to the existence of joint distributions on non-broadcastings sets rather than the existence of a full joint distribution from which the marginal distributions might be recovered. Here, a set $\bm{U}$ of observable nodes is non-broadcasting if $\An{\bm{U}}$ does not two distinct copies of a node which have a parent in common.

An analysis along these lines has already been carried out successfully by \citet{Chaves2015infoquantum} in the derivation of entropic inequalities that are valid for all quantum distributions. Although \citet{Chaves2015infoquantum} do not invoke inflation DAGs, they do seem to employ a similar type of structure to model the conditioning of a variable on a ``setting'' variable, and this also gives rise to non-broadcasting sets. \citet{Chaves2015infoquantum} take pains to avoid including full joint probability distributions in any of their initial entropic inequalities, precisely as we would want to do in constructing our initial probability inequalities, and they successfully derive quantumly valid entropic inequalities. But so far, no inequalities polynomial in the probabilities have been derived using this method.

% Our current inflation DAG method can be employed to derive causal infeasibility criteria for general causal structures, thus generalizing Bell inequalities somewhat. From a quantum foundations perspective, however, generalizing Tsirelson inequalities \cite{Tsirelson1980,Brunner2013Bell}---the ultimate constraints on what quantum theory makes possible---is even more desirable. 

A tight set of inequalities characterizing quantum distributions would provide the ultimate constraints on what quantum theory allows. Deriving additional inequalities that hold for quantum distributions is therefore a priority for future research.








\section{Conclusions}
%\purp{Not yet written. To discuss: 
%\begin{compactitem}
%\item Relations to other work.
%\item Why linear elimination is weaker than nonlinear. 
%\item Why a single inflation DAG yields necessary but not sufficient inequalities. 
%\item Why not maximal degree yet known on inequalities. 
%\item Extent to which quantum latent variables violate these inequalities as desideratum for future research.
%\end{compactitem}
%}

Our main contribution is a new way of deriving causal incompatibility witnesses, namely the inflation DAG approach. An inflation DAG naturally carries inflation models, and the existence of an inflation model implies inequalities which constrain the set of distributions on observable nodes compatible with the original causal structure. Polynomial inequalities can be obtained through \emph{linear} inequalities which are necessary conditions for a collection of given marginal distributions to arise from a joint distribution (marginal problem). For deriving such inequalities in turn, we have considered the methods of computing all facets of the marginal polytope via facet enumeration, and deriving looser constraints more efficiently by enumerating hypergraph transversals.

The resulting polynomial inequalities are necessary conditions on a joint distribution to be explained by the causal structure. We currently do not know to what extent they can also be considered sufficient, and there is somewhat conflicting evidence: as we have seen, the inflation DAG approach reproduces all Bell inequalities; but on the other hand, we have not been able to use it to rederive Pearl's instrumental inequality, although the instrumental scenario also contains only one latent node. By excluding the W-type distribution on the Triangle scenario, we have seen that our polynomial inequalities are stronger than entropic inequalities in at least some cases.
% yet just how strong they are is still unclear. A distribution might satisfy all our polynomial inequalities and yet not be realizable from the causal structure. %What would such superficially-feasible distributions look like? Are inflation DAG inequalities ever tight? 
% Our methods yields tight causal infeasibility criteria for Bell scenarios, but those scenarios are exceptional in that the sets of realizable distributions form a convex polytope.

% The most elementary of all causal infeasibility criteria are the conditional independence (CI) relations. Our method explicitly incorporates all marginal independence relations implied by a causal structure. We have found that some CI relations also appear to be implied by our polynomial inequalities. In future research we hope to clarify the process through which CI relations are manifested as properties of the inflation DAG.

A single causal structure has unlimited potential inflations. Selecting a good inflation from which strong polynomial inequalities can be derived is an interesting challenge. To this end, it would be desirable to understand how particular features of the original causal structure are exposed when different nodes in the DAG are duplicated. By isolating which features are exposed in each inflation, we could conceivably quantify the causal inference strength of each inflation. In so doing, we might find that inflated DAGs beyond a certain level of variable duplication need not be considered. The multiplicity beyond which further inflation is irrelevant may be related to the maximum degree of those polynomials which tightly characterize a causal scenario. Presently, however, it is not clear how to upper bound either number, or whether finite upper bounds can even be expected.

Concerning the relation to quantum theory, our method turns the quantum no-broadcasting theorem \cite{NoCloningQuantum1996,NoCloningGeneral2006} on its head by crucially relying on the fact that classical hidden variables \emph{can} be cloned. The possibility of classical cloning motivates the inflation DAG method, and is often critical for deriving strong incompatibility witnesses. We have found that in the case of non-broadcasting inflations, our method also yields causal incompatibility witnesses that constitute necessary constraints even for \emph{quantum} or \emph{general probabilistic} causal scenarios, a common desideratum in recent works \cite{fritz2012bell,pusey2014gdag,Chaves2015infoquantum,ChavesNoSignalling,BeyondBellII}. 

It would be enlightening to understand the extent to which our (classical) polynomial inequalities are violated in quantum theory. A variety of techniques exist for estimating the amount by which a Bell inequality \cite{NPA2008Long,I3322NPA1} is violated in quantum theory, but even finding a quantum violation of one of our \emph{polynomial} inequalities presents a new task for which we currently lack a systematic approach. Nevertheless, we know that there exists a difference between classical and quantum also beyond Bell scenarios~\cite[Theorem~2.16]{fritz2012bell}, and we hope that our polynomial inequalities will perform better in witnessing this difference than entropic inequalities do~\cite{pusey2014gdag,Chaves2015infoquantum}.

%that one inflation DAG is or is not ``stronger" than another? Can we upper bound the maximum dimension of those polynomials which in-principle tightly characterize a given causal structure? A ``yes" answer to any of the aformentioned questions means that that perhaps inflation DAGs beyond a certain level of complexity need not be considered. These remain open questions, however.

%\end{spacing}





\begin{acknowledgments}
%\bigskip\noindent\textbf{Acknowledgments}
T.F.~would like to thank Guido Mont\'ufar for discussion and references. Research at Perimeter Institute is supported by the Government of Canada through Industry Canada and by the Province of Ontario through the Ministry of Economic Development and Innovation.
\end{acknowledgments}


\onecolumngrid
\newpage
\appendix
\numberwithin{equation}{section}
%\renewcommand{\theequation}{A-\arabic{equation}}
\setcounter{equation}{0}
\renewcommand\section{\clearpage\stdsection}






\section{Algorithms for Solving the Marginal Problem}\label{sec:projalgorithms}

By solving the marginal problem, what we mean is to determine all the facets of the marginal polytope for a given marginal scenario. Since the vertices of this polytope are precisely the deterministic assignments of values to all variables, which are easy to enumerate, solving the marginal problem is an instance of a \tblue{facet enumeration problem}: given the vertices of a convex polytope, determine its facets. This is a well-studied problem in combinatorial optimization for which a variety of algorithms are available~\cite{avis_convexhull_2015}. 

A generic facet enumeration problem takes a matrix of vertices $V\in\mathbb{R}^{n\times d}$, where each row is a vertex, and asks what is the set of points $x\in\mathbb{R}^d$ that can be written as a convex combination of the vertices using weights $w\in\mathbb{R}^n$ that are nonnegative and normalized,
\begin{align}
	\label{projsimplex}
	\left\{\: x\in\mathbb{R}^d \quad\bigg|\quad \exists w\in\mathbb{R}^n:\; x = w V ,\;\; w\geq 0,\;\; {{\sum_i}{w_i}}=1 \:\right\}.
\end{align}
% It should be evident from Eqs. (\ref{eq:nonnegativity}-\ref{eq:marginalequalities222}) that the marginal problem is precisely of this form. The individual probabilities of the joint outcome correspond to the weights in the convex hull problem. Recasting the marginal problem as a convex hull problem means that optimized convex hull algorithms can be used to directly solve the marginal problem. Fine's Theorem~\cite{FineTheorem} also follows along these lines. Fine's theorem states that the existence of a joint distribution is equivalent to having the observables marginal distribution lie inside the convex hull of all deterministic joint distributions. \purp{Tobias, can you say the previous sentence better perhaps?}
% Geometrically, linear quantifier elimination is equivalent to projecting a high-dimensional polytope in halfspace representation (inequalities and equalities) into a lower-dimensional quotient space.
% Polytope projection is a well-understood problem in combinatorial optimization,
The oldest-known method for facet enumeration relies on \tblue{linear quantifier elimination} in the form of Fourier-Motzkin (FM) elimination~\cite{fordan1999projection,DantzigEaves}. This refers to the fact that one starts with the system $x=wV$, $w\geq 0$ and $\sum_i w_i = 1$, which is the half-space representation of a convex polytope (a simplex), and then one needs to project onto $x$-space by \emph{eliminating} the variables $w$ to which the existential \emph{quantifier} $\exists w$ refers. The Fourier-Motzkin algorithm is a particular method for performing this quantifier elimination one variable at a time; when applied to~\cref{projsimplex}, it is equivalent to the \emph{double description method}~\cite{motzkin_double_1953,Fukuda1996}. Linear quantifier elimination routines are available in many software tools\footnote{For example \textit{MATLAB$^{_{\textit{\tiny\texttrademark}}}$}'s \href[pdfnewwindow]{http://people.ee.ethz.ch/~mpt/2/docs/refguide/mpt/@polytope/projection.html}{\texttt{MPT2}}/\href[pdfnewwindow]{http://ellipsoids.googlecode.com/svn-history/r2740/branches/issue_119_vrozova/tbxmanager/toolboxes/mpt/3.0.14/all/mpt3-3_0_14/mpt/modules/geometry/sets/@Polyhedron/projection.m}{\texttt{MPT3}}, \textit{Maxima}'s \href[pdfnewwindow]{http://maxima.sourceforge.net/docs/manual/de/maxima_75.html}{\texttt{fourier\_elim}}, \textit{lrs}'s \href[pdfnewwindow]{http://cgm.cs.mcgill.ca/~avis/C/lrslib/USERGUIDE.html\#fourier}{\texttt{fourier}}, or \textit{Maple$^{_{\textit{\tiny\texttrademark}}}$}'s (v17+) \href[pdfnewwindow]{http://www.maplesoft.com/support/help/maple/view.aspx?path=RegularChains/SemiAlgebraicSetTools/LinearSolve}{\texttt{LinearSolve}} and \href[pdfnewwindow]{http://www.maplesoft.com/support/help/Maple/view.aspx?path=RegularChains/SemiAlgebraicSetTools/Projection}{\texttt{Projection}}. The efficiency of most of these software tools, however, drops off markedly when the dimension of the final projection is much smaller than the initial space of the inequalities. FM elimination aided by Chernikov rules \cite{Shapot2012,Bastrakov2015} is implemented in \href[pdfnewwindow]{http://sbastrakov.github.io/qskeleton/}{\textit{qskeleton}} \cite{qskeleton}.}. The authors found it convenient to custom-code a linear elimination routine in \textit{Mathematica$^{_{\textit{\tiny\texttrademark}}}$}.
% Other possible algorithms for facet enumeration come in two kinds: first, algorithms that are---like Fourier-Motzkin---also based on linear quantifier elimination, or equivalently on computing the half-space representation of a polytope given in half-space representation; second, algorithms that solve the facet enumeration problem in a different manner. Other projection methods than FM include 

Other algorithms for facet enumeration that are not based on linear quantifier elimination include the following. \emph{Lexicographic reverse search} (LRS)~\cite{Avis2000lrs}, which explores the entire polytope by repeatedly pivoting from one facet to an adjacent one, and is implemented in~\href[pdfnewwindow]{http://cgm.cs.mcgill.ca/~avis/C/lrslib/USERGUIDE.html#Installation\%20Section}{\texttt{lrs}}. Equality Set Projection (ESP)~\cite{jones2004equality,JonesThesis2005} is also based on pivoting from facet to facet, though its implementation is less stable\footnote{ESP  \cite{jones2004equality,JonesThesis2005,Jones2008} is supported by \href[pdfnewwindow]{http://people.ee.ethz.ch/~mpt/2/docs/refguide/mpt/@polytope/projection.html}{\texttt{MPT2}} but not \href[pdfnewwindow]{http://people.ee.ethz.ch/~mpt/3/}{\texttt{MPT3}}, and by the (undocumented) option of \href[pdfnewwindow]{https://github.com/tulip-control/polytope/blob/master/polytope/polytope.py\#L1406}{projection} in the \href[pdfnewwindow]{https://pypi.python.org/pypi/polytope}{\textit{polytope}} (v0.1.1 2015-10-26) python module.}. These algorithms could be interesting to use in practice, since each pivoting step churns out a new facet; by contrast, Fourier-Motzkin type algorithms only generate the entire list of facets at once, after all the quantifiers have been eliminated one by one.

It may also be possible to exploit special features of marginal polytopes in order to facilitate their facet enumeration, such as their high degree of symmetry: permuting the outcomes of each variable maps the polytope to itself, which already generates a sizeable symmetry group, and oftentimes there are additional symmetries given by permuting some of the variables. This simplifies the problem of facet enumeration~\cite{bremner_symmetries_2009,Schurmann2013}, and it may be interesting to apply the dedicated software\footnote{Such as \href[pdfnewwindow]{http://comopt.ifi.uni-heidelberg.de/software/PANDA/}{\texttt{PANDA}}, \href[pdfnewwindow]{http://mathieudutour.altervista.org/Polyhedral/}{\texttt{Polyhedral}}, or \href[pdfnewwindow]{http://www.math.uni-rostock.de/~rehn/software/sympol.html}{\texttt{SymPol}}.} to the facet enumeration problem of marginal polytopes~\cite{Kaibel2010,rehn_tools_2012,panda_2015}.

%\footnote{FM elimination algorithms make intermittent calls to a linear-programming subroutine for eliminating redundant inequalities. The authors found an efficient implementation of this subroutine in \textit{Mathematica$^{_{\textit{\tiny\texttrademark}}}$}, see \cref{sec:redundancy} for further details.}.
%Fourier-Motzkin elimination appreciable suboptimal, however, when the dimension of the final projection is much smaller than the initial space of the inequalities, i.e. when there are many gedankenprobabilities. See Refs. \cite{jones2004equality,JonesThesis2005,Jones2008} and \cref{sec:projalgorithms} for further detail.

% The generic task of polytope projection assumes that the initial polytope is given \emph{only} in halfspace representation. If, however, a \tblue{dual description} of the initial polytope is available, i.e. we are also given its extremal vertices, then the projection problem can be significantly optimized \cite{projectiondual,Avis2000lrs}. Such dual-description algorithms are used, for example, by modern convex hull solvers. The marginal problem can be explicitly recast as a a special convex hull problem, which can be seen as follows.

%The marginal problem is included among such special cases: the extremal vertices of the initial polytope are just the various distinct possible deterministic joint distributions! Indeed, the initial polytope of the marginal problem is a \tblue{probability simplex}, such that every non-negativity inequality is saturated by all one point.

%Here we present an explicit algorithm for polytope projection when a dual description in available. Without loss of generality we assume that the halfspace representation consists of only inequalities. This is generic, as any equality can either be solved-for as a preliminary step (substituting the solution into the system of inequalities) or converted to two inequalities (of the form $\vec{a}.\vec{x}\geq 0$ and $-\vec{a}\vec{x} \geq 0$). In this notation $\vec{x}$ represents a list of variables \emph{appended by 1}, and $\vec{a}$ indicate the coefficients of the variables, appended by some constant.

%A halfspace representation of a polytope is therefore $\brackets{\vec{x}|\hat{A}.\vec{x}\geq 0}$. The matrix element $A_{j,k}$ corresponds to the coefficient of variable $x_k$ in the $j$'th inequality.

%Consider a list of extreme points $\hat{V}$, such that the row $\vec{V_m}$ corresponds to extremal vertex $\#m$ of the polytope, with the vertex coordinates \emph{appended by 1}. Appending 1 to each vertex is useful, as inequality $\vec{A_j}$ is saturated by $\vec{V_m}$ iff $\vec{V_m}.\vec{A_j}=0$. 
%Let's introduce a binary matrix $\hat{Q}$ so that the matrix element $Q_{j,m}$ is $0$ whenever $\vec{V_m}.\vec{A_j}=0$ and $1$ whenever $\vec{V_m}.\vec{A_j}>0$. 

%Elimination of the variable $x_k$ is now performed as follows. Let $\bm{j}^+$ be a list of those $j$ for which $A_{j,k}>0$, let $\bm{j}^-$ be a list of those $j$ for which $A_{j,k}<0$, and let $\bm{j}^0$ be a list of those $j$ for which $A_{j,k}=0$. 

%Let $\hat{A}^+\coloneqq \hat{A}_{\bm{j}^+}$, i.e. the rows of $\hat{A}^+$ are precisely rows $\bm{j}^+$ extracted from $\hat{A}$. In the same manner, construct $\hat{A}^-$, $\hat{A}^0$, $\hat{Q}^+$, $\hat{Q}^-$, and $\hat{Q}^0$. Next, we construct the matrix $\hat{A}^{\pm}$, possessing $|\bm{j}^+|\times|\bm{j}^-|$ rows which we index by $i=|\bm{j}^-|^{j^+}+j^-$. 
%Let row ${\vec{A^{\pm}_i}\coloneqq \left(-A^{-}_{j^-,k}\right)\vec{A^{+}_{j^+}}+\left(A^{-}_{j^-,k}\right)\vec{A^{+}_{j^+}}}$, such that the $k$'th column of $\hat{A}^{\pm}$ is uniformly zero. 
%The minus sign in front of $A^{-}_{j^-,k}$ is important, as by itself $A^{-}_{j^-,k}<0$. Update the matrix of inequalities $\hat{A}$ to be equal to $\hat{A}^{\pm}$ joined with $\hat{A}^0$. At this point the inequities  (rows) of $\hat{A}$ may be redundant, and so we prepare for redundancy elimination as follows.

%We construct the matrix $\hat{Q}^{\pm}$, which like $\hat{A}^{\pm}$ possesses $|\bm{j}^+|\times|\bm{j}^-|$ rows indexed by $i=|\bm{j}^-|^{j^+}+j^-$. Let row ${\vec{Q^{\pm}_i}\coloneqq \vec{Q^{+}_{j^+}}\oplus\vec{A^{+}_{j^+}}}$, which the binary vector addition is taken to be such that $0\oplus0=0$ but $0\oplus1=1\oplus0=1\oplus1=1$. Update the binary matrix $\hat{Q}$ to be equal to $\hat{Q}^{\pm}$ joined with $\hat{Q}^0$. The form of binary addition is chosen in order to preserve the property $Q_{j,m}=\begin{cases} 0 & \text{if } \vec{V_m}.\vec{A_j}=0 \\ 1 & \text{otherwise} \end{cases}$. The key idea is that any extremal point which does not saturate $\vec{A^{+}_{j^+}}$ or which does not saturate $\vec{A^{-}_{j^-}}$ will, either way, surely not saturate $\vec{A^{\pm}_i}$. 

%Redundancy elimination is now rapidly accomplished by identifying indices of redundant rows in $\hat{Q}$ and then deleting those rows from both $\hat{Q}$ and $\hat{A}$. If (and only if) the set of extremal points which do not saturate $\vec{A_j}$ comprise a superset of the points which do not saturate $\vec{A_{j'}}$ then $\vec{A_j}$ is redundant. An equivalent but far more efficient criterion is that if $\vec{Q_{j'}}-\vec{Q_j}$ is entirely nonnegative then  $\vec{A_j}$ is redundant. This can be used to rapidly filter \emph{all} redundant inequalities.

%This algorithm can be thought of as an improvement to Fouerir-Chernikov elimination \cite{Shapot2012,Bastrakov2015}, which uses a similar $\hat{Q}$ to partially filter redundant inequalities.


\begin{table*}[ht]\centering\caption{A comparison of different approaches for constraining the distributions on the pre-injectable sets. The primary divide is producing inequalities, as in the more difficult first three approaches, versus satisfiability which can witness the infeasibility of specific distributions. The approaches subdivide further into nonlinear, linear, and possibilistic variants.}
\begin{tabularx}{\linewidth}{ |C|C|C|C| } 
\toprule
Approach & General problem & Standard algorithm(s) & Difficulty \\
\midrule
Nonlinear quantifier elimination & Real quantifier elimination & Cylindrical algebraic decomposition, see \cite{ChavesPolynomial} & Very hard \\
\hline
Solve marginal problem & Facet enumeration & Fourier-Motzkin~\cite{fordan1999projection,DantzigEaves,Bastrakov2015,BalasProjectionCone,Jones2008}, lexicographic reverse search~\cite{Avis2000lrs} & Hard \\
\hline
Enumerate Hardy-type paradoxes & Hypergraph transversals & See~\citet{eiter_dualization_2008} & Very easy \\
\midrule
Nonlinear satisfiability & Nonlinear optimization & See \cite{BarFT-SMTLIB}, and semidefinite relaxations~\cite{laurent_polynomial_2012} & Easy \\
\hline
Linear satisfiability & Linear programming & Simplex method \cite{Korovin2012ImplementingCRA,Bobot2012SimplexSAT} & Very easy \\
%\hline 
%Possibilistic satisfiability & --- & --- & Super easy
\bottomrule
\end{tabularx}
\end{table*}

% We find the strategy of employing linear programming in concert with the second Chernikov rule to be extremely efficient.

%There are plenty of other algorithms for computing the facets of a polytope from its vertices. The Equality Set Projection (ESP) algorithm~\cite{jones2004equality,JonesThesis2005} could be an interesting algorithm to use in practice, as it starts churning out facets one by one from the very beginning, whereas Fourier-Motzkin only has the ability to generate the entire list of facets all in one, which quickly becomes computationally intractable. So ESP may provide a useful tool for deriving an incomplete list of inequalities on inflation DAGs that are too large for the Fourier-Motzkin elimination to work.
% ideal for handling inflation DAGs, because its computational complexity scales only according to the facet count of the final projection. Our use of larger-and-larger inflation DAGs to obtain causal infeasibility criteria on the same underlying original DAG means that while the complexity of the starting polytope is unbounded, the complexity of the projection is finite. Practically, this suggests that the ESP algorithm could parse the implications due to very large inflation DAGs efficiently. Formally, ESP should require minimal computational overhead to consider a larger inflation DAG relative to considering a much smaller inflation DAG, when the \emph{implications} of the small and large inflations are similar. By contrast, the computation complexity of Fourier-Motzkin (FM) elimination algorithm scales with the number of quantifiers being eliminated. The number of gedankenprobabilities requiring elimination is exponentially related to the number of variables in the inflation DAG. The FM algorithm, therefore, is utterly impractical very for large inflation DAGs.

% Another positive feature of the ESP algorithm is that it commences outputting quantifier-free inequalities immediately, and terminates upon deriving the complete set of inequalities. By contrast, FM works by eliminating one quantifier at a time. Terminating the ESP algorithm before it reaches completion would result in an incomplete list of inequalities. Even an incomplete list is valuable, though, since the causal infeasibility criteria we are deriving are anyways necessary but not sufficient.

% Vertex projection (VP) algorithms are another computational tool which may be used to assist in linear quantifier elimination \cite{Avis2000lrs}. VP works by first enumerating the vertices of the initial polytope (H-rep to V-rep), projecting the vertices, and then converting back to inequalities (V-rep to H-rep). For generic high-dimensional polytopes, the operation of converting from a representation in terms of halfspaces to one in terms of extremal-vertices representations can be computationally costly (high-$d$ H-rep to V-rep). Starting from a vertex representation in a high dimensional space, however, one can immediately determine the vertex representation of the polytope's projection in a lower dimensional space. The projection is along the coordinate axes, so one just ``discards" the coordinate of the eliminated quantifier. To obtain the inequalities which characterize the projected polytope one then applies a convex hull algorithm to the projected vertices (low-$d$ V-rep to H-rep).

% For probability distributions, however, the extremal vertices are precisely the deterministic possibilities. Since the extremal vertices of the initial polytope are easily enumerated, it is possible to avoid the high-$d$ V-rep to H-rep step entirely. There is a one-to-one correspondence between the inflation-DAG's initial generating inequalities and its initial extreme observable probability distributions. 
% We used this V-rep to H-rep technique to project the initial marginals-problem polytope implied by \cref{fig:Tri222} to an intermediate 23-dimensional polytope, where the 23 remaining dimensions correspond to the probabilities pertainining to  pre-injectable sets.  Only then did we apply translate those probabilities into probabilities pertaining to the original DAG, and in so doing we convert linear inflation-DAG inequalities to polynomial inequalities pertaining to the original DAG. We found that the V-rep to H-rep technique, using \textit{lrs} [\href[pdfnewwindow]{http://cgm.cs.mcgill.ca/~avis/C/lrslib/USERGUIDE.html#Installation\%20Section}{\texttt{lrs}}], was orders-of-magnitude faster than FM elimination at obtaining the same result.

% Yet another technique is also possible. Suppose the initial polytope is given by $\brackets{\vec{x},\vec{y}\,|\hat{A}.\vec{x}+\hat{B}.\vec{y}\geq\bm{c}}$, where $y$ are the quantifiers. If we can find any completely nonnegative vector $\bm{w}$ such that $\bm{w}.\hat{B}=\vec{0}$ then we automatically establish the quantifier-free inequality $\bm{w}.\hat{A}.\vec{x}\geq\bm{w}.\bm{c}$. Solving for ``random" nonnegative vectors $\bm{w}$ is easy; solving for all possible solutions is rather more difficult. \citet{BalasProjectionCone} refined this method so that each extremal construction of $\bm{w}$ corresponds to an irredundant inequality in the H-rep description of the projected polytope. Nevertheless, even without utilizing the full projection cone, this technique can be used to rapidly obtain a few quantifier-free inequalities. 

%\section{Optimized Algorithm for Recognizing Redundant Inequalities}\label{sec:redundancy}

% When performing Fourier-Motzkin linear quantifier elimination one must periodically filter out redundant inequalities from the set of linear inequalities. Equivalently, the means identifying redundant halfspace constraints in the description of the polytope. An individual constraint in a set is redundant if it is implied by the other constraints. 

% An individual linear inequality is redundant if and only if it is a \emph{positive} linear combination of the others [Thm. 5.8 in \citealp{fordan1999projection}]\footnote{The ``if" is obvious. The ``only if" is a consequence of Farka's lemma \cite{fordan1999projection}.}. This is related to the V-rep characterization of polyhedral cones: If a cone is defined such that $W_{\hat{M}}\coloneqq\brackets{\vec{x}\,|\exists_{\bm{v}\geq\bm{0}}:\, \hat{M}.\bm{v}=\vec{x}}$ then $\vec{b}\in W_{\hat{M}}$ if and only if the linear system of equations $\hat{M}.\bm{v}=\vec{b}$ has a solution such that all the elements of $\bm{v}$ are nonnegative.  Thus, the computational tool required is one which accepts as input the matrix $\hat{M}$ and the column vector $\vec{b}$ and returns $\vec{b}\in W_{\hat{M}}$ as True or False. 


%It turns out that we can optimize the detection of redundant halfspaces when considering polyhedral cones as opposed to polytopes. Happily, the inequalities that pertain to the nonnegativity of probability describe a polytope which is identically the intersection of a cone with a hyperplane. The cone is given by the usual nonnegativity inequalities, just without defining $\p{}=1$. The hyperplane, then, is exactly $\p{}=1$. As the hyperplane-intersection constraint has no bearing on the quantifiers, we can set it aside, perform the projection, and then re-incorporate the $\p{}=1$ normalization condition after the Fourier-Motzkin procedure has completed.

%Consider a polyhedral cone with halfspace representation $W=\brackets{\vec{x}\,|\hat{A}.\vec{x}\geq \bm{0}}$. Each row in $\hat{A}$ is an inequality. Now consider the polar dual of $W$, namely $W^*=\brackets{\vec{x}\,|\exists_{\bm{v}\geq\bm{0}}:\, \hat{A}^{T}.\bm{v}=\vec{x}}$, in extremal-rays representation. Every halfspace (row of $\hat{A}$) in $W$ is associated with a ray (column of $\hat{A}^{T}$) in $W^*$. Critically, though, every irredundant halfspace in $W$ corresponds to an \emph{extremal} ray in $W^*$. Determining if a given ray is extremal or not is a computationally fast task. A ray in $W^*$ is \emph{not} extremal if it \emph{can} be expressed as a \emph{positive} linear combination of the other rays in $W^*$. \purp{Comment about one-to-one if $W$ is convex. Otherwise this trick detects only some, but not all, of the redundant inequalities. Also, citations needed.}

%Thus, the computational tool required is one which accepts as input the matrix $\hat{M}$ and the column vector $\vec{b}$, and which determines if the linear system of equations $\hat{M}.\bm{v}=\vec{b}$ has any solutions such that all the elements of $\bm{v}$ are nonnegative.

%For our purposes, $\hat{M}$ is the set of all rays \emph{other} than $\vec{b}$, where the columns of $\hat{M}$ are the other rays. 

% Below, we present two possible \textit{Mathematica$^{_{\textit{\tiny\texttrademark}}}$} implementations which assess if a given column $\vec{b}$ can be expressed as a positive linear combination of the columns of $\hat{M}$. The former function is easy to understand, but the latter utilizes efficient low-level code and \textit{Mathematica$^{_{\textit{\tiny\texttrademark}}}$}'s internal error-handling to rapidly recognize infeasible linear programs.

\begin{comment}
\begin{align*}
 &\hspace{-\mathindent}\texttt{PositiveLinearSolveTest}[{M}\_?\texttt{MatrixQ},{b}\_]{:=}
 \texttt{With}[\{{vars}=\texttt{Thread}[\texttt{Subscript}[x,{\texttt{Dimensions}[{M}][[2]]}]]\},
 \\&\texttt{Resolve}[
 \texttt{Exists}[\texttt{Evaluate}[{vars}],\;
 \texttt{AllTrue}[{vars},\;\texttt{NonNegative}],
 \\&\texttt{And}\texttt{@@}\texttt{Thread}[{M}.\texttt{vars}==\texttt{Flatten}[{b}]]]]];
%\\\shortintertext{or, using efficient lower-level functions,}
       \\&\hspace{-\mathindent}\text{or}\quad\texttt{PositiveLinearSolveTest}[{M}\_?\texttt{MatrixQ},{b}\_]\texttt{/;Dimensions}[{b}]\texttt{===}\{\texttt{Length}[{M}],1\}{:=}
       \\&\hspace{-6ex}\texttt{Module}[\{{rowcount},{columncount},{fakeobjective},{zeroescolumn}\},
       \\&\hspace{-5ex}\{{rowcount},{columncount}\}=\texttt{Dimensions}[{M}];
       \\&\hspace{-5ex}{fakeobjective}=\texttt{SparseArray}[\{\},\{{columncount}\},0.0];
       \,{zeroescolumn}=\texttt{SparseArray}[\{\},\{{rowcount},1\}];
       \\&\hspace{-5ex}\texttt{Internal$\grave{}$HandlerBlock}[\{\texttt{Message},\texttt{Switch}[\#1,\texttt{Hold}[\texttt{Message}[\texttt{LinearProgramming::lpsnf},\_\_\_],\_],\texttt{Throw}[\texttt{False}]]\texttt{\&}\},
       \\&\texttt{Quiet}[\texttt{Catch}[
       \\&\quad\texttt{LinearProgramming}[{fakeobjective},{M},\texttt{Join}[{b},{zeroescolumn},2],\texttt{Method}\to \texttt{Simplex}];\texttt{True}
       \\&],\{\texttt{LinearProgramming::lpsnf}\}]]];
\end{align*}
To illustrate examples of a when a positive solution to the linear system exists and when it does not, consider the following two examples:.
\begin{align*}
%&\hspace{-\mathindent}
&\texttt{PositiveLinearSolveTest}[
\begin{pmatrix}
 1 & 0 & 1 \\
 0 & 1 & -1 
\end{pmatrix},
\begin{pmatrix}
 1 \\
 -2 
\end{pmatrix}]\;==\;\texttt{False} 
%\quad\text{and}\quad
\\&\texttt{PositiveLinearSolveTest}[
\begin{pmatrix}
 1 & 0 & 1 \\
 0 & 1 & -2 
\end{pmatrix},
\begin{pmatrix}
 1 \\
 -1 
\end{pmatrix}]\;==\;\texttt{True} 
\end{align*}

If $\hat{A}$ is the matrix who's rows are nonnegativity inequalities, then the following test determines if row $n$ is redundant. 
\begin{align*}
 &\hspace{-\mathindent}\texttt{RedundantRowQ}[A\_?\texttt{MatrixQ},n\_\texttt{Integer}]\text{:=}\texttt{PositiveLinearSolveTest@@Reverse}[\texttt{Transpose/@TakeDrop}[A,\{n\}]].
\end{align*}
Note that a \texttt{True} response from \texttt{RedundantRowQ} indicates that the row $n$ is redundant.

%Obtaining a redundancy-free collection of (convex) polyhedral-cone inequalities is therefore related to obtaining a non-negative factorization of a matrix. \purp{Need citation. Also, give explicit connection between NN factorization and redundancy elimination. Does this connection appears elsewhere in the literature?}
\end{comment}











\section{On Identifying All Coinciding Marginal Distributions}\label{sec:coincidingdetails}

%\purp{T: All the copy bijections that we ever use are also graph isomorphisms, so there's no point in introducing the notion of copy bijection separately}

%By restricting our attention to inflation models, we see that sometimes the distributions of different injectable sets coincide, i.e. $\pdf{\bm{X}}=\pdf{\bm{Y}}$ whenever both $\bm{X}$ and $\bm{Y}$ are injectable, and $\subsim{\bm{X}}=\subsim{\bm{Y}}$. Sometimes sets of random variables which are \emph{not} injectable, however, can also be shown to have necessarily coinciding distributions. Coinciding (not necessarily injectable) marginal distributions among the inflation DAG variables is a consequence of \cref{eq:funcdependences}. For example, we can verify that $\pdf{A_1 A_2 B_1}=\pdf{A_1 A_2 B_2}$ follows from \cref{fig:Tri222}, even though $\brackets{A_1 A_2 B_1}$ and $\brackets{A_1 A_2 B_2}$ are not injectable sets. Here we formalize this type of constraint more generally.

Whenever one considers some inflation DAG, the corresponding inflated causal hypothesis includes the constraint that every copy of a variable in the inflation DAG has the same functional dependence on its parents, i.e. \cref{eq:funcdependences}. In particular, this implies that the single-variable marginal distribution of all copies are equal,
\begin{align}
     \forall i,j \in \mathbb{N}:\; \pfunc{A_i}=\pfunc{A_j}.
\end{align}
As we will see in the following, the inflation hypothesis also implies similar constraints on the marginal distributions of certain \emph{sets} of variables.

Given sets of nodes $\bm{U}$ and $\bm{Y}$ in an inflation DAG $G'$, let us say that a map $\varphi:\bm{U}\to\bm{V}$ is a \tblue{copy isomorphism} if it is a graph isomorphism such that $\varphi(X)\sim X$ for all $X\in\bm{U}$, meaning that $\varphi$ maps every node $X\in\bm{U}$ to a node $\varphi(X)\in\bm{Y}$ that is equivalent to $X$ under dropping the copy index. We have been using the notion of a copy isomorphism previously by writing $\bm{U}\sim\bm{V}$ whenever there exists a copy isomorphism $\varphi:\bm{U}\to\bm{V}$.

% A special case of a bijection is a graph isomorphism, which is a bijection between the nodes of one graph and the nodes and another such that the action of the bijection is to reversibly transform the first graph into the second one.
% A copy isomorphism, therefore, is defined as a graph isomorphism that is also a copy bijection. The same specialization apply to graph automorphism: an inflation DAG can posses many symmetries, i.e. permutations of the nodes such that the permutation leaves the DAG invariant. Generally, however, only a subset of the possible automorphisms are also copy bijections, i.e. copy automorphisms are a subset of graph isomorphism.

Furthermore, we say that a copy isomorphism $\varphi : \bm{U}\to\bm{V}$ is an \tblue{inflationary isomorphism} whenever it can be extended to a copy isomorphism $\Phi : \ansubgraph{\bm{U}}\to\ansubgraph{\bm{V}}$. If one starts with such a $\Phi$, then one can reconstruct $\varphi$ by restricting the domain of $\Phi$ to $\bm{U}$. If the image of this restriction is $\bm{V}$, then one obtains an inflationary isomorphism; that this restriction is indeed a copy isomorphism follows automatically. So in practice, one can either start with $\varphi : \bm{U}\to\bm{V}$ and try to extend it to $\Phi : \ansubgraph{\bm{U}}\to\ansubgraph{\bm{V}}$, or start with such a $\Phi$ and see whether it restricts to a $\varphi:\bm{U}\to\bm{V}$.

%which satisfies two conditions simultaneously: 1) the copy bijection is a graph isomorphism relating $\ansubgraph[G']{\bm{X}}$ to $\ansubgraph[G']{\bm{Y}}$, and 2) the copy bijection takes $\bm{X}$ to $\bm{Y}$.

% re is only one copy isomorphism between the two ancestral subgraphs, namely the bijections which leaves all nodes invariant except for $\{B_1,B_2,Z_1,Z_2\}\leftrightarrow\{B_2,B_1,Z_2,Z_1\}$. This bijection clearly takes $B_1$ to $B_2$, and hence both the requisite conditions are met.

By the very definition of inflation model, the inflation hypothesis implies an equation between marginal distributions:

\begin{lemma}
	If $\varphi:\bm{U}\to\bm{V}$ is an inflationary isomorphism, then $P_{\bm{U}} = P_{\bm{V}}$ in any inflation model, where the variables in $\bm{U}$ and $\bm{V}$ are matched up according to $\varphi$.
	\label{lem:coincide}
\end{lemma}

In order to equate the distribution $P_{\bm{U}}$ with the distribution $P_{\bm{V}}$, one needs to specify a correspondence between the variables that make up $\bm{U}$ and those that make up $\bm{V}$. This is exactly the data provided by $\varphi$. If $\varphi$ is not the identity map, then the resulting equation $P_{\bm{U}} = P_{\bm{V}}$ is nontrivial equation even when $\bm{U}=\bm{V}$: in this case, the equation is effectively requiring $P_{\bm{U}}$ to be invariant under permuting the variables according to $\varphi$.

We illustrate Lemma~\ref{lem:coincide} with an example. For the inflation of~\cref{fig:Tri222}, the map
\[
	\varphi \: : \: A_1 \mapsto A_1,\qquad A_2\mapsto A_2,\qquad B_1\mapsto B_2
\]
is a copy isomorphism between $\bm{U}=\{A_1 A_2 B_1\}$ and $\bm{V}=\{A_1 A_2 B_2\}$, since it maps each copy to a copy of the same type, and trivially implements a graph isomorphism between $\subgraph{\bm{U}}$ and $\subgraph{\bm{V}}$, since neither of these graphs has any edges. There is a unique choice to extend $\varphi$ to a copy isomorphism $\ansubgraph{\bm{U}}\to\ansubgraph{\bm{V}}$ given by each of $Y_2$, $X_1$, and $Y_1$ mapping to itself, as well as $Z_1\mapsto Z_2$. Therefore $\varphi$ is indeed an inflationary isomorphism. From Lemma~\ref{lem:coincide}, we then conclude that any inflation model satisfies $P_{A_1 A_2 B_1} = P_{A_1 A_2 B_2}$.

\begin{figure}[b]
    \centering
    \begin{minipage}[t]{0.2\linewidth}      \centering
    \includegraphics[scale=1]{ISorigDAG.pdf}
    \caption{The instrumental scenario of \citet{pearl1995instrumental}.}
    \label{fig:ISorigDAG}
    \end{minipage}\hfill
    \begin{minipage}[t]{0.3\linewidth}      \centering
    \includegraphics[scale=1]{IScopyDAG.pdf}
    \caption{An inflation DAG of the instrumental scenario which illustrates why coinciding ancestral subgraphs doesn't necessarily imply the coinciding marginal distributions.}
    \label{fig:IScopyDAG}
    \end{minipage}\hfill    
    \begin{minipage}[t]{0.3\linewidth}      \centering
    \includegraphics[scale=1]{ISancestorDAG.pdf}
    \caption{The ancestral subgraph of \cref{fig:IScopyDAG} for either $\{X_2 Y_2 Z_1\}$ or $\{X_1 Y_2 Z_2\}$.}
    \label{fig:ancestralsubgraphnotenough}
    \end{minipage}
\end{figure}

For a non-example, consider the inflation DAG in \cref{fig:IScopyDAG}. The identity map implements a copy isomorphism $\ansubgraph{X_2 Y_2 Z_1}\sim\ansubgraph{X_1 Y_2 Z_2}$, since both of these ancestral subgraphs are the DAG of~\cref{fig:ancestralsubgraphnotenough}. Nevertheless there is no inflationary isomorphism between ${X_2 Y_2 Z_1}$ and ${X_1 Y_2 Z_2}$, since the subgraphs on these two sets of nodes are not even copy isomorphic, i.e.~$\{X_2Y_2 Z_1\}\not\sim\{X_1 Y_2 Z_2\}$.

%In general there may exist multiple inflationary isomorphisms, for example the sets $\brackets{A_1 A_2 B_1}$ and $\brackets{A_1 A_2 B_2}$ can be related by either 
%\begin{align}
%    \begin{pmatrix}
%     A_1 \leftrightarrow A_1 \\
%     A_2 \leftrightarrow A_2 \\
%     B_1 \leftrightarrow B_2 \\
%    \end{pmatrix}
%    \qquad\text{or}\qquad
%    \begin{pmatrix}
%     A_1 \leftrightarrow A_2 \\
%     A_2 \leftrightarrow A_1 \\
%     B_1 \leftrightarrow B_2 \\
%    \end{pmatrix}
%\end{align}
% Two inflation DAGs $G'_1$ and $G'_2$ are said to be inflationarily isomorphic if there exists a graph isomorphism between them which is itself also an inflationary isomorphism. This occurs iff $G'_1\sim G'_2$.
% We say that one isomorphim \tblue{contains} another if the rule-set of the latter is a subset of the rule-set of the former.

% If two sets $\bm{X}$ and $\bm{Y}$ are inflationarily isomorphic, then $P(\bm{X}) = P(\bm{Y})$ in any inflation model.

%In other words, $\pdf{\bm{X}}=\pdf{\bm{Y}}$ if and only if an inflationary graph isomorphism exists between $\ansubgraph[G']{\bm{X}}$ and $\ansubgraph[G']{\bm{Y}}$ which preserves the inflationary isomorphism between $\bm{X}$ and $\bm{Y}$. 

One can try to make use of Lemma~\ref{lem:coincide} when deriving polynomial inequalities with inflation via solving the marginal problem, by imposing $P_{\bm{U}} = P_{\bm{V}}$ as an additional constraint for every inflationary isomorphism $\varphi : \bm{U}\to\bm{V}$ between sets of observable nodes. This is advantageous to speed up to the linear quantifier elimination, since one can solve each of the resulting equations for one of the unknown joint probabilities and thereby eliminate that probability directly without Fourier-Motzkin elimination.

Moreover, one could also hope that these additional equations would result in tighter constraints on the marginal problem, which then results in better polynomial inequalities. However, our computations have so far not revealed any example of such a tightening. In some cases, this lack of impact can be explained as follows.
% We define two ordered sets of observable nodes $\bm{X}$ and $\bm{Y}$ in some inflation DAG $G'$ to be \tblue{irrelevant to the marginal problem} if there exists a copy bijection $\varphi:\nodes{G'}\to\nodes{G'}$ which satisfies two conditions:
%\begin{enumerate}
%	\item $\varphi$ is a graph isomorphism,
%	\item $\varphi$ maps $\bm{X}$ onto $\bm{Y}$.
%\end{enumerate}
Suppose that $\varphi:\bm{U}\to\bm{V}$ is an inflationary isomorphism which is not just the restriction of a copy isomorphism between the ancestral subgraphs, but even the restriction of a copy isomorphism $\Phi':G'\to G'$ of the entire inflation DAG onto itself; in particular, this assumption implies that $\Phi'$ also restricts to a copy isomorphism $\Phi:\ansubgraph{\bm{U}}\to\ansubgraph{\bm{V}}$ between the ancestral subgraphs. In this case, the irrelevance of the additional constraint $P_{\bm{U}} = P_{\bm{V}}$ to the marginal problem for inflation models can be explained by the following argument. Suppose that some joint distribution $P$ solves the unconstrained marginal problem, i.e.~without requiring $P_{\bm{U}} = P_{\bm{V}}$. Now apply $\Phi'$ to the variables in $P$, switching the variables around, to generate a new distribution $P'$. Because the set of marginal distributions that arise from inflation models is invariant under this switching of variables, we conclude that $P'$ is also a solution to the unconstrained marginal problem. Taking the uniform mixture of $P$ and $P'$ is therefore still a solution of the unconstrained marginal problem. But this uniform mixture also satisfies the supplementary constraint $P_{\bm{U}} = P_{\bm{V}}$. Hence the supplementary constrained is satisfiable automatically whenever the unconstrained marginal problem is solvable, which makes adding the constraint irrelevant.

Note that the argument does not apply if the inflationary isomorphism $\varphi:\bm{U}\to\bm{V}$ cannot be extended to a copy isomorphism of the entire inflation DAG. It also does not apply if one uses the conditional independence relations on the inflation DAG as well, since this destroys linearity. We do not know what happens in either of these cases.





\section{Using Inflation to Certify a DAG as ``Interesting"}
By considering all possible $d$-separation triples implied by a given DAG one can infer that certain conditional independence (CI) relations must hold in an observable joint distribution compatible with the given DAG. In the presence of nontrivial latent nodes, the set of \emph{observable} CI relations is a strict subset of the set of all possible CI relations. Part of Ref. \cite{pusey2014gdag} is concerned with identifying DAGs for which satisfying all observable CI relations is \emph{not} a sufficient criterion for compatibility of any observable distribution with a given DAG. \citet{pusey2014gdag} use the term \tblue{interesting} to refer to any DAG which exhibits a discrepancy between the set of observable distributions genuinely compatible with it and the set of observable distributions compatible with merely its observable CI relations.

\citet{pusey2014gdag} derived novel necessary criteria on the structure of a DAG in order for it to be interesting, and they conjectured that their criteria may, in fact, be necessary and sufficient. As evidence, they enumerated all possible DAGs with no more than six nodes satisfying their criteria for further testing. They found only 21 unique classes of potentially interesting DAGs after accounting for symmetry. Of those 21, \citet{pusey2014gdag} further proved that 18 were unambiguously interesting. For each DAG class they certified as interesting, they generated an explicit incompatible distribution which nevertheless satisfied the DAGs observable CI relations. Incompatibility of the constructed distribution was certified by means of entropic inequalities. 

That left three classes of DAGs which \emph{likely} interesting. For each such class \citet{pusey2014gdag} derived all Shannon-type entropic inequalities in two different ways, once by accounting for non-observable CI relations and once without. The existence of \emph{novel} Shannon-type inequalities upon accounting for non-observable CI relations is evidence that the DAG is likely interesting. The only loophole is that perhaps those novel Shannon-type inequalities are actually non-novel non-Shannon-type inequalities implied by the observable CI relations alone.

One way to close the loophole would be to show that the novel Shannon-type inequalities imply constraints beyond some inner approximation to the genuine entropy cone absent non-observable CI relations, perhaps along the lines of Ref.~\cite{weilenmann2016entropic}. Another is to use incompatibility witnesses beyond entropic inequalities to identify some CI-respecting incompatible observable distribution. \citet{piannaar2016interesting} accomplished precisely this, and should be credited with the original insight to explicitly consider the different values that an observable root variable might take. 

We here demonstrate how the inflation DAG technique can be used for this purpose. 
\begin{figure}[b]
\centering
\begin{minipage}[b]{0.4\linewidth}
\centering
\includegraphics[scale=1]{scen15DAG.pdf}
\caption{DAG \#15 in Ref. \cite{pusey2014gdag}.}\label{fig:GDAG15}
\end{minipage}
\hfill
\begin{minipage}[b]{0.5\linewidth}
\centering
\includegraphics[scale=1]{scen15InflationDAG.pdf}
\caption{A useful inflation of \cref{fig:GDAG15}.}\label{fig:Inflated15}
\end{minipage}
\end{figure}

The following representative polynomial inequalities follow from Hardy-type derivations, per \cref{sec:TSEM}. 
\begin{align}
 0
\leq
{\p[A E]{0 1} \p[A F]{0 1} -\p[A]{0} \p[A D E]{0 0 1} + \p[A]{0} \p[A D F]{0 0 0}} \\
 0
\leq
{\p[A F]{1 0} \p[A E]{0 0} - \p[A]{0}\p[A D F]{1 0 0} + \p[A]{1} \p[A D E]{0 0 1}} \label{eq:DAG15ineqs}
\end{align}

A distribution incompatible with \cref{fig:GDAG15} proposed by \citet{piannaar2016interesting} is as follows.

\begin{align}\label{eq:pienaardistro}
P^{\text{JP}}_{A D E F}=\frac{[1101]+[0110]+[1000]+[0000}{4},\quad\text{i.e.}\quad P^{\text{JP}}_{A D E F}(a d e f)=\begin{cases}\tfrac{1}{4}&\text{if }\;  e\eql \SmallNamedFunction[2]{mod}{a\cramp{\cdot} d\cramp{+}d} \land f\eql a\cramp{\cdot} d , \\ 0&\text{otherwise}.\end{cases}
\end{align}

This distribution is rejected \cref{eq:DAG15ineqs}, which may be explicitly derived as follows. One Hardy-type probabilistic inequality required for consistency of marginal distributions is
\begin{align}\label{eq:hardyforpienaar}
     \pdf[A_1 A_2 D F_1]{1000} \leq \pdf[A_1 A_2 D E_2]{1101}+ \pdf[A_1 A_2 E_2 F_1]{1100}.
\end{align}
Applying factorization consistent with the inflation DAG in we obtain the precursor to \cref{eq:DAG15ineqs}, namely
\begin{align}
 \p[A_2]{0} \p[A_1 D F_1]{100} \leq \p[A_1 F_1]{10} \p[A_2 E_2]{00} + \p[A_1]{1} \p[A_2 D E_2]{001}.    
\end{align}
%which is equivalent to \cref{eq:DAG15ineqs}.

Interesting, the distribution $P^{\text{JP}}$ can also be certified incompatible with \cref{fig:GDAG15} using only our results from the Triangle scenario. By marginalizing over the random variable $A$ in \cref{eq:pienaardistro} we obtain tripartite distribution which is rejected by inequality $\#38$ in the table of inequalities on \cpageref{page:nontrivlist}. This is the same inequality which rejects the W-distribution per \cref{eq:wdistribution1}.

It is worth noting that  $P^{\text{JP}}$ not only satisfies all Shannon-type entropic inequalities pertinent to \cref{eq:pienaardistro}, but lies within an inner approximation to the genuine entropy cone for that scenario\citet[private correspondence]{weilenmann2016entropic}. In other words, a distribution with the same entropic profile as  $P^{\text{JP}}$ \emph{is} compatible with \cref{fig:GDAG15}.



%\purp{T: I've commented this out since we're not concerned here with observational equivalence, and I have the feeling that it would still require a major clean-up operation. The example of 6 observationally equivalent DAGs is not very good since they are all equivalent to A <-- S --> B, which doesn't contain any latent nodes}
\begin{comment}
\clearpage\section{Recognizing observationally equivalent DAGs}

%\purp{Notes to self: Comment about matching-up latent variables between causal structures, for ObsEquiv test.}

%Without loss of generality we herein consider only deterministic DAGs where all latent variables are parentless. \purp{Either prove this, or remove it. If not invoked we should discuss adding edges TO latent variables.}

One expects that an edge $A\to B$ can be added to DAG $G$ while leaving $G$ observationally invariant if the new connection does not introduce any new information about observable variables to $B$. %If the added connection does inform $B$ about some observable $C$, that's still ok so long as the new informational cannot be exploited to increase the correlation between $B$ and $C$.
We can formalize this notion in the language of sufficient statistics. To do so, however, a few background definitions are in order.

\tblue{Perfectly Predictable:} The random variable $X$ is perfectly predictable from a set of variables $\bm{Z}$, hereafter $\mblue{\bm{Z}\vDash X}$, if $X$ can be completely inferred from knowledge of $\bm{Z}$ alone. In a deterministic DAG, for example, every non-root node is perfectly predictable given its parents, ${\NamedFunction{pa}{\!X\!}\vDash X}$. Indeed, in a deterministic DAG the node $X$ is perfectly predictable from $\bm{Z}$ if $X$ is a deterministic descendant of $\bm{Z}$. Operationally, $X$ is a deterministic descendant of $Z$ if the intersection of {[the ancestors of $X$]} with {[the non-ancestors of $Z$]} is a subset of {[the descendants of $Z$]}. Happily though, perfectly predictability can be extrapolated from a causal structure with minimal effort: ${\bm{Z}\vDash X}$ if every directed path to $X$ from any root node is blocked by $\bm{Z}$. 

\tblue{Markov Blanket:} The Markov Blanket for a set of nodes $\bm{V}$, hereafter $\mblue{\NamedFunction{MB}{\!\bm{V}\!}}$, is the set of all of $\bm{V}$'s children, parents, and co-parents. The Markov Blanket is so defined because the nodes in $\bm{V}$ are conditionally independent of \emph{everything} given $\NamedFunction{MB}{\!\bm{V}\!}$. If the random variables in the Markov Blanket $\NamedFunction{MB}{\!\bm{V}\!}$ are known, then information about nodes inside $\bm{V}$ has no bearing on nodes outside the Markov Blanket and vice versa.

\tblue{Markov Partition:} \purp{New! I made this up Nov 24. Useful do you think?} A set of variables $\bm{Z}$ is a Markov Partition for a pair of random variables $X$ and $Y$, hereafter $\mblue{X\cramp{\dashv}\bm{Z}\cramp{\vdash}Y}$, if the pair are conditionally independent of eachother given \emph{any superset} of $\bm{Z}$. Operationally, this means that $X$ and $Y$ are $d$-separated by every superset of $\bm{Z}$. Equivalently, ${X\cramp{\dashv}\bm{Z}\cramp{\vdash}Y}$ if $\NamedFunction{MB}{\!\bm{V}\!}\subseteq \bm{Z}$ and $X\in \bm{V}$ while $Y\not\in \bm{V}$, or if $\NamedFunction{MB}{\!\bm{V}\!}\subseteq \bm{Z}$ and $Y\in \bm{V}$ while $X\not\in \bm{V}$. Happily though, Markov Partitions can be extrapolated from a causal structure with minimal effort: ${X\cramp{\dashv}\bm{Z}\cramp{\vdash}Y}$ if and only if $X$ and $Y$ would be in \emph{disconnected components} under the deletion of all edges initiation from $\bm{Z}$. 

\tblue{Sufficient Statistic:} A set of nodes $\bm{Z}$ is a sufficient statistic for $A$ relative to $X$, hereafter $\mblue{\bm{Z}\vdash A|X}$,
%$\bm{Z}\in\NamedFunction{SS}{\!A|X\!}$, 
if and only if all inferences about $X$ which can be made given knowledge of $A$ are also inferable \emph{without} knowing $A$ but with knowing $\bm{Z}$ instead. In other words, learning $A$ can never teach anything new about $X$ if $\bm{Z}$ is already known. If $X=A$, then the \emph{only way} $\bm{Z}$ can stand in for $A$ when making inferences about $A$ is if $A$ is perfectly predicable given $\bm{Z}$, i.e. ${\bm{Z}\vdash A|A\iff \bm{Z}\vDash A}$. If $A\neq B$ then there are four \purp{and only four?)} ways that $\bm{Z}\vdash A|X$ can be implied by a DAG: If $\bm{Z}\vDash A$, if $\bm{Z}\vDash X$, if $\NamedFunction{MB}{\!\bm{V}\!}\subseteq \bm{Z}$ and $A\in \bm{V}$ while $X\not\in \bm{V}$, or if $\NamedFunction{MB}{\!\bm{V}\!}\subseteq \bm{Z}$ and $X\in \bm{V}$ while $A\not\in \bm{V}$. \purp{Alternatively:} If $A\neq B$ then there are THREE ways that $\bm{Z}\vdash A|X$ can be implied by a DAG: If $\bm{Z}\vDash A$, if $\bm{Z}\vDash X$, and if ${A\cramp{\dashv}\bm{Z}\cramp{\vdash}X}$.

\begin{theorem}\label{theo:edgeadding}
An edge $A\to B$ can be added to $G$ without observational impact if $\NamedFunction{pa}{\!B\!}$ are a sufficient statistic for $A$ relative to all observable nodes, i.e. $\forall{\text{observable }X}:\NamedFunction{pa}{\!B\!}\vdash A|X$.\\
In particular, the edge $A\to B$ can always be added whenever $\NamedFunction{pa}{\!B\!}\vDash A$, including, but not limited to, the instance  $\NamedFunction{pa}{\!A\!}\subseteq \NamedFunction{pa}{\!B\!}$.\\
Furthermore, the edge $\Lambda\to B$ can be also always be added whenever $\Lambda$ is latent and $\NamedFunction{MB}{\!\Lambda\!}\subseteq\NamedFunction{pa}{\!B\!}$.
\end{theorem}

We can also define an analogous condition for when an edge can be removed from a DAG without impacting it observationally.
\begin{corollary}\label{cor:edgedropping}
An edge $A\to B$ can be dropped from $G$ to form $G'$ such that $G$ and $G'$ are observationally equivalent if \sout{and only if} the edge $A\to B$ can be added (back) to $G'$ while leaving $G'$ observationally invariant per \cref{theo:edgeadding}.
\end{corollary}

On the subject of adding observationally-invariant edges, it is important to recognize when latent nodes can be introduced (or dropped) without observational impact.
\begin{theorem}\label{theo:latentadding}
A (root) latent node $\Lambda$ can be removed from $G$ without observational impact if $\Lambda$ has only one child node and no co-parents ($\Lambda$ is ``equivalent to local randomness"), or if $\Lambda$'s children are also all children of another single latent node ($\Lambda$ is ``covered-for by another latent node"). Conversely, a new root latent node $\Lambda$ can be introduced along with various outgoing edges, without observational impact, if $\Lambda$ would be equivalent to local randomness or covered-for by another latent node.
\end{theorem}

\clearpage
Naturally, two causal structures are observationally equivalent if one can be transformed into the other without observational impact, via \cref{theo:edgeadding,theo:latentadding}. Some examples of observationally equivalent scenarios, and the steps which interconvert them, are given in \cref{fig:equivalences}.
\begin{figure}[hb]
\centering
\includegraphics[width=\linewidth]{ObservationalEquivalencesExamples.pdf}
\caption{A set of observational equivalent causal structures. The reasons the changes are observational invariant are as follows: \\
(a)$\sim$(b) because $\Gamma$ is useless in (b), and as such $\Gamma$ can be dropped from (b) per \cref{theo:latentadding}.\\
(b)$\sim$(c) because $\NamedFunction{MB}{\!\Gamma\!}\subseteq\NamedFunction{pa}{\!A\!}$ in (b), and as such $\Gamma\to A$ can be added to (b) per \cref{theo:edgeadding}.\\
(c)$\sim$(d) because $\Lambda$ is redundant to $\Gamma$ in (c), and as such $\Lambda$ can be dropped from (c) per \cref{theo:latentadding}.\\
(d)$\sim$(e) because $\NamedFunction{pa}{\!S\!}\subseteq\NamedFunction{pa}{\!A\!}$ in (e), and as such $S\to A$ can be added to (e) per \cref{theo:edgeadding}.\\
(e)$\sim$(f) because $\NamedFunction{pa}{\!A\!}\subseteq\NamedFunction{pa}{\!S\!}$ in (e), and as such $A\to S$ can be added to (e) per \cref{theo:edgeadding}.
}\label{fig:equivalences}
\end{figure}

%Recall now the two steps of the transformation $\mathsf{ReduceToPCC}$. Imagine after the first step is finished, that an edge $A\to B$ remains, where $A$ is not a root node. Have directly connected all causal pathways, we know that $\NamedFunction{pa}{\!A\!}\subseteq\NamedFunction{pa}{\!B\!}$. As such, $\NamedFunction{pa}{\!B\!}\vDash A$, that is to say, $A$ is perfectly predictable given the parents of $B$. By \cref{theo:edgeadding}, therefore, if the edge $A \to B$ were not in the DAG, we would be able to add that edge without observational impact. By \cref{cor:edgedropping}, therefore, removing that edge has no observational impact. Indeed, the second step of $\mathsf{ReduceToPCC}$ leaves the post-first-step DAG observationally invariant. This allows us to quickly determine if a DAG is PCC-lossless.

%\begin{prop}\label{prop:PCClossless}
%A causal structure $G$ is PCC-lossless if every new edge in $\NamedFunction{ReduceToPCC}{\!G\!}$ relative to $G$ can be accounted for by adding edges to $G$ while leaving $G$ observationally invariant, pursuant to %\cref{theo:edgeadding,theo:latentadding}.
%\end{prop}

%%%%%%%%%%%% Enumeration via lowercase letters
\renewcommand{\labelenumi}{(\alph{enumi})}
\renewcommand{\theenumi}{(\alph{enumi})}
\renewcommand{\labelitemi}{$\circ$}
\end{comment}



\section{The Copy Lemma and Non-Shannon type Entropic Inequalities}\label{sec:NonShannon}

As it turns out, the inflation DAG technique is also useful outside of the problem of causal inference. As we argue in the following, inflation is secretly what underlies the \tblue{Copy Lemma} in the derivation of non-Shannon type entropic inequalities~\cite[Chapter~15]{yeung_network_2008}. The following formulation of the Copy Lemma is the one of \citet{kaced_equivalence_2013}.

\begin{lemma}
	Let $A$, $B$ and $C$ be random variables with distribution $P_{ABC}$. Then there exists a fourth random variable $A'$ and joint distribution $P_{AA'BC}$ such that:
	\begin{enumerate}
		\item $P_{AB} = P_{AB'}$,
		\item $A' \perp AC \:|\: B$.
	\label{copylemma}
	\end{enumerate}
\end{lemma}



\begin{proof}
	Consider the original DAG of~\cref{fig:beforecopy} and the associated inflation DAG of~\cref{fig:aftercopy}. If the original distribution $P_{ABC}$ is compatible with~\cref{fig:beforecopy}, then the associated inflation model marginalizes to a distribution $P_{A_1 A_2 B C}$ which has the required properties. To see that every $P_{ABC}$ is compatible with~\cref{fig:beforecopy} one may take $X$ to be any \tblue{sufficient statistic} for the joint variable $(A,C)$ given $B$, such as $X := (A,B,C)$.
\end{proof}

While it is also not hard to write down a distribution with the desired properties explicitly~\cite[Lemma~15.8]{yeung_network_2008}, our purpose of rederiving the lemma via inflation is our hope that more sophisticated applications of the inflation technique will result in \emph{new} non-Shannon type entropic inequalities. For example, all the non-Shannon type inequalities derived by \citet{zeger_2011_nonshannon} are applications of the Copy Lemma, and each proof therein can be recast in terms of some Shannon type inequalities appied to a suitable inflation.

\begin{figure}[H]
\centering
\begin{minipage}[t]{0.25\linewidth}
\centering
\includegraphics[scale=1]{shannonNOcopyV1.pdf}
\caption{A causal structure that is compatible with any distribution $P_{ABC}$.}\label{fig:beforecopy}
\end{minipage}
\hfill
\begin{minipage}[t]{0.3\linewidth}
\centering
\includegraphics[scale=1]{shannonYEScopyV1.pdf}
\caption{An inflation of \cref{fig:beforecopy}. Not that this inflation is non-broadcasting. The Copy Lemma is therefore valid even in the paradigm of quantum mechanics or any generalized probability theory, as we would expect.}\label{fig:aftercopy}
\end{minipage}
\hfill
\begin{minipage}[t]{0.3\linewidth}
\centering
\includegraphics[scale=1]{shannonYEScopyKacedV1.pdf}
\caption{An equivalent representation of inflation of \cref{fig:aftercopy} so as to match the terminology of \citet{kaced_equivalence_2013}.}
\end{minipage}
\end{figure}

\section{Classifying polynomial inequalities for the Triangle scenario}
\label{sec:38ineqs}

The following polynomial inequalities for the Triangle scenario have been derived via the linear quantifier elimination method of~\cref{sec:ineqs} using the inflation DAG of~\cref{fig:Tri222}. Initially this has resulted in 64 symmetry classes of inequalities, where the symmetries are given by permuting the variables and inverting the outcomes. For the resulting 64 inequalities, numerical checks have found violations of only 38 of them: although they are all facets of the marginal polytope over the distributions on pre-injectable sets, there is no guarantee that they are also nontrivial inequalities at the level of the original DAG, and this has indeed turned out not to be the case for 26 of these symmetry classes of inequalities. Moreover, it is still likely to be the case that some of these inequalities are redundant; we have not yet checked whether for every inequality there is a distribution which violates the inequality but satisfies all others.

In the following table, the inequalities are listed in expectation-value form, where we assume the two possible outcomes of each variables to be $\{-1,+1\}$. Each row in the table gives the coefficients with one inequality, which is then $\geq 0$.

\begin{table*}[ht]\centering\caption{List of inequalities as table of coefficients.}
\resizebox{\textwidth}{!}{
\begin{tabular}{c@{\hspace{1em}}ccc@{\hspace{1em}}ccc@{\hspace{1em}}c@{\hspace{1em}}ccc@{\hspace{1em}}ccc@{\hspace{1em}}c} 
%\begin{array}{rrrrrrrrrrrrrrrr}
  constant & \(\expec{A}\) & \(\expec{B}\) & \(\expec{C}\) & \(\expec{A B}\) & \(\expec{A C}\) & \(\expec{B C}\) & \(\expec{A B C}\) & \(\expec{A}\expec{B}\) & \(\expec{A}\expec{C}\) & \(\expec{B}\expec{C}\) & \(\expec{C}\expec{A B}\) & \(\expec{B}\expec{A C}\) &
   {\(\expec{A}\expec{B C}\)} & {\(\expec{A}\expec{B}\expec{C}\)}   \\\bottomrule
 1 & 0 & 0 & 0 & 1 & 1 & 0 & 0 & 0 & 0 & 1 & 0 & 0 & 0 & 0 \\
 2 & 0 & 0 & 0 & 0 & -2 & 0 & 0 & 0 & 0 & 0 & -1 & 0 & 0 & 1 \\
 3 & 1 & 1 & 1 & 3 & -1 & 0 & 0 & 0 & 0 & -1 & 1 & -1 & 0 & 1 \\
 3 & 1 & 1 & -1 & 3 & 1 & 0 & 0 & 0 & 0 & 1 & -1 & -1 & 0 & 1 \\
 3 & 0 & 0 & 1 & -2 & 0 & -2 & 0 & 1 & 0 & 0 & -1 & -1 & 0 & 1 \\
 3 & 0 & 1 & 0 & 1 & 0 & -2 & 0 & -1 & 1 & 0 & 1 & -1 & 0 & 1 \\
 3 & 0 & 1 & 0 & 1 & 0 & -2 & 0 & 1 & -1 & 0 & 1 & 1 & 0 & -1 \\
 3 & 1 & 1 & 1 & 2 & 2 & 2 & -1 & 1 & 1 & 1 & 1 & 1 & 1 & -1 \\
 3 & 1 & 1 & 1 & 2 & 0 & -2 & 1 & 1 & -1 & 1 & 1 & 1 & -1 & -1 \\
 4 & 0 & 0 & 2 & -2 & -2 & 0 & -1 & 2 & 0 & 2 & 1 & 1 & 1 & 0 \\
 4 & 0 & -2 & 0 & -2 & 0 & -3 & 1 & 0 & 0 & 1 & 1 & -1 & 0 & 1 \\
 4 & 0 & 0 & -2 & -2 & -2 & -3 & 1 & 2 & 0 & 1 & 1 & 1 & 0 & -1 \\
 4 & 0 & 0 & 0 & 2 & -2 & 1 & 1 & 2 & 2 & -1 & 1 & -1 & 0 & -1 \\
 4 & 0 & 0 & 0 & 2 & -2 & 1 & 1 & -2 & 2 & -1 & 1 & 1 & 0 & 1 \\
 4 & 0 & 0 & 0 & -2 & 0 & 3 & 1 & 2 & 0 & 1 & -1 & -1 & 0 & 1 \\
 4 & 0 & 0 & -2 & -2 & -2 & -2 & 1 & 2 & 0 & 0 & 1 & 1 & -1 & 0 \\
 4 & 0 & 0 & 0 & -2 & -2 & -2 & 1 & 2 & 2 & 2 & 1 & 1 & 1 & 0 \\
 5 & 1 & 1 & 1 & 3 & 1 & -4 & 0 & -2 & 0 & 1 & 1 & -1 & 0 & 1 \\
 5 & 1 & 1 & 1 & 3 & -1 & -4 & 0 & 2 & -2 & 1 & 1 & 1 & 0 & -1 \\
 5 & 1 & -1 & 1 & 1 & 2 & -2 & -2 & -2 & -1 & 1 & 1 & -2 & -2 & 0 \\
 5 & 3 & 1 & 1 & 1 & 3 & 1 & -1 & 2 & 0 & 0 & -2 & 0 & 0 & 2 \\
 5 & 1 & 1 & 1 & 1 & 2 & -2 & -1 & 0 & -1 & -1 & 2 & 1 & 1 & -2 \\
 5 & -1 & 1 & 1 & 1 & 1 & -1 & 1 & -2 & -2 & 2 & -2 & -2 & -2 & 0 \\
 5 & 1 & 1 & 1 & 2 & 1 & -1 & 1 & -1 & 0 & 2 & -1 & -2 & -2 & 1 \\
 5 & 1 & 1 & 1 & -1 & 2 & 2 & 1 & -2 & -1 & -1 & 2 & 1 & -1 & -2 \\
 6 & 0 & 0 & 0 & -3 & -4 & 0 & 0 & 1 & 2 & 2 & -1 & -2 & -2 & 1 \\
 6 & 0 & 2 & 0 & 3 & -4 & 0 & 0 & 1 & 2 & 0 & 1 & -2 & -2 & 1 \\
 6 & -2 & 2 & 0 & -3 & -5 & 0 & 0 & 1 & 1 & 0 & 1 & -1 & -2 & 2 \\
 6 & 0 & 0 & 0 & 1 & -3 & 2 & 0 & 1 & 1 & -4 & 1 & -1 & -2 & -2 \\
 6 & 0 & 0 & 2 & 0 & 3 & -5 & 0 & -2 & 1 & 1 & 2 & 1 & 1 & -2 \\
 6 & 0 & 0 & -2 & 2 & -2 & 1 & 0 & -4 & 2 & -1 & 2 & 2 & 1 & 1 \\
 6 & 0 & 0 & 0 & -3 & -2 & -2 & -2 & 1 & 0 & 4 & -1 & -2 & 0 & 1 \\
 7 & 1 & 1 & 1 & 2 & 1 & -3 & 3 & 1 & -2 & 2 & 3 & 2 & -2 & -1 \\
 8 & 0 & 0 & 0 & -4 & -2 & -2 & -3 & 4 & 2 & -2 & 1 & -1 & -3 & 2 \\
 8 & 2 & -2 & 0 & 1 & -6 & 0 & 1 & -1 & 0 & 2 & 2 & 1 & -3 & 3 \\
 8 & 2 & 0 & 0 & 6 & 1 & -2 & 1 & 0 & 1 & 2 & -1 & -2 & -3 & 3 \\
 8 & 0 & -2 & -2 & 0 & -6 & 1 & 1 & 2 & 0 & -1 & 3 & 1 & -2 & -3 \\
 8 & 0 & 0 & 2 & 2 & 1 & -6 & 1 & -2 & 1 & 0 & 3 & 2 & -1 & -3 
\end{tabular}}
\end{table*}

\clearpage\label{page:nontrivlist}
\includepdf[pages=-,scale=0.90]{nontrivlist.pdf}

%\section*{References}
%\nocite{*}
%\setlength{\bibsep}{\smallskipamount}
%\clearpage
\setlength{\bibsep}{3pt plus 3pt minus 2pt}
\bibliographystyle{apsrev4-1}
\nocite{apsrev41Control}
\bibliography{hardyinference}

\end{document}
